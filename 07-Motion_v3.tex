
\newcommand{\sageurlforcurvature}{http://bmw.byuimath.com/dokuwiki/doku.php?id=curvature_calculator}
%{http://aleph.sagemath.org/?z=eJzFl8Fu4zYQhu98ioEdwJKjKE3QUwEdtlug7cUtUN-SdMGItEWEFgWSkuFu9t07JCVbttzESg_rg0Mr_L8ZDmfIUUN1NLOzGE4_0ynjuaSagy04NFQL-iy5gZ2qZ1JCbTg8c6m2KdGRjbOH6G5u48R__33_1CJ-FQ0HChXVdMOtFv9QK1QJauWhea0bTuzPqi6ZySKb3Nwn93HPg78stcH-s5_jjMOWlhZE6cYaKqksqZQobXY3WMLBAXvTUFnjyMK2EHkRQKqWDKR4QRMKmKZboLiyhudWaZMSMp1-VpuqtsFpKzacNBkTq1WkE3sSsWnup7bRwsDkwu4IDdObt6fTPOeSa2-GmIpzBhk0aan0JuqFo3vQSo3XSl6ubeEiSlvPU7J08tsA-m-rdSksWFquOcYzSAlbMoti7_Sy7_ReyZa3zLp5pj9vb-xonoGo22Zq64a3VmLSPsHcQgai9kudHs8Xp2s8h0vJImDsrfvqh2267NbpHtJub93O_qkxZzxPc1NLa1KA34EyhstIjdhUUqx2X1a1lFF8-I0pvEY2potT8pI5pzjNiwSMe0aPkLAVmE9SqRdoCYKzlBR2I1Prqil6IA96cvWIboG-miSukt4z_pQcNA1qmpEaiho6QrPSNP_KzLevzH7DOL8Gw684vAK_65PE_xnpxhLdWI52I0i9L6h32_1xhGkRZgRC8pV9HYIecVJhXUweX2hVUUDyPsdHBmaB2sVlmqeYfOdUrrzpyacw152Q4M6PSeKP5AQmsOVQ0CacOCsl8cIQ5fqnyZtF4MUfKITxulAM43QXF8R4d0JRfMSdM4XxvzCH4hiH-VCBjA9UKJKLdW2hLNQW7xI7M7ChL64tWGtaFaTK9v1J_sU1FO56D03JsC1664O3F95YrpEwvf6mus48NNw9UZfecYI1xWyR_ZjgjS6Vzmbozm6WSFFyY3eSZzNGTcHZLD5lNAeGsVTbrIMOmfqcnl6uX2vOyyFheTnhWdL8ZUhYXE6oal1JPkT00vwCyo670-csxYxZTe086U6-37jmmE_UN6LuBNVOgs1KaAN9JwkNUDxeqc-M0vWgON3h00mHcTt_E3YbawQZOvSACfj445Am4AOJw2UCYS2-7cF2zP2r5u0vk0CIV7aYEGIKtY0q9FdTJmrzx-pzV3XZ3e1pBZ7N6X7LGCBumXspybF95LoP7oJ3PbA5XxwsHYED5BjcxUYXyl8pA9p7hd8BzsA9EAauv0sk01_ca4J3Wehc8mOqe0tQlb9IG653eByUa-DS8NQn3ckp0-bfwItzgRvU3TxXrnG5Pq2muRFl5DIZX6d-SO7nlYiT_vES71PiXyqCqZY=&lang=sage}


\noindent 
This unit covers the following ideas.  
\begin{enumerate}
 \item Develop formulas for the velocity and position of a projectile, if we neglect air resistance and consider only acceleration due to gravity. Show how to find the range, maximum height, and flight time of the projectile.
 \item Develop the $TNB$ frame for describing motion. Explain why $\vec T$, $\vec N$, and $\vec B$ are all orthogonal unit vectors, and how to perform the computations to find these three vectors.
\begin{itemize}
	\item Compute the unit tangent and unit normal vector of space curves
\end{itemize}
 \item Explain the concepts of curvature $\kappa$, radius of curvature $\rho$, center of curvature, and torsion $\tau$. Including what these quantities mean geometrically.
 \item Find the tangential and normal components of acceleration. Show how to obtain the formulas $a_T=\frac{d}{dt}|\vec v|$ and $a_N=\kappa |\vec v|^2=\frac{|\vec v|^2}{\rho}$, and explain what these equations physically imply.
\end{enumerate}


\vskip0.2cm
Table \ref{motion table} summarizes most of the concepts we'll discuss. The goal of this chapter is to explain how the vectors in this table are related.  You'll also find this \href{\sageurlforcurvature}{Sage notebook (click on the link)} can greatly speed up all the computations in this chapter. Prof. Woodruff has also created a YouTube playlist to go along with this section. There are 11 videos, each 4-6 minutes long.
\begin{itemize}
 \item \href{http://www.youtube.com/playlist?list=PL30EE81142B1ED1F0&feature=plcp}{YouTube playlist for 07 - Motion and The TNB Frame}.
 \item \href{http://db.tt/FmEGk9p5}{A PDF copy of the finished product} (so you can follow along on paper).
\end{itemize}

\clearpage

\vspace*{-1cm}
\begin{table}[h]
\begin{tabular}{|c|c|c|}
\hline
Quantity & Symbol & Formula\\\hline\hline
Position (``r''adial vector) & $\vec r$ & $\vec r(t) = (x(t),y(t),z(t))$\\\hline
Velocity  & $\vec v$ & $\ds \vec v(t) = \frac{d\vec r}{dt}$\\\hline
Speed  & $v = \ds\frac{ds}{dt}$ & $\ds v(t) = |\vec v(t)|$\\\hline
Acceleration  & $\ds \vec a$ & $\ds \vec a(t) = \frac{d \vec v}{dt}= \frac{d^2\vec r}{dt^2}= \frac{d}{dt}\frac{d\vec r}{dt}$\\\hline
Unit Tangent Vector & $\vec T$ & $\ds\frac{d\vec r}{ds} = \frac{d\vec r/dt}{ds/dt} = \frac{\vec r^\prime(t)}{|\vec r^\prime(t)|}$\\\hline
Curvature Vector & $\vec \kappa $& $\ds\frac{d\vec T}{ds} =\frac{d\vec T/dt}{ds/dt} = \frac{d\vec T/dt}{|\vec v|} = \frac{\vec T^\prime(t)}{|\vec r^\prime(t)|} $\\\hline
Curvature (a scalar)& $ \kappa $&$\ds \left|\frac{d\vec T}{ds}\right| =\left|\frac{d\vec T/dt}{ds/dt}\right| = \frac{\left|d\vec T/dt\right|}{|\vec v|}= \frac{|\vec T^\prime(t)|}{|\vec r^\prime(t)|}  $ \\\hline
Curvature of $y=f(x)$& $ \kappa(x) $&$\ds \kappa(x) = \frac{|f''(x)|}{(1+(f')^2)^{3/2}}.  $ \\\hline
Principal unit normal vector & $ \vec N$& $\ds \frac{d\vec T/dt}{|d\vec T/dt|} =  \frac{\vec T^\prime(t)}{|\vec T^\prime(t)|}=\frac{1}{\kappa}\frac{d\vec T}{ds} = \frac{1}{\kappa |\vec v|}\frac{d\vec T}{dt}$\\\hline
Binormal vector & $ \vec B$& $ \vec T\times\vec N$\\\hline
Radius of curvature & $ \rho$ & $1/\kappa$\\\hline
Center of curvature &  & $\vec r(t)+\rho(t)\vec N(t)$ \\\hline
Torsion & $ \tau $ & $\ds \pm\left|\frac{d\vec B}{ds}\right|$ (pick the sign) or $\ds-\frac{d\vec B}{ds}\cdot \vec N $\\\hline
Tangential Component of acceleration & $ a_T$ & $\ds \vec a \cdot \vec T = \frac{d}{dt}|\vec v|$\\\hline
Normal Component of acceleration & $ a_N$ & $\ds \vec a \cdot \vec N = \kappa \left(\frac{ds}{dt}\right)^2 = \kappa |\vec v|^2$\\\hline
Acceleration (sum the components)& $ \vec a$ & 
$\vec a 
= a_T\vec T+a_N\vec N 
= \left(\frac{d}{dt}|\vec v|\right) \vec T 
 +\left(\kappa |\vec v|^2\right) \vec N  $\\\hline


\end{tabular}
\caption{This table summarizes the key ideas in this unit. Most of our work in this unit will be to explain the connections between these variables.\label{motion table}} 
\end{table}
\normalsize


\clearpage

\section*{Optional: Projectile Motion}

Have you ever dropped a rock from the top of a waterfall, or skipped a rock across a lake. This section explores some simple connections between position, velocity, and acceleration. If we wanted to send a rocket to space, or shoot a missile across an ocean, the same principles will apply. If we know how much thrust a rocket provides (the acceleration), can we determine the velocity of our rocket at any time along its path?  Could we predict the flight path of the rocket? To make a good flight plan, we'd need to know how to determine position and velocity from acceleration.  That's the content of this section.  

\begin{review*}
 If $y'(t) = 3t^2+12e^{2t}$ (the velocity) and $y(0)=2$ (initial height), then what is $y(t)$? See footnote \footnote{Integrate to get $y(t) = t^3+6e^{2t}+C$. Since $y(0)=2$, we know $2=0+6(1)+C$, which gives $C=-4$. So the height is $y(t) = t^3+6e^{2t}-4$. } for an answer. 
\end{review*}


To solve the next exercise, we need to know that acceleration is the derivative of velocity, and that velocity is the derivative of position.  These facts hold true for vector-valued functions as well.


\begin{problem}
Consider a rocket in space (so we can neglect air resistance and gravity). The rocket's boosters apply an acceleration $\vec a(t) = (2t,-8)$ m/s$^2$. The rocket's initial velocity is $\vec v(0) = (4,5)$ m/s.  The initial position is $\vec r(0) = (1,16)$ m. Use this information to determine the position of the object after 2 seconds, and after 3 seconds. 

[Hint: Integrate each component to get velocity, then repeat to get position. Don't forget the 4 arbitrary constants you get from integration. Use the initial velocity and initial position to determine these constants.]
\end{problem}

Suppose we fire a projectile (like a pumpkin) from a cannon. The projectile leaves the cannon with an initial speed $v_0$, at an angle of $\alpha$ above the $x$-axis. All the motion in this exercise occurs within a plane, and we'll use $x$ and $y$ to represent motion in that plane. Our goal is to find the velocity $\vec v(t)$ and position $\vec r(t)$  of the projectile at any time $t$. 

We need some assumptions prior to solving. 
\begin{itemize}
 \item Assume the only force acting on the object is the force due to gravity. We will neglect air resistance. 
 \item The force due to gravity is the mass of the projectile multiplied by the acceleration of gravity. The mass of the object will not be important in our work here, though in future classes you may study how mass affects energy computations. 
 \item The projectile is shot over a small enough range that we can assume gravity only pulls the object straight down.
 \item Most branches of science use the letter $g$ to represent the magnitude of the vertical component of acceleration, so we can write the acceleration of the projectile as 
$$\vec a(t) = (0,-g) \quad \quad \text{or}\quad \quad \vec a(t)= 0{\bf i}-g{\bf j}.$$ 
 \item Our text uses the approximations $g\approx 9.8$ m/s$^2$ or $g\approx32$ ft/s$^2$. 
\end{itemize}


You've probably heard before that when you throw a baseball to a friend, the path of the baseball is parabolic. The next exercise proves this. If you feel shaky on getting a Cartesian equation from a parametrization, please tackle this review exercise, otherwise, jump straight to the exercise.
\begin{review*}
 The function $\vec r(t) = (2t+3, 4t^2+7t+5)$ is a parametrization of a plane curve.  Give a Cartesian equation of the curve. 
 See \footnote{Since $t=\dfrac{x-3}{2}$, a Cartesian equation is $y = 4\left(\dfrac{x-3}{2}\right)^2+7\left(\dfrac{x-3}{2}\right)+5$. } for an answer.
\end{review*}


\begin{problem}
\marginpar{Watch a \href{http://www.youtube.com/watch?v=dW0bm7cLB8E&list=PL30EE81142B1ED1F0&index=1&feature=plpp_video}{YouTube video}.}%
\marginpar{ 
\thomasee{You can practice finding position from velocity and acceleration with exercises 13.2: 11-18, and especially 13.2: 29.}
\stewarts{You can practice finding various combinations of position, velocity and acceleration vectors with exercises: 13.4:9-18, particularly focus on 15-18}
}
Suppose a projectile is fired from the point $(x_0,y_0)$ with an initial velocity $\vec v(0)=(v_{x_0},v_{y_0})$, and that gravity is the only force acting on the object. This means the acceleration on the object is $\vec a(t) = (0,-g)$.
\begin{enumerate}
 \item Show that the velocity at any time $t$ is $\vec v(t) = (c_1,-gt+c_2)$. What are $c_1$ and $c_2$? Explain
 \item Show that the position at any time $t$ is $\vec r(t) = (v_{x_0}t+c_3, -\frac{1}{2}gt^2+v_{y_0}t+c_4)$. What are $c_3$ and $c_4$? 
% \item Give parametric equations $x=x(t)$ and $y=y(t)$ that give the horizontal and vertical position of the projectile at time $t$. 
 \item Eliminate the parameter $t$ to give a Cartesian equation of the projectile's path. This will prove that the path of the particle is parabolic.

[Hint: Use substitution.]
\end{enumerate}
\end{problem}

If a projectile starts at $(x_0,y_0)$, we can move the origin to this point. As long as we are not trying to gauge the location of two projectiles simultaneously, we could always make the origin $(0,0)$ our starting point.   
We make the following definitions for a projectile that starts at $(0,0)$ and hits the ground at $(R,0)$.
\begin{itemize}
 \item The range is the horizontal distance $R$ traveled by the projectile.  
 \item The flight time is how long the projectile is in the air. It is the time $t$ at which $\vec r(t)=(R,0)$.
 \item The maximum height is the largest $y$ value obtained by the projectile. 
\end{itemize}


\begin{problem}\marginpar{Watch a \href{http://www.youtube.com/watch?v=a6PHAvynNWM&list=PL30EE81142B1ED1F0&index=2&feature=plpp_video}{YouTube video}.}%
 Answer the following questions. Assume we fire a projectile from the origin, which means the acceleration, velocity, and position are 
$$ \vec a(t) = (0,-g),\quad 
\vec v(t) = (v_{x_0}, -gt+ v_{y_0}),\quad
\vec r(t) = (v_{x_0}t, -\frac12 gt^2+ v_{y_0}t)
.$$
\begin{enumerate}
 %\item What should the velocity vector equal when the object has reached the maximum height?
 \item What's the time to max height?  What's the flight time? 
 \item Show why the maximum height is $\ds y_{\max}=\frac{v_{y_0}^2}{2g}$ and the range is $\ds R=\frac{2v_{x_0}v_{y_0}}{g}$.
 \item If the initial speed is $v_0$, with a firing angle of $\alpha$ above the horizontal, rewrite $v_{x_0}$ and $v_{y_0}$ in terms of $v_0$ and $\alpha$, and then state the range in terms of $v_0$ and $\alpha$. 
\end{enumerate}
\end{problem}

\bmw{The next exercise comes from your text. (See section 13.2.)  Try it without reading the text.  It's a fun application of the ideas above.}
\begin{problem}
\marginpar{This exercise was created around the opening ceremony of the Barcelona Spain Olympics.  Antonio Rebollo was the archer, but he didn't try to hit the flame at the peak of the flight. You can \href{http://www.youtube.com/watch?v=b5gZeT4TVds}{watch a YouTube video} of the opening ceremony by following the link.}%
\marginpar{
\thomasee{See 13.2: 19-28 for more practice.}
\stewarts{See 13.4:19-36 for a variety of application exercises} 
}

 An archer stands at ground level and shoots an arrow at an object which is 90 feet away in the horizontal direction and 74 ft above ground. The arrow leaves the bow at about 6 ft above ground level (not the origin). 
 The archer wants the arrow to hit the target at the peak of its parabolic path. 
 For the purposes of this exercise, Let $g = 32 \text{ft}/\text{s}^2$. 
 What initial speed $v_0$ and firing angle $\alpha$ are needed to achieve this result? 
 [Hint: This is much easier to solve if you first find $v_{x_0}$ and $v_{y_0}$, the horizontal and vertical components of the velocity. You may want to move the origin as well, so that you can use the formulas from above.]
\end{problem}

%\newpage
\instructor{Old Day split occurred here.}
%
%\stepcounter{unitday}
%\uday
%\normalsize


\section{Arc Length and the Unit Tangent Vector}
This section will...
\begin{itemize}
\item Motivate and define the \textbf{Unit Tanget Vector}
\end{itemize}


The previous section developed a way to predict position and velocity from acceleration, let's look more in depth at the actual path taken by a projectile. We'll need to be able to compute the actual distance an object travels (not the displacement, but the distance).  This requires that we study arc length.  This was covered for curves in the plane in chapter's 3 and 4. (See \ref{arc length formula})

\begin{review*}
 A horse runs once around an elliptical track, which is parametrized by $\vec r(t) = (3\cos t,4\sin t)$.  Set up, do not solve, an integral formula that tells us the distance the horse traveled. What's the displacement? See 
\footnote{The velocity is $\vec v(t) = (3\sin t, -4\cos t)$. The speed is $v(t) = \sqrt{9\sin^2t+16\cos^2t}$. The distance traveled is the arc length $\ds s=\int_0^{2\pi} \left(\sqrt{9\sin^2t+16\cos^2t}\right)dt$. Since the horse's initial and final position are equal, the displacement is zero. Arc length does not equal displacement. }
for an answer.
\end{review*}
 

Let's now develop a formula for the arc length of a space curve, a curve in 3D. We can always parameterize a space curve with $\vec r(t) = (x,y,z)$ (one input, 3 outputs).\\

Remember: When we talk about speed and velocity it is importation to recognize their differences. Velocity has direction and (often components) while speed is the total movement or the magnitude of the velocity.\\

\begin{problem}
\marginpar{Watch a \href{http://www.youtube.com/watch?v=jZpAU2T6iI4&list=PL30EE81142B1ED1F0&index=3&feature=plpp_video}{YouTube video}.}
A space ship travels through the galaxy. Let $\vec r(t) = (x,y,z)$ 
\marginpar{Technically, we should write $\vec r(t) = (x(t),y(t),z(t))$. However, we already know that $x$, $y$, and $z$ depend on $t$, hence we'll just leave the dependence on $t$ off.}%
be the position of the space ship at time $t$, with the earth at the origin $(0,0,0)$. 
\begin{enumerate}
 \item What are the velocity and speed of the space ship at time $t$? You answers should involve some derivatives (such as $\frac{dx}{dt}$).
 \item If the space ship travels for a really small time $dt$, then the speed is about constant. Since distance is speed times time, about how much distance (we'll call it $ds$) will the space ship travel in this short amount of time?
 \item As the ship travels from time $t=a$ to time $t=b$, explain why the distance traveled (the arc length of the path followed) is $$s=\int_a^b |\vec r '(t)|\ dt = \int_a^b \sqrt{\left(\frac{dx}{dt}\right)^2+\left(\frac{dy}{dt}\right)^2+\left(\frac{dz}{dt}\right)^2}\ dt .$$ \label{arc length2}
\end{enumerate}

\end{problem}

In all our work that follows, we want to consider space curves that have nice smooth paths.  What does this mean?  We want to be able to compute tangent vectors at any point, so we will require that a parametrization $\vec r$ be differentiable.  However, this isn't enough.  
\begin{problem}
 We've encountered the polar curve $r = 1-\sin\theta$ before (we called it a cardioid, and it looked like heart).  Recall that we can switch from polar to Cartesian using the coordinate transformation $x=r\cos\theta$ and $y=r\sin\theta$. 
\begin{enumerate}
 \item Draw the curve.
 \item Give the parametrization of this curve: $\vec r(\theta) = ((1-\sin\theta)\cos\theta, ?)$.  
\end{enumerate}
The parametrization is completely differentiable.  
\begin{enumerate}[resume]
	\item Find $\dfrac{d\vec r}{d\theta}$.
	\item You should notice a sharp cusp in the graph. At what $\theta$ does this cusp occur?  What is the value of the derivative $\dfrac{d\vec r}{d\theta}$ at this value of $\theta$. 
\end{enumerate}
\end{problem}

We'd like to avoid paths that contain a cusp, because at a cusp the direction of motion changes rather abruptly. This can happen physically, but it requires the speed of an object to reach zero, the object stops moving, and then the path changes direction. The fact that the speed reaches zero will mean we can't divide by it in our work that follows.  To avoid this, we make a definition that requires the path is differentiable, and the velocity is never zero.

\begin{definition}[Smooth Curves]\label{def:smooth curve}
 Let $\vec r(t)=(x,y,z)$ be a parametrization of a space curve $C$. We say that $\vec r$ is smooth if $\vec r$ is differentiable, and the derivative is never the zero vector. If $\vec r$ is a smooth parameterization, then we call $C$ a smooth curve. 
\end{definition}


\begin{problem}\label{prob:basic helix}
\marginpar{Watch TWO \href{http://www.youtube.com/watch?v=m25oxYTfXfU&list=PL30EE81142B1ED1F0&index=4&feature=plpp_video}{YouTube Videos} (Arc-Length Parameter and Unit Tangent Vector).  }%
\marginpar{
\thomasee{See 13.3: 1-10 for more practice.}
\stewarts{See 13.3:1-6 (finding length) and 13.3:17-20 (finding a unit tangent) for more practice. For 17-20, just find the unit tangent vector (for now)}
}%
 Consider the helical space curve $C$ with parameterization $\vec r(t)=(\cos t, \sin t, t)$. 
\begin{enumerate}
	\item Is C a smooth curve?  
	\item Find the length of this space curve for $t\in[0,2\pi]$ using the formula in \ref{arc length2}. Compute any integrals. 
	\item Now find the length of the space curve from $t=0$ to time $t=t$. 
	\item Give a vector tangent to the curve at $t=2\pi$.
	\item Now give a vector of length 1 that is tangent to the curve at $t=2\pi$. 
\end{enumerate}
\end{problem}

In the previous exercise, you developed two big ideas.  You showed how to obtain a unit tangent vector to a curve. You also developed a formula for the length of a curve from time $t=0$ to any time $t=t$.  This gives us a function $s(t)$ that tells us how far we have traveled after $t$ seconds. We can now predict distance traveled from time. Predicting the future is powerful. Before moving on, let's examine the derivative of $s(t)$, because it's a quantity we already know.

\begin{review*}
 Compute $\ds \int_0^t 3x^2 dx$ and $\ds \int_0^t 3p^2 dp$ and $\ds \int_0^t 3\tau^2 d\tau$. 
 Does it matter what you call the variable inside the integral? 
 Then compute $\ds\frac{d}{dt} \int_0^t 3\tau^2 d\tau$. See 
\footnote{
The first integral is $x^3|_0^t = t^3$. The other two are the same. You can change the variable inside the integral whenever you want.  For this reason, some people call it a dummy variable. 
The last part is $\ds\frac{d}{dt} \int_0^t 3\tau^2 d\tau = \frac{d}{dt} t^3 = 3t^2$, 
we just replaced $\tau$ with $t$ in $3\tau^2$.}
for an answer. 
\end{review*}


\begin{problem}\label{fundamental theorem of calculus as it applies to arc length parameter}
\marginpar{You can remember $\ds\frac{ds}{dt} = \left|\frac {d\vec r}{dt} \right|$ as follows. We use the differential $ds$ to represents a change in distance, and $dt$ represents a change in time. So the speed of an object is the change in distance $ds$ over the change in time $dt$. }%
 Let $\vec r(t)=(x,y,z)$ be a parametrization of a smooth space curve. Let $\ds s(t)=\int_0^t \left|\frac {d\vec r}{d\tau} \right|\ d\tau$.  
\begin{enumerate}
	\item Explain why $\ds\frac{ds}{dt} = \left|\frac {d\vec r}{dt} \right|$, the speed. 
	\item Now explain why $s(t)$ is an increasing function.
\end{enumerate}

 [Hint: Look up the fundamental theorem of calculus. To answer why is $s$ increasing, what does ``smooth'' mean?] 
\end{problem}

We'll call $\ds s(t)=\int_0^t \left|\frac {d\vec r}{d\tau}\right|\ d\tau$ the arc length parameter.  It tells us how far we've have traveled after $t$ seconds. We can now predict distance traveled from time elapsed. Because $s(t)$ is an increasing function, we can also invert this process and give time elapsed from distance traveled. This means we could compute derivatives with respect to $s$ instead of $t$.  
When we take a derivative with respect to $s$, we ask how much a curve changes if we increase length by 1 unit, instead of increasing time by 1 unit.  We'll write
$$\ds\frac{d\vec r}{ds} =\ds\frac{d\vec r/dt}{ds/dt} = \frac{d\vec r/dt}{|d\vec r/dt|} = \frac{\vec v}{|\vec{v}|}.$$

\hrule

%\newpage
%\large \textbf{In-Class/After-Class:}
%\normalsize

\begin{problem}
\marginpar{
\thomasee{See 13.3: 11-14 for more practice.}
\stewarts{\textcolor{red}{TBD?}}
}%
Consider again the helical space curve $\vec r(t)=(\cos t, \sin t, t)$.  We've shown that $s(t) = t\sqrt{2}$ in \ref{prob:basic helix}. 
\begin{enumerate}
	\item Solve for $t$ in terms of $s$ (so find the inverse of $s(t)$). 
	\item Use your answer to determine how much time has elapsed if you've travelled 4 units of distance.
	\item Compute $D\vec r(t)$ and $Dt(s)$.  You should have a 2 by 1 matrix, and a 1 by 1 matrix. 
	\item Use the chain rule to compute the derivative of $\vec r(t(s))$.
	\item Compute the length $\left|\ds\frac{d\vec r}{ds}\right|$.
\end{enumerate}
\end{problem}

The previous exercise motivates the following definition.

\begin{definition}[Unit Tangent Vector]\label{def:unit tangent vector}
 Let $\vec r(t)$ be a parametrization of a smooth space curve. We define the unit tangent vector $\vec T(t)$ to be the derivative of $\vec r$ with respect to arc length, which means
$$\vec T = \ds\frac{d\vec r}{ds}=\ds\frac{d\vec r/dt}{ds/dt} = \frac{d\vec r/dt}{|d \vec r/dt|} = \frac{\vec v}{|\vec v|}.$$
This is exactly the same as a unit vector in the same direction as the velocity.
\end{definition}

As we progress through this unit, one of our key goals is to learn new notation.  We've got position $\vec r$,  velocity $\vec v$, speed $v$ or $ds/dt$, acceleration $\vec a$, the unit tangent vector $\vec T$, and the derivative of position with respect to arc length $d\vec r/ds$.  The last two are the exact same since $\vec T = d\vec r/ds$. Did you also notice that $ds/dt$ and $v$ are both the speed?  We'll need to start realizing that the same quantity can be developed in many ways. 

\begin{problem}
\marginpar{
\thomasee{See 13.3: 1-10 for more practice.}
\stewarts{\textcolor{red}{TBD?}}
}%
 Suppose an object moves along the space curve given by  $\vec r(t)=(a\cos t,a\sin t,b t)$. 
\begin{enumerate}
 \item Find the object's velocity and speed. What is $ds/dt$?
 \item Compute $\frac{d\vec r}{ds}$, the derivative of $\vec r$ with respect to arc length. Leave your answer in terms of $t$.[Hint: Divide the top and bottom by $dt$ and then compute $d\vec r/dt$ and $ds/dt$.]  
 \item State the unit tangent vector $\vec T(t)$.
\end{enumerate}
\end{problem}

As we progress in this chapter, we'll be computing more derivatives with respect to $s$, instead of $t$. Did you notice in the previous exercise that to compute a derivative with respect to $s$, you just compute the regular derivative with respect to $t$, and then divide by the speed. Please do the following review exercise to make sure you've got down what $\frac{d}{ds}$ means. 

\begin{review*}
 Suppose $\vec r(t)=(3\cos t,3\sin t,4t)$.  Compute $v$, $ds/dt$, $d\vec r/ds$, $\vec T$, and $d\vec T/ds$. See 
\footnote{
We have $\vec v = \dfrac{d\vec r}{dt} = (-3\sin t, 3\cos t, 4)$. 
The speed is $\dfrac{ds}{dt}=|\vec v| = \sqrt{9\sin^2t+9\cos^2t+16}=5$.  
We then compute $\dfrac{d\vec r}{ds}=\dfrac{d\vec r/dt}{ds/dt} = \dfrac{1}{5}(-3\sin t, 3\cos t, 4)$, 
which equals $\vec T$.
We finally compute
$$
\dfrac{d\vec T}{ds}=\dfrac{d\vec T/dt}{ds/dt}=\left(\dfrac{1}{ds/dt}\right)\dfrac{d\vec T}{dt} = \left(\dfrac{1}{5}\right)\dfrac{1}{5}(-3\cos t, -3\sin t, 0) =  \dfrac{1}{25}(-3\cos t, -3\sin t, 0).  
$$
} for an answer.
\end{review*}

\instructor{Old day split occurred here}
%\uday
%\normalsize

\section{Some Examples and The TNB Frame}
In this section you will...
\begin{itemize}
\item develop the notion of `curvature' and the associated scalar/vector values. 
\end{itemize}


Imagine that you are at a park with some kids (a nephew, a daughter, etc.).  The park has a merry-go-round where the kids can sit down, hold on, and then spin in circles. As the spinning speed increases, they'll feel a greater  force trying to throw them  outwards. To stay on the merry-go-round, they have to counteract this outward acceleration with their own inward acceleration. For the next few examples, let's examine the connections between $\vec v$, $\vec a$, $\vec T$, $d\vec T/dt$, $d\vec T/ds$, and $|d\vec T/ds|$ in the context of spinning around on a merry-go-round. These few examples are enough to develop just about the entire chapter. 

\begin{problem}
 Sammy sits on a merry go round. He sits 3 feet from the center of the merry ground, and lets his big sister spin him around. We can parameterize Sammy's path with the vector equation $\vec r(t) = (3\cos t, 3\sin t)$, where $t$ is in seconds.  The point $(0,3)$ corresponds to the time $t=\pi/2$.
\begin{enumerate}
	\item Draw the curve for $0\leq t\leq 2\pi$. 
	\item Compute $\vec v$ and $\vec a$. 
	\item At $(0,3)$ (so $t=\pi/2$), draw these two vectors. Are these two vectors orthogonal? 
	\item Compute $\vec T$ and $\dfrac{d\vec T}{dt}$. 
	\item At $(0,3)$, draw these two vectors. Are these two vectors orthogonal? 
	\item Compute $\dfrac{d\vec T}{ds}$ and $\left|\dfrac{d\vec T}{ds}\right|$. 
	\item How is the length of $\dfrac{d\vec T}{ds}$ related to the circle?  
\end{enumerate}
\end{problem}

%\begin{problem}
 %Sammy's older sister grabs another friend to help push the merry-go-round. The spinning speed doubles. If we replace $t$ with $2t$ in our parameterization, we get the exact same path, but traverse it twice as fast.  Sammy's path is now parametrized by $\vec r(t) = (3\cos 2t, 3\sin 2t)$.
%\begin{enumerate}
 %\item Draw the curve. How long does it take to go around once?  Note that $t=\pi/4$ now corresponds to $(0,3)$. 
 %\item Compute $\vec v$, $\vec a$, $\vec T$, $\dfrac{d\vec T}{dt}$, $\dfrac{d\vec T}{ds}$, and $\left|\dfrac{d\vec T}{ds}\right|$. At $(0,3)$, add the vectors to your picture.
 %\item Compute $\dfrac{d\vec T}{ds}$ and $\left|\dfrac{d\vec T}{ds}\right|$. How is the length of $\dfrac{d\vec T}{ds}$ related to the circle?  
%\end{enumerate}
%\end{problem}
%
%Look back at the previous two problems.  We computed $\vec v$, $\vec a$, $\vec T$, $\dfrac{d\vec T}{dt}$, $\dfrac{d\vec T}{ds}$, and $\left|\dfrac{d\vec T}{ds}\right|$ in each. Which of these changed when we changed the speed?  Which stayed the same? The next problem has you develop this. 

\begin{problem}
We started with the parameterization $\vec r(t) = (3\cos t, 3\sin t)$, now we'll replace $t$ with $5t$ and $\omega t$ to understand which values can be used as descriptors of the curve vs. the motion. 
\begin{enumerate}
	\item Replace $t$ with $5t$ and compute $\vec v$, $\vec a$, $\vec T$, $\dfrac{d\vec T}{dt}$, $\dfrac{d\vec T}{ds}$, and $\left|\dfrac{d\vec T}{ds}\right|$. 
	\item If we still want to evaluate at the time corresponding to the point $(0,3)$, what is $t=$?. [Hint: It should be smaller than $\frac{\pi}{2}$
	\item Replace $t$ with $\omega t$ and find $\vec v$, $\vec a$, $\vec T$, $\dfrac{d\vec T}{dt}$, $\dfrac{d\vec T}{ds}$, and $\left|\dfrac{d\vec T}{ds}\right|$. 
	\item Explain why we should evaluate at $t=\frac{\pi}{2\omega}$ for the point $(0,3)$.
	\item Evaluate each at $t=\frac{\pi}{2\omega}$, the time corresponding to point $(0,3)$.
	\item At $(0,3)$, which of these 6 quantities remain the same?
% \item In part 2, show that $|\vec a| = |\vec v|^2\left|\dfrac{d\vec T}{ds}\right|.$ So tripling the speed causes the acceleration to increase by a factor of \rule{1in}{.5pt}.
\end{enumerate}
\end{problem}

\marginpar{Gauss developed and expanded many of the ideas in this section while on a map making mission for a king.  He wanted to create extremely high quality 2D images of the 3D world we are in.}
\note{I will be seeking for a good reference to put here for further study.}

We started with motion on a circle with a relatively simple speed.  We then multiplied the speed by 5, and then chose a variable speed.  You should have noticed that at the same point on the curve, regardless of speed, the quantities $\vec T$, $\dfrac{d\vec T}{ds}$, and $\left|\dfrac{d\vec T}{ds}\right|$ remained the same. Will this pattern continue if we were to change from constant speeds to variable speeds?  What if we left the path of a circle, and moved to any smooth curve? How far can we take this pattern?  Is it true always?  If so, then the quantities $\vec T$, $\dfrac{d\vec T}{ds}$, and $\left|\dfrac{d\vec T}{ds}\right|$ are pretty important quantities that describe the curve we're on. 

Because $\dfrac{d\vec T}{ds}$ and $\left|\dfrac{d\vec T}{ds}\right|$ show up a lot, let's give them a definition.
\begin{definition}[Curvature $\kappa$ and The Curvature Vector $\vec \kappa$]\label{def:curvature}
 Let $\vec r(t)$ be a smooth curve (so that $\vec v$ is never zero).
\begin{itemize}
 \item The vector $\vec \kappa = \dfrac{d\vec T}{ds}$ we'll call the curvature vector. It measures how quickly the unit tangent vectors changes as we increase in length (not time).
 \item The number $\kappa = \left|\dfrac{d\vec T}{ds}\right|$ we'll call the curvature.
\end{itemize}
\end{definition}


Let's apply this new definition to a circle of any radius.  We'll quickly see that the curvature and radius of a circle are related.
The curvature vector $d\vec T/ds$ tells us how quickly $\vec T$ changes as we increase in length, which is a measure of how sharply we turn a corner.  If our curve is a circle of radius $3$, then the curvature is $1/3$.  \textbf{A larger circle should result in smaller curvature.  }

\begin{problem}
 Consider the curve $\vec r(t)=(a\cos t, a\sin t)$.
 \begin{enumerate}
  \item Draw the curve, and state the radius $\rho$ of the best approximating circle.
  \item Find the curvature vector $\vec \kappa$ by performing a computation.
  \item What relationship exists between $\rho$ and $\kappa$?  If the radius $\rho$ were to increase, what would happen to $\kappa$?
 \end{enumerate}
\end{problem}

For any curve, we could approximate how rapidly the curve turns at a point by drawing a circle that best approximates the curve (kind of like a Taylor polynomial, only now we'll use a circle.) We want the circle to meet the curve $\vec r$ tangentially, and we want the curvature of the circle to match the curvature of the curve. Since the curvature of the circle must match the curvature of the curve, we know they are inverse related.  This gives the following definition.

\begin{definition}[Radius of Curvature]\marginpar{Watch a \href{http://www.youtube.com/watch?v=cHez5K1EWPs&list=PL30EE81142B1ED1F0&index=8&feature=plpp_video}{YouTube Video}.  }%
When the curvature $\kappa$ of a smooth curve is nonzero, we'll define the radius of curvature, written $\rho$, to be the reciprocal $\rho = \dfrac{1}{\kappa}$. The curvature and radius of curvature are inversely related. 
\end{definition}



Now let's look at what happens if we change the speed in a nonlinear way.  For simplicity, let's replace $t$ with $t^2$.  You could pick any other change. If you aren`t confident with your product rule, then please do the first review exercise. If you need to remember how to show vectors are orthogonal, do the second.

\begin{review*}
 Suppose $f(x) = \sin(x^2)$.  Compute $f'(x)$ and $f''(x)$.  See 
\footnote{
We have $f'(x) = 2x\cos(x^2)$ and $f''(x) = -4x^2\sin(x^2)+2\cos(x^2)$.
}.
\end{review*}
\begin{review*}
 Show that $(-2,3,1)$ and $(4,2,2)$ are orthogonal. See 
\footnote{The dot product of these two vectors is 
$(-2,3,1)\cdot(4,2,2) = -8+6+2=0$. Because the dot product is zero, the vectors are orthogonal.
}.
\end{review*}

%\newpage

%\Large In-Class/After-Class:
%\normalsize

\begin{problem}
 The big kids pushing Sammy on the merry-go-round decide to constantly increase the spinning rate.  To parametrize Sammy's path, let's replace $t$ with $t^2$ in the original parametrization to obtain $\vec r(t) = (3\cos t^2, 3\sin t^2)$. Fill in the missing values below, and then answer the questions that follow.
$$\begin{array}{|c|c|}\hline
 \vec v & ( -6 t \sin \left(t^2\right),6 t \cos \left(t^2\right) ) \\\hline
 ds/dt & 6 t \\\hline
 \vec a & %(-6 \sin \left(t^2\right)-12 t^2 \cos \left(t^2\right),6 \cos \left(t^2\right)-12 t^2 \sin \left(t^2\right))
   \\\hline
 \vec T & \left(-\sin \left(t^2\right),\cos \left(t^2\right)\right) 
\\\hline
 d\vec T/dt & \left(-2 t \cos \left(t^2\right),-2 t \sin \left(t^2\right)\right)\\\hline
 \vec \kappa = d\vec T/ds & %\left\{-\frac{1}{3} \cos \left(t^2\right),-\frac{1}{3} \sin \left(t^2\right)\right\} 
\\\hline
 \kappa =|d\vec T/ds| & \frac{1}{3} \\\hline
\end{array}$$
\begin{enumerate}
 \item Show that $\vec v$ and $\vec a$ are not orthogonal, but that $\vec T$ and $d\vec T/dt$ are. 
\end{enumerate}
Because $\vec r (t)$ includes periodic trig functinos at $(0,3)$ we could have $t^2 = \frac{\pi}{2}$, $t^2=2\pi+frac{\pi}{2}$ or $t^2=2\pi k + \frac{\pi}{2}$ for any integer $k$.
\begin{enumerate}[resume]
	\item Show that $\vec T$ is the same at $(0,3)$, regardless of what time we pass through $(0,3)$. 
	\item Show the same is true for $d\vec T/ds$.
	\item When the speed is $\frac{ds}{dt}=6$ (so $t=1$), how long are $d\vec T/dt$ and $d\vec T/ds$? 
	\begin{enumerate}
		\item When the speed is $\frac{ds}{dt}=12$ (so $t=2$), how long are $d\vec T/dt$ and $d\vec T/ds$? 
		\item When the speed is $\frac{ds}{dt}=30$ (so $t=5$), how long are $d\vec T/dt$ and $d\vec T/ds$? 
		\item If the speed were some random number $v$, how long are $d\vec T/dt$ and $d\vec T/ds$? 
	\end{enumerate}
\end{enumerate}

\end{problem}

\begin{observation}\label{curvature observations}
 The previous exercise showed the following patterns.
\begin{itemize}
 \item The vectors $\vec v$ and $\vec a$ are not always orthogonal.
 \item The vectors $\vec T$ and $\dfrac{d\vec T}{dt}$ are always orthogonal.
 \item The vectors $\vec T$ and $\dfrac{d\vec T}{ds}$ are independent of speed. They depend only on the shape of the curve, not the speed at which you traverse the curve.
 \item If we multiply $\dfrac{d\vec T}{ds}$ by $\dfrac{ds}{dt}$, we'll get $\dfrac{d\vec T}{dt}$. The two vectors point in the same direction, which from here on out we'll call $\vec N$, the direction normal to the motion.
\end{itemize}
\end{observation}

Let's see if this pattern continues when we swap to a different curve. 
Rather than try to connect the curve to some physical real world example, let's look at a curve we are familiar with, a parabola. If the pattern holds there as well, then we may have found a key pattern that works with all curves.  Then we can take our new knowledge and apply it to ANYTHING (like space flight, roller coasters, missiles, and anything that moves).

If you want to perform the computations below by hand, you must master the product, quotient, and chain rule, as well as working with rational exponents in algebra. Feel free to try the computations by hand (if you can get them, then well done). You may also simply use Sage to do the computations below.



\begin{problem}
\marginpar{Use the \href{\sageurlforcurvature}{Sage link}.}
 Consider the curve $\vec r(t) = (t, t^2)$. The computations get intense, so let's use a computer algebra system, such as Sage (\href{\sageurlforcurvature}{follow this link}) to help us. If we don't use a computer, we could spend hours on this exercise. The computer will do all the computations, and give graphs, in seconds.
\begin{enumerate}
 \item Use this \href{\sageurlforcurvature}{Sage link} to compute $\vec v$, $\vec a$, $\dfrac{ds}{dt}$, $\vec T$, $\dfrac{d\vec T}{dt}$, $\dfrac{d\vec T}{ds}$, $\left|\dfrac{d\vec T}{ds}\right|$, and the unit vector $\vec N$ that is orthogonal to $\vec T$. Evaluate each at $t=1$ and $t=\sqrt{3}/2$ (use ``sqrt(3)/2''). Record your answers in the provided table.
 \item Let's now double the speed at which we traverse along the curve. Replace $t$ with $2t$, and then repeat the exercise above. However, instead of putting in $t=1$ and $t=\sqrt{3}/2$, we now need to use $t=1/2$ and $t=?$. Use Sage to record your answers in the provided table.
 \item After completing the table, does Observation \ref{curvature observations} still hold?
 \item How are $\vec T$ and $\vec N$ related?  Conjecture a pattern.
\vspace*{-0.5cm}
\begin{center}
 \begin{tabular}{|c|c|c|c|c|}\hline
  \multirow{2}{*}{Value}&\multicolumn{2}{|c|}{$\vec r(t)=(t,t^2)$}&\multicolumn{2}{|c|}{$\vec r(t)=(2t,(2t)^2)$}\\
  &\quad \quad at $t=1$ \quad \quad & \quad at $t=\sqrt3/2$\quad \quad &\quad \quad at $t=1/2$\quad \quad  & \quad at $t=?$\quad\quad\quad \quad \\\hline
  $\vec r$   & $(1,1)$ & $(3/4,3/4)$ & $(1,1)$ & $(3/4,3/4)$ \\\hline
  $\vec v$   & $(1,2)$& & & \\\hline
  $\vec a$   & $(0,2)$& & & \\\hline
  $\frac{ds}{dt} $    &$\sqrt{5}$ & & & \\\hline
  $\vec T$   & $(\frac{\sqrt{5}}{5},\frac{2\sqrt{5}}{5})$ & & & \\\hline
  $\frac{d\vec T}{dt} $    &$(-\frac{4\sqrt{5}}{25},\frac{2\sqrt{5}}{25})$ & & & \\\hline
  $\vec\kappa=\frac{d\vec T}{ds} $   & $(-\frac{4}{25},\frac{2}{25})$& & & \\\hline
  $\kappa=\left|\frac{d\vec T}{ds}\right| $   &$\frac{2\sqrt{5}}{25}$ & & & \\\hline
  $\vec N$   &$(-\frac{2\sqrt{5}}{5},\frac{1\sqrt{5}}{5})$ & & & \\\hline
 \end{tabular}
 
\end{center}
\end{enumerate}
\end{problem}

Our observation still holds.  Let's try it on one more curve in 3D by hand.
\begin{problem}
 Consider the helix $\vec r(t)=(4\cos t,4\sin t, 3t)$. \marginpar{Because the speed is constant, many of the computations are simplified. You can use the \href{\sageurlforcurvature}{Sage link} to check your work.} Compute $\vec v$, $\vec T$, $\dfrac{d\vec T}{dt}$, and $\dfrac{d\vec T}{ds}$. Then do the following:
\begin{enumerate}
 \item Show that $\vec T$ and $\dfrac{d\vec T}{dt}$ are orthogonal. (Do this for any time $t$.)
\item At $t=2\pi$, compute $ds/dt$, $\vec T$, $\dfrac{d\vec T}{dt}$, and $\dfrac{d\vec T}{ds}$.
 \item Let's slow down how quickly we travel along the curve.  Replace $t$ with $t/5$.  Then at $t=10\pi$, give $ds/dt$, $\vec T$, $\dfrac{d\vec T}{dt}$, and $\dfrac{d\vec T}{ds}$ for the new curve $\vec r_2(t) =(4\cos(\frac{t}{5}),4\sin(\frac{t}{5}), 3(\frac{t}{5}))$.
 \item Give a unit vector that points in the same direction as $\dfrac{d\vec T}{dt}$ or $\dfrac{d\vec T}{ds}$.
\end{enumerate}

\end{problem}

%\newpage

%\uday
%\normalsize
\section{More Practice}

\begin{itemize}
\item Motivate and compute the \textbf{Unit Normal} and \textbf{Binormal Vectors} 
\end{itemize}
\vskip0.2in

The observation still holds. It looks like we've found some key facts that pertain to any curve.  We only have 3 examples to justify our conclusion, but it's enough to make some definitions and then seek for a concrete proof that holds for every smooth curve.

Let's prove one of the facts above, namely that $\vec T$ and its derivative will always be orthogonal.  The key here is that $\vec T$ is a vector of constant length.
\begin{theorem}\label{vector valued functions of constant length}
 If a vector valued function $\vec r(t)$ has constant length, then the vector $\vec r$ and its derivative $\ds\frac{d\vec r}{dt}$ are always orthogonal. Vector valued functions of constant length are orthogonal to their derivative.  
\end{theorem}

\begin{problem}[Optional Proof of Theorem \ref{vector valued functions of constant length}]\marginpar{Watch a \href{http://www.youtube.com/watch?v=08Ygw_M-4yM&list=PL30EE81142B1ED1F0&index=6&feature=plpp_video}{YouTube Video} for the answer.  }%
 Prove the theorem above. Here are some hints [as an alternative to watching the YouTube video].
\begin{itemize}
 \item We know that $\vec r(t)$ has constant length, so we write $|\vec r|=c$ for some constant $c$. 
 \item We need to get from a magnitude to a dot product. Look in your text for a way to relate magnitude to the dot product. See exercise \ref{dot product facts}.
 \item After writing $|\vec r(t)|=c$ in terms of a dot product (squaring both sides may help), take the derivative of both sides. Apply the product rule to the dot product.
\end{itemize}
\end{problem}

The above fact is so crucial, that we'll repeat what it says.
\begin{quote}
If the vector $\vec v(t)$ has constant length, then the vector and its derivative $\frac{d\vec v}{dt}$ are orthogonal.
\end{quote}


\begin{problem}\label{T and N are orthogonal}\marginpar{Watch a \href{http://www.youtube.com/watch?v=aJttU3kS_p8&list=PL30EE81142B1ED1F0&index=7&feature=plpp_video}{YouTube Video}.  }%
 Let $\vec r$ be a smooth parametrization of a curve.
\begin{enumerate}
	\item How long is the {\it unit} tangent vector $\vec T(t)$? 
	\item Explain why $\vec T$ is orthogonal to $\dfrac{d\vec T}{dt}$.
	\item Then give a formula for computing a unit vector that is orthogonal to $\vec T(t)$. 
\end{enumerate}
\end{problem}


Based on our answer above, let's make the following definition of the principle unit normal vector.  The key idea is that this vector points in the direction of normal acceleration. 
\begin{definition}[Principle Unit Normal Vector]
 Suppose $\vec r(t)$ is a smooth parametrization of a space curve, where the unit tangent vector is $\vec T(t)$.  
We define the principle unit normal vector $\vec N(t)$ to be the vector
 $$\vec N(t) = \ds\frac{d\vec T/dt}{|d\vec T/dt|},$$
 provided of course that $|d\vec T/dt|\neq 0$. 
 From exercise \ref{T and N are orthogonal} we know that $\vec T$ and $\vec N$ are orthogonal.
\end{definition}

The vector $\vec N$ is a unit vector in the same direction as the curvature vector.  Since $\dfrac{d\vec T}{dt}$ and $\dfrac{d\vec T}{ds}$ point in the same direction, we could have written $\vec N = \dfrac{d\vec T/ds}{|d\vec T/ds|} = \dfrac{\vec \kappa}{\kappa}$.

The vectors $\vec T$ and $\vec N$ give us exactly the pieces needed to describe velocity and acceleration.  We'll visit this more in a bit.  When we are in 3D, these two vectors describe a plane. The binormal vector of the TNB frame is the normal vector to this plane.  

\begin{definition}[Binormal Vector]
 If $\vec r$ is a parametrization of a smooth space curve with unit tangent vector $\vec T$ and principle unit normal vector $\vec N$, then we define the binormal vector $\vec B$ to be the cross product
$$\vec B = \vec T\times \vec N.$$
 It's normal to both $\vec T$ and $\vec N$, hence called the \textit{bi}normal vector.
\end{definition}

We now have the entire $TNB$ frame.  This gives us a moving collection of unit vectors that act like an $xyz$ coordinate system.  Many of you will use this frame a ton in your dynamics course. The TNB frame shows up in physical chemistry as well. A key fact to remember is that all three vectors are unit vectors, and they are each orthogonal to the other.

\begin{review*}
 Find the area of a parallelogram with corners $(0,0,0)$, $(1,0,2)$, and $(1,1,3)$, and $(2,1,5)$.  See \footnote{
We find area of a parallelogram by the vectors that form the edges. Since one point is the origin (subtracting it from the 2nd and 3rd points won't change anything), we compute the cross product
$(1,0,2)\times 
(1,1,3) 
= (-2,-1,1) $. The magnitude of the cross product is the area $A = \sqrt{4+1+1}=\sqrt{6}$.
}.
\end{review*}


\begin{problem}
Answer the following questions (this will review your knowledge of the dot and cross products).
\begin{enumerate}
	\item What is $\vec T\cdot \vec N$? Explain. 
	\item Now explain why $\vec T\cdot \vec B=0$ and $\vec N\cdot \vec B=0$.
	\item Both $\vec T$ and $\vec N$ are unit vectors. They are also orthogonal. Why is $\vec B$ is a unit vector? [What's the connection between the cross product and area.] 
	\item We defined $\vec B=\vec T\times \vec N$. This means that $\vec N\times \vec T=-\vec B$.  Do we have $\vec T\times \vec B$ equal to $\vec N$ or $-\vec N$? Explain. [Hint: What does the right hand rule say? Compare to $\hat i\times \hat k$. ]
\end{enumerate}
\end{problem}

Let's carry through a single exercise where we compute $\vec T$, $\vec N$, and $\vec B$. 

\begin{problem} \label{helix example of T N and B}
\marginpar{
\thomasee{See 13.4: 9-16 and 13.5: 9-16 (the relevant parts) for more practice.}
\stewarts{See 13.3:47-48 for more practice, You can also go back and do 13.3:17-20, this time just do the unit normal vector}
}%
Consider the helix $\vec r(t) = (3\cos t,3\sin t, 4t)$.  Find the unit tangent vector $\vec T(t)$, principle unit normal vector $\vec N(t)$, and the binormal vector $\vec B(t)$. 
\end{problem}

\clearpage

\subsection{Concluding Thoughts on TNB}

%Once the speed is no longer constant, things get a lot messier than the previous exercise. Ask me in class to show you what happens with the computations when you consider something like $r(t)=(t,t^2,t^3)$. Things get ugly really fast. Fortunately, when you're working with a curve that lies in a plane, there are some simplifications that occur.





%----------------
%Extra Practice Stuff.....
%
%
%
%
%
%\begin{problem}
%\marginpar{
%\thomasee{See 13.4: 7-8 for more practice, and perhaps a hint.}
%}%
 %Suppose you have already computed the unit tangent vector for a curve in the plane and found at a specific time it equals $\vec T=(a,b)$.   
%\begin{enumerate}
 %\item State a nonzero vector that is orthogonal to $(a,b)$. (Guess one, and use the dot product to check. If you're struggling because of the variables, find a vector orthogonal to $(2,3)$.)
 %\item Let $\vec r(t) = (t,t^2)$. We then have $\frac{d\vec r}{dt} = (1,2t)$ and $\vec T(t) = \frac{(1,2t)}{\sqrt{1+4t^2}}$. Without computing any more derivatives, guess the principle unit normal vector $\vec N(t)$?
 %\item Draw a picture of the curve. At $t=1$ add to your picture the tangent vector $(1,2)/\sqrt{5}$ and your guessed normal vector. (If your guess was off by a sign, tell us how to modify your guess.)
 %\item Why does $\vec B=(0,0,1)$?
%\end{enumerate}
%\end{problem}
%
%\begin{observation}
%From the problem above, we learn the following fact.  If the tangent vector to a planar curve is $\vec T(t) = (a(t),b(t))$, then the principle unit normal vector is either $\vec N(t)=(-b(t),a(t))$ or $\vec N(t)=(b(t),-a(t))$.  You just reverse the components, and then negate one of them.  To determine which one to negate, draw a picture.
%\end{observation}
%
%\begin{problem}
%Consider the curve $\vec r(t)=(t^2,t)$. Compute $\vec T(t)$. Reverse the order and negate one of the component to find $\vec N(t)$. To know if you guessed the write component to negate, draw the curve and on your graph include these vectors at $t=1$. Finally, state $\vec B(t)$.
%\end{problem}
%
%\begin{problem}
%\marginpar{
%\thomasee{See 13.4: 1-4 for more practice. Use the previous exercises.}
%}%
 %Consider the curve $y=\sin x$, parametrized by $r(t)=(t,\sin t)$. We know that $\vec T(t) = \dfrac{(1,\cos t)}{\sqrt{1+\cos^2t}}$. Draw the curve from $-\pi$ to $\pi$. Then on your graph draw $\vec T$ and $\vec N$ at $t=\pi/2$, $\pi/4$, $-\pi/4$.  
%\begin{enumerate}
 %\item What is  $\vec T(t)$ at each of $t=\pi/2$, $\pi/4$, $-\pi/4$?
 %\item Show how to get $\vec N(t)$ at each of $t=\pi/2$, $\pi/4$, $-\pi/4$?
 %\item What is  $\vec B(t)$ at each of $t=\pi/2$, $\pi/4$, $-\pi/4$? \marginpar{What happens if $t=0$?}
%\end{enumerate}
%\end{problem}


\indent You've now developed the TNB frame for describing motion. Engineers will see this again when they study dynamics. Mathematicians who study differential geometry will use these ideas as well. Any time you want to analyze the forces acting on a moving object, the TNB frame may save the day. Chemists will encounter the TNB frame briefly when they study P-chem and the motion of subatomic particles.


%
%\begin{problem}
 %Consider the curve $\vec r(t)=(t,\sin 3 t)$. We can compute 
%$\vec v(t) = (1,3\cos t)$  and 
%$\vec T(t) = \frac{(1,3\cos 3t)}{\sqrt{1+9\cos^2(3t)}}$.
%Using the quotient rule, we find that 
%$$\ds\frac{dT}{dt} = \frac{\sqrt{1+9\cos^2(3t)}(?,?) - (1,3\cos3t)(?)}{1+9\cos^2(3t)}.$$
%\begin{enumerate}
 %\item Fill in the blanks in the quotient rule above. 
 %\item\marginpar{Use this \href{\sageurlforcurvature}{Sage link} to check your work, and see if your picture is correct. You'll have to type the appropriate function in, so use ``[t,sin(3*t)]'' and ``point=pi/6.''}%
 %Show at $t=\pi/6$ that $d\vec T/dt = (0,-9)$. Then state $\vec \kappa$, $\kappa$, and $\rho$ at $t=\pi/6$. 
 %\item Draw the curve $\vec r(t)$ and on your curve at $t=\pi/6$ add the circle of curvature.
 %\item Where is the center of curvature (the center of the best approximating circle) at $t=\pi/6$.
%\end{enumerate}
%
%\end{problem}
%
\begin{review*}
 If you are standing at $(2,1,-3)$ and you wish to move 6 units in the direction of the unit vector $(1/3, 2/3, -2/3)$, where are you? See \footnote{We want to start at $(2,1,-3)$ and move $6(1/3, 2/3, -2/3) = (2,4,-4)$. We just add the vectors, giving $(4,5,-7)$.} for an answer.	
\end{review*}

\begin{problem}[Extra Practice] \marginpar{Use this \href{\sageurlforcurvature}{Sage link} to check your work, and see if your picture is correct. You'll have to type the appropriate function in, so use ``[t,sin(t),cos(t)]'' and ``point=pi/2.''}%
 Consider the helix $\vec r(t)=(t,\sin t,\cos t)$. 
\begin{enumerate}
	\item Find the curvature and radius of curvature $t=\pi/2$. 
	\item Draw the curve, and draw the circle of curvature at $t=\pi/2$. 
	\item Finally, find the center of curvature at $t=\pi/2$. 
	\item Take a guess at the center of curvature at $t=\pi$? [Hint: If you're struggling with how to get from the curve to the center of curvature, please do the review exercise above.]
\end{enumerate}
\instructor{This may not have all the background/support needed for students to succeed.}
\end{problem}


\indent When a civil engineering team builds a road, they have to pay attention to the curvature of the road.  If the curvature of the road is too large, accidents will happen and the civil engineering team will be liable. How do they make sure the curvature never gets to large?  They use the circle of curvature. When they want to cause a road to turn, they'll find the center of curvature, send a surveyor out to the center, and then have the surveyor make sure that the road follows the circle of curvature for a short distance. They actually pace out the circle of curvature and then build the road along this circle for a hundred feet or so.  Then, they recompute the radius of curvature (if they need the direction to change again), and pace out another circle.  In this way, they can guarantee that the curvature never gets large. In the next section we'll see how curvature is directly related to normal acceleration (which is what causes semis to tip, and vehicles to slide off icy roads.)


%\newpage

%\uday
%\normalsize


\section{Tangential And Normal Components}
\begin{itemize}
\item Applications of multi-variable calculus (Tangential and Normal Components) 
\end{itemize}


\indent The unit tangent vector $\vec T$ provides us with a unit vector in the direction of motion. We can obtain the direction of motion from the velocity. If we stay on a straight course, then our acceleration is in the same direction as our motion, and would only cause us to speed up or slow down. We'll call this tangential acceleration.

\indent If we want to design a roller coaster, build an F15 fighter plane, send a satellite in orbit, or construct anything that doesn't move in a straight line, we need to understand how acceleration causes us to leave a straight path. We may still be speeding up or slowing down (tangential acceleration), but now we'll have a component that veers us off the straight path.  We'll call this normal acceleration, it's orthogonal to the velocity. 

\indent Back in the vector chapter, we practiced writing a force $\vec F$ as the sum of the component parallel to a displacement $\vec d$ and the component orthogonal to $\vec d$.  We could write this as $\vec F = \vec F_{|| \text{ to }\vec d} + \vec F_{\perp \text{ to }\vec d}.$ The parallel part came from a projection.  The orthogonal part came from vector subtraction.  If you've forgotten how to do this, please do this review exercise.

\begin{review*}
 Consider the force vector $\vec F = (0,-10)$, and displacement vector $\vec (2,-1)$.  Compute the projection of $\vec F$ onto $\vec d$, and then write $\vec F$ as the sum of a vector parallel to $\vec d$ and a vector orthogonal to $\vec d$. See \footnote{
The projection is $\proj_{\vec d}\vec F = \frac{10}{5}(2,-1) = (4,-2)$.  This is the parallel component $\vec F_{|| \text{ to }\vec d} =(4,-2)$.  To get the orthogonal component, we know that $\vec F = \vec F_{|| \text{ to }\vec d} + \vec F_{\perp \text{ to }\vec d}$. Vector subtraction gives
$\vec F_{\perp \text{ to }\vec d} = \vec F -\vec F_{|| \text{ to }\vec d} = (0,-10)-(4,-2) = (-4,-8)$.
We now write 
$$\vec F = (4,-2) + (-4,-8).$$
}. 
\marginpar{This is a good time to look back over the projection section from Unit 1: \ref{prob:force intro}}
\end{review*}


% \begin{problem}
%  Suppose that a projectile is fired from the ground on some distance planet, and that the only acceleration is due to gravity. We can parametrized the path of the particle with $r(t)=(40t,-5t^2+40t)$. It takes 8 seconds to hit the ground.  
% \begin{enumerate}
%  \item \marginpar{\href{http://aleph.sagemath.org/?q=7f8cd279-2ee5-41e3-9ce0-1a7415e2cf20&lang=sage}{Check your work with this Sage link.}}%
% Find $\vec v$ and $\vec a$.  We're on a distant planet because you should find the acceleration is always pointing straight down with magnitude 10. 
%  \item Find $\vec T(t)$, and then at $t=1$ please compute $\text{proj}_{\vec T(1)}\vec a(1)$. 
%  \item State $\vec N(t)$ (remember you can flip the order and change a sign), and then at $t=1$ compute $\text{proj}_{\vec N(1)}\vec a(1)$. 
%  \item If we write $\vec a = a_T\vec T +a_N\vec N$, then what are $a_T$ and $a_N$?
% \end{enumerate}
% \end{problem}

If we throw a pebble from a 64 ft tall cliff, then we could parameterize the path after $t$ seconds using $\vec r(t) = (3t,64-16t^2)$. The numbers below get rather large in a hurry, so let's use a simpler parameterization to gain understanding about the connection between $\vec v$, $\vec a$, $\vec T$, and $\vec N$.  Sometimes the key to understanding is to simplify the exercise.
\begin{problem}
Consider the parameterization $\vec r(t) = (t,9-t^2)$ (it's a simplified version of tossing a pebble off a building). Our goal, at time $t=1$, is to write $\vec a$ in the form $\vec a = a_T\vec T+a_N\vec N.$ Follow the steps below.
\begin{enumerate}
	\item Compute $\vec v$, $\vec a$, and $\vec T$ at time $t$, and then at time $t=1$. 
	\item At $t=1$, compute the projection of $\vec a$ onto $\vec T$, i.e. compute $\proj_{\vec T(1)}\vec a(1)$.
	\item State $\vec N(t)$ (remember you can flip the order and change a sign), and then at $t=1$ compute $\text{proj}_{\vec N(1)}\vec a(1)$. 
	\item If we write $\vec a = a_T\vec T +a_N\vec N$, then what are $a_T$ and $a_N$?
\end{enumerate}
  
\end{problem}



\begin{definition}[Tangential and Normal Components of Acceleration]
 Suppose that $\vec r(t)$ is a smooth parametrization of a moving object.  Let $\vec T$ be the unit tangent vector. The tangential component of acceleration and the normal component of acceleration are the scalars $a_T$ and $a_N$ that we obtain by writing the acceleration as the sum of a vector parallel to $T$ and a vector orthogonal to $\vec T$, i.e. the scalars that satisfy
$$\vec a = a_T\vec T+a_N\vec N.$$
\end{definition}

Let's return to the example of Sammy on a merry-go-round.  From this example, we'll see one of the key ideas in this section.
\begin{problem*}[Optional, we already basically did this]
 Suppose that Sammy sits $\rho$ feet away from the center of the merry-go-ground. His sister decides to spin him around at different speeds.  Let $\vec r(t) = (\rho \cos \omega t, \rho \sin \omega t)$ be a parametrization of Sammy's postion.
\begin{enumerate}
 \item Show that Sammy's speed is $|\vec v|=\rho \omega$.  
 \item Find the curvature of Sammy's path at any time $t$ (it happens to be constant - and we did this already when we computed the curvature along a circle).
 \item Find the acceleration vector, and show that $\ds |\vec a| = \kappa |\vec v|^2 = \frac{|\vec v|^2}{\rho}$.
\end{enumerate}
\end{problem*}

In the exercise above, all of the acceleration is in the normal direction. The interesting thing to note is that the normal acceleration is ALWAY $a_N =  \frac{|\vec v|^2}{\rho}=\kappa |\vec v|^2$.  That's what we'll now show.
We'll show that the acceleration of an object moving along a curve $\vec r(t)$ with velocity $\vec v(t)$ is the sum
$$\vec a(t) = a_T\vec T+a_N\vec N=\frac{d}{dt}|\vec v(t)| \vec T + \kappa |\vec v|^2 \vec N.$$
The scalars $a_T=\dfrac{d}{dt}|\vec v(t)|$ and $a_N=\kappa |\vec v|^2$ 
\marginpar{Engineers often use the equivalent formula $a_N = \frac{|\vec v|^2}{\rho}$, as $\rho$ is a physical distance that they can measure.} 
are the tangential and normal components of acceleration.  All we have to do is write the vector $\vec a(t)$ as the sum of a vector parallel to $\vec T$ and a vector orthogonal to $\vec T$. 

Before we decompose the acceleration into its tangential and normal components, let's look at two examples to see what these facts physically represent.

% \begin{review*}
%  Consider the force vector $\vec F = (0,-10)$ and displacement vector $\vec (2,-1)$.  Draw $\vec F$, $\vec d$, and the projection of $\vec F$ onto $\vec d$. See \footnote{The solution is
% \begin{tikzpicture}
%  draw[->] (0,0)->(0,-10);
%  draw[->] (0,0)->(2,-1);
% \end{tikzpicture}
% }.
% \end{review*}


% 
% \begin{problem}
% \marginparbmw{See 13.5: 17-20 for more practice.}%
%  Consider the path of an object in projectile motion that has been fired from the origin. Draw a typical path followed by a projectile.  The acceleration $\vec a(t)=(0,-g)$ acts straight down for any time $t$.  
% \begin{itemize}
%  \item Pick a point on your path before the max height occurs. At that point, draw both $\vec T$, $\vec a$, and the projection of $\vec a$ onto $\vec T$.  Is $a_T$ positive or negative? 
%  \item At the point you chose above, is the speed of the projectile increasing or decreasing as it climbs higher?
%  \item Now pick a point after the projectile passes the peak.  Repeat the last two parts at this new point.
% \end{itemize}
% \end{problem}

\begin{problem}
 Imagine that you are riding as a passenger on a road and encounter a series of switchbacks (so the road starts to zigzag up the mountain). Right before each bend in the road, you see a yellow sign that tells you a U-turn is coming up, and that you should reduce your speed from 45 mi/hr to 15 mi/hr.  Assume the largest curvature along the turn is $\kappa$. Recall that $a_N=\kappa |\vec v|^2$. The engineers of the road designed the road so that if you are moving at 15 mi/hr, then the normal acceleration will be at most $A$ units. 
\begin{enumerate}
 \item Suppose that your driver (Ben) ignores the suggestion to slow down to 15 mi/hr.  He keeps going 45 mi/hr through the turn. Had he slowed down, the max acceleration would be $A$.  You're traveling 3 times faster than suggested.  What will your maximum normal acceleration be? [Hint: It's more than $3A$.]
 \item You yell at Ben to slow down (you don't want to die). So Ben decides to only slow to 30 mi/hr. He figures this means you'll only feel twice as much acceleration as $A$.  Explain why this line of reasoning is flawed.
 \item Ben gets frustrated by the fact that he has to slow down. He complains about the engineers who designed the road, and says, ``they should have just built a larger corner so I could keep going 45.''  How much larger should the radius of the circle be so that you can travel 45 mi/hr instead of 15 mi/hr, and still feel the same acceleration $A$?
 \item Which will cause the normal acceleration to decrease more, halving your speed or halving the curvature (doubling the radius)?
\end{enumerate}
\end{problem}

% \begin{problem}
%  We defined the principle unit normal vector as $\vec N = \dfrac{d\vec T/dt}{|d\vec T/dt|}$.  Explain why we can write $\vec N = \dfrac{d\vec T/ds}{|d\vec T/ds|}$ as well. Then use this fact to explain why , which means we can write $\ds\kappa|\vec v|\vec N=\frac{d\vec T}{dt}$.  
% \end{problem}


\begin{challenge}\marginpar{Watch a \href{http://www.youtube.com/watch?v=cSh2Bdd-yTg&feature=bf_next&list=PL30EE81142B1ED1F0&lf=plpp_video}{YouTube Video}.  }%
 Prove that $\ds \vec a(t) = a_T\vec T+a_N\vec N=\frac{d}{dt}|\vec v| \vec T + \kappa |\vec v|^2 \vec N.$ Here's some hints.
\begin{itemize}
 \item Rewrite the velocity $\vec v$ as a magnitude $|\vec v|$ times a direction $\vec T$, so $\vec v = |v|\vec T$.  
 \item We know that $\vec a(t) = \frac{d}{dt}\vec v(t)$. Take the derivative of $\vec v = |\vec v|\vec T$ by using the product rule (on the scalar product $|\vec v|\vec T$).
 \item You should encounter the quantity $d\vec T/dt$. We know that $\frac{d\vec T}{dt} = |\vec v|\frac{d\vec T}{ds}$. Why does $\ds d\vec T/dt=|\vec v|\kappa\vec N$?
 \item Conclude to explain why $a_N =\kappa |\vec v|^2$.
\end{itemize}
\end{challenge}

Let's now use the fact above to get an extremely useful formula for the curvature. 

\begin{challenge}
Show that $$\kappa = \frac{|\vec v\times \vec a|}{|\vec v|^3} = \frac{|\vec r'\times \vec r''|}{|\vec r'|^3}.$$

[Hint: We know that $\ds \vec a = a_T\vec T+a_N\vec N=\frac{d}{dt}|\vec v| \vec T + \kappa |\vec v|^2 \vec N.$  
Cross both sides with $\vec v$. You should be able to cancel $\vec v\times \vec v$ (why). Then take the magnitude of each side and solve for $\kappa$. You'll have to explain why $|\vec v\times \vec N| = |\vec v|$.]
\end{challenge}

We can use the above formula for curvature to get a quick way to compute the curvature of a function $y=f(x)$. If you use the previous exercise, this formula falls out almost instantly. You'd see this formula in dynamics\bmw{, and it shows up on the Fundamentals of Engineering exam (where you just have to use the formula, not prove where it comes from)}. This is the culminating idea from this chapter that you'll use again and again in engineering courses.

\begin{challenge}\label{formula for curvature}
\marginparbmw{See 13.4: 5.}%
 The function $y=f(x)$ can be given the parametrization  $\vec r(x) = (x,f(x))$.  Use this parametrization (and the previous exercise) to show that the curvature is 
$$\kappa(x) = \frac{|f''(x)|}{(1+(f')^2)^{3/2}},$$ 
and that the radius of curvature is
$$\rho(x) = \frac{(1+(f')^2)^{3/2}}{|f''(x)|}.$$ 
\end{challenge}





\subsection*{Optional: Torsion}
\valpo{We will be skipping this section, however, if you want to see more tie-in to physics and engineering, this skim this section.}


\begin{definition}[Torsion]
 Let $\vec r(t)$ be a parametrization of a smooth curve $C$ with unit tangent vector $\vec T(t)$.  
 The derivative of $\vec B$ with respect to $s$ tells us how rapidly the plane containing $\vec T$ and $\vec N$ rotates. We'll define the torsion vector to be 
\marginpar{Watch a \href{http://www.youtube.com/watch?v=MVtUc2peJn0&feature=bf_next&list=PL30EE81142B1ED1F0&lf=plpp_video}{YouTube Video}.  }%
 $$\vec \tau = \dfrac{d\vec B}{ds} = \dfrac{d\vec B/dt}{ds/dt}=\dfrac{d\vec B/dt}{|d\vec r/dt|}.$$ 
 The torsion $\tau$, up to a sign, is the length of this vector. We say there is positive torsion if $\vec \tau$ causes a counterclockwise rotation about $\vec T$ (as you look down $\vec T$), which occurs precisely when $\vec tau$ and $\vec N$ point in opposite directions. We can summarize this is $$\tau=\left|\dfrac{d\vec B}{ds}\right|\quad \text{or}\quad \tau=-\left|\dfrac{d\vec B}{ds}\right|,$$ where you choose ``$+$'' if $\vec N$ and $\vec \tau$ point in opposite directions. 
\end{definition}


The computations involved in getting $\tau$ require a lot of work. Let's use the computer to help us. You can do all of this with the aid of Sage.  I'll let you decide from the code what $\tau$ is.  That will be your decision to make. 

\begin{problem*}[Optional]
\marginpar{
\thomasee{See 13.4: 9-16 and 13.5: 9-16 (the relevant parts) for more practice.}
\stewarts{\textcolor{red}{TBD?}}
}%
Consider the helix $r(t)=(3\cos t, 3\sin t, 4t)$. In exercise \ref{helix example of T N and B} we found 
\begin{align*}
 \vec T &= (-\frac{3}{5}\sin t,\frac{3}{5}\cos t,\frac{4}{5})\\
 \vec N &= (-\cos t,-\sin t,0)\\
 \vec B &= (\frac{4}{5}\sin t,-\frac{4}{5}\cos t,\frac{3}{5})
\end{align*}
Compute the torsion vector $\vec \tau=\dfrac{d\vec B}{ds}$, and then give the torsion $\tau$ (you'll need to determine the speed).  Is the torsion positive or negative.  Ask me in class to show you how you would be able to determine this physically (without any computations).
\end{problem*}

\begin{problem*}[Optional]
 Consider the helix $r(t)=(4\sin t, 4\cos t, 3t)$. Use a computer to find $\vec T$, $\vec N$, $\vec B$, $\vec \kappa$, and $\vec \tau$. State your answers. \href{\sageurlforcurvature}{(This sage link will help.)} Use your answers to then give $\kappa$ and $\tau$. (When you present on the board, just write down the 5 vectors, and then explain how you obtained $\kappa$ and $\tau$ from these vectors. If you follow the link, this is mostly already done for you. )
\end{problem*}

In the examples above, you should have noticed that $\vec \tau$ was either parallel to $\vec N$ or anti-parallel to $\vec N$.  Let's now show this is always the case. The key is to use the product rule on the cross product, together with some key fact about the cross product.
\begin{review*}
 What is the cross product of $(1,2,3)$ and $(2,4,6)$?  If two vectors are parallel, then what is their cross product?  In particular, what is the cross product of $\vec N$ and $\vec \kappa$? See \footnote{The cross product of parallel vectors is always the zero vector $(0,0,0)$. This is because the area of the parallelogram formed using the parallel vectors is always zero. So all three answers are $(0,0,0)$.} for an answer.
\end{review*}


\begin{problem*}[Optional] \marginpar{Watch a \href{http://www.youtube.com/watch?v=MVtUc2peJn0&feature=bf_next&list=PL30EE81142B1ED1F0&lf=plpp_video}{YouTube Video}.  }%
 Suppose a curve $\vec r(t)$ has the frame $\vec T(t)$, $\vec N(t)$, and $\vec B(t)$. Prove that $\dfrac{d\vec B}{ds}$ is either parallel to $\vec N$, or points opposite $\vec N$. Here are some steps.
 \begin{itemize}
  \item Why is $\dfrac{d\vec B}{ds}$ orthogonal to $\vec B$? [Hint: How long is $\vec B$? See Theorem \ref{vector valued functions of constant length}.]
  \item We know $\vec B=\vec T\times \vec N$. Compute the derivative of both sides using the product rule and explain why $\frac{d\vec T}{ds}\times \vec N$ cancels out. Then explain why $\dfrac{d\vec B}{ds}$ is orthogonal to $\vec T$.
  \item If $\dfrac{d\vec B}{ds}$ is orthogonal to both $\vec B$ and $\vec T$ why must it be either parallel or anti-parallel to $\vec N$?
 \end{itemize}
\end{problem*}





