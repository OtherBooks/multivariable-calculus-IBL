\note{After 3 semesters of trying this, I need to rewrite the section again.  I put lots of notes in my notebook for this semester. I'll rewrite the section soon. Some problems need to be combined.  Some need to be dropped.  Other sections need more simpler problems (work/flux).  Get rid of many of the ``use a computer to integrate'' problems.  Just have them focus on what I'll ask them to do anyway. }

\noindent 
This unit covers the following ideas. In preparation for the quiz and exam, make sure you have a lesson plan containing examples that explain and illustrate the following concepts.  
\begin{enumerate}
\item Describe how to integrate a function along a curve. Use line integrals to find the area of a sheet of metal with height $z=f(x,y)$ above a curve $\vec r(t)=\left(x,y\right)$ and the average value of a function along a curve.
\item Find the following geometric properties of a curve: centroid, mass, center of mass, inertia, and radii of gyration.
\item Compute the work (flow, circulation) and flux of a vector field along and across piecewise smooth curves.
\item Determine if a field is a gradient field (hence conservative), and use the fundamental theorem of line integrals to simplify work calculations.
\end{enumerate}
You'll have a chance to teach your examples to your peers prior to the exam.

In this chapter, we generalize integrals along the $x$-axis from previous semesters in calculus to integrals along any curve.

\section{Surface Area}



In this section, we'll first generalize the concept of the integral.  We'll approach everything from the point of view of area, though the applications are much more extensive. The first problem is a review problem from first-semester calculus.  The second problem generalizes the idea to integrals along a curve (which we call a line integral). The third problem has you generalize your results. Let's start with a quick review of sigma notation.

\begin{example}
Recall that we write $\ds \sum_{i=1}^{20} i^3$ as short hand for $$\ds \sum_{i=1}^{20} i^3=1^3+2^3+3^3+4^3+\cdots + 20^3.$$ This notation is called sigma notation.  It allows us to express really long sums in a very short space.  We could also write 
$$\sum_{i=30}^{4000}x_i = x_{30}+x_{31}+x_{32}+\cdots +x_{4000}$$ if we needed to add up the numbers starting at $x_{30}$ and ending at $x_{4000}$. If we wanted to add the integers starting at 1 and ending at $n$, then we would write 
$$1+2+3+\cdots+n=\sum_{i=1}^n.$$  
\end{example}


This first problem is a step-by-step review of how we did integrals in first-semester calculus. It also asks you to write sums in sigma notation.

\begin{problem}
 Consider the region in the $xy$ plane that is below the function $f(x)=x^2+1$ and above the $x$-axis where $x\in[-1,2]$.  Think of this region as a metal plate. We will find its surface area.
\begin{enumerate}
 \item Draw the curve over the given bounds, and shade the region. 
 \item  \marginpar{We could have used the left endpoint point, or the midpoint, or any other point in each partition. I chose the right endpoint to make sure we all had the same answers.}%
 Now partition the interval $[-1,2]$ into 6 equally-spaced parts. On your graph, draw 6 rectangles  to approximate the area under $f$.  Use the right endpoint of each interval to determine the height of each rectangle.
 %\item 
 The width of each rectangle we call $\Delta x$. What is $\Delta x$ in this example?
 \item Recall from first semester calculus that we typically name the $x$-coordinates of the ends of our rectangles using the notation $x_0$, $x_1$, $x_2$, $\ldots$. In this example, we have 
$$
x_0=-1,\, 
x_1=-\frac{1}{2}, \, 
x_2=0, \, 
x_3=\frac{1}{2}, \, 
x_4=1, \, 
x_5=\frac{3}{2}, \, 
x_6=2.
$$
The area of the first rectangle is $\Delta A_1=f(x_1)\Delta x$. The area of the second rectangle is $\Delta A_2= f(x_2)\Delta x$. 
The total combined area of the 6 rectangles you drew above is the sum
\begin{equation*}
\begin{split}
&\Delta A_1+\Delta A_2+\Delta A_3+\Delta A_4+\Delta A_5+\Delta A_6 \\
&\quad =f(x_1)\Delta x+
f(x_2)\Delta x+
f(x_3)\Delta x+
f(x_4)\Delta x+
f(x_5)\Delta x+
f(x_6)\Delta x.
\end{split}
\end{equation*}
Write this sum using sigma notation (i.e., using a $\sum$). 

\item Instead of using 6 rectangles, let's now use $n$ equally wide rectangles. Let $\Delta x=dx$ be the width of each rectangle, and the right endpoint of each segment we'll call $x_1,x_2,\ldots,x_n$. What is the area $\Delta A_i$ of the $i$th rectangle? The sum of these $n$ little areas is approximately the total area under the curve, i.e. $$A\approx \Delta A_1+\Delta A_2+\cdots +\Delta A_n.$$ Write this sum using sigma notation.

\item Explain why the area under $f$ over the interval $[-1,2]$ is $A=\int_{-1}^2 (x^2+1) dx$. 
%\item Find the area (surface area) of the metal sheet.
\end{enumerate}

\end{problem}

The next problem should mimic the steps in the previous problem.  The only difference is that you are now integrating over a curve, not over in interval---this is the generalization from first-semester calculus that we are looking at now.

\begin{problem}%
\marginpar{\href{http://www.youtube.com/watch?v=sYsMcqtXBrc&list=PL04DF68E73B7ECD54&index=1&feature=plpp_video}{Watch a YouTube video.}}%
\marginpar{See \href{http://aleph.sagemath.org/?z=eJx1j8EOgyAQRO98hTeBroniqQe-pFFjECupFQJro39ftPXQxN5mknk7sz1dYGXymi21yNZaEE-RSSq4siEqSEoezBQVI-jb6a4lRchBcGcY8UU0c4xv0C2vINlFUcHMe8o3Ezk1-5durENjpyA7o5AqO1ovU6-7FHAw6jHpEGT5zQbpWt8-NXqjGjdapHtRHtd8NgDnP0fZ5RQoGPzJkzBojSc1B0Dn-GRxjA-XPf8GnQ5jSA}{Sage} for a picture of this sheet.}%
 Consider the surface in space that is below the function 
$f(x,y)=9-x^2-y^2$ and above the curve $C$ parametrized by 
$\vec r(t)=(2\cos t, 3\sin t)$ for $t\in[0,2\pi]$.  Think of this region as a metal plate that has been stood up with its base on $C$ where the height above each spot is given by $z=f(x,y)$.
\begin{enumerate}
 \item \marginpar{See Problem \ref{horse track chain rule introduction}}%
  Draw the curve $C$ in the $xy$-plane.
 \item Now partition the curve into 6 parts, using equally spaced time intervals. 
 Draw a straight line between the spots on the curve given by $\vec r(0)$, $\vec r(\pi/3)$, $\vec r(2\pi/3)$, etc.  
 You should have 6 straight lines connecting points on an ellipse. 
 The length of each segment you drew is called $\Delta s$ (an approximation to arc length). If we drew lots of tiny segments, we would use $\Delta s=ds$ to represent this length.  Why does $ds=\sqrt{(-2\sin t)^2+(3\cos t)^2}dt$?
 \item We'll call the $t$-coordinates of our partition $t_0$, $t_1$, $t_2$, $\ldots$. We have 
$$
t_0=0, \,
t_1=\pi/3,  \,
t_2=2\pi/3,  \,
t_3=\pi,  \,
t_4=4\pi/3,  \,
t_5=5\pi/3,  \,
t_6=2\pi.
$$
We need the surface area of the sheet that lies above the ellipse, but under the function $f(x,y)$. Above each little straight segment of length $\Delta s$, we could approximate the area by assuming height of $f(x,y)$ along the entire segment is the same as the height above the right endpoint, i.e. using a height of $f(\vec r(t_i))$. In a 3D picture, add the surface, ellipse, and 6 rectangles. See the Sage link above. 

The area of the first rectangle is \marginpar{We'll use $\sigma$ (a lower-case ``sigma'') to stand for surface area.}  $\Delta \sigma_1=f(\vec r(t_1))\Delta s_1$. The area of the second rectangle is $\Delta \sigma_2=f(\vec r(t_2))\Delta s_2$.  
Using our 6 rectangles, the total surface area of the sheet would approximately be the sum
\begin{equation*}
\begin{split}
&\Delta \sigma_1 + \Delta \sigma_2 + \Delta \sigma_3 + \cdots + \Delta \sigma_6\\
& \qquad =
f(\vec r(t_1))\Delta s_1+
f(\vec r(t_2))\Delta s_2+
f(\vec r(t_3))\Delta s_3+\cdots+
f(\vec r(t_6))\Delta s_6
\end{split}
\end{equation*}
Write this sum using sigma notation.

\item Instead of using 6 rectangles, let's now use $n$ rectangles, equally spaced by time. Let $\Delta s_i$ be the width of the $i$th rectangle. Use the time values $t_1,t_2,\ldots,t_n$ to find the heights of the $i$th rectangle. What is the surface area $\Delta \sigma_i$ of the $i$th rectangle? The sum of these $n$ little surface areas is approximately the total surface area under $f$ above $C$, i.e we have
$$\sigma \approx \Delta \sigma_1+\Delta \sigma_2+\cdots +\Delta \sigma_n.$$ Write this sum using sigma notation.


\item Use your sum from the previous part to explain why the area of the metal sheet that lies above $C$ and under $f$ is given by the integral
$$\sigma = \int_C (9-x^2-y^2)ds = \int_0^{2\pi}(9-(2\cos t)^2-(3\sin t)^2)\sqrt{(-2\sin t)^2+(3\cos t)^2}dt.$$
%What integral gives the area under $f$ over the curve $C$? Be prepared to share with the class how you can modify your sigma notation from the previous part to obtain the integral you gave here. (What happens to $\Sigma$ and $\Delta s$?)
%\item Find the surface area of the metal sheet. [Use technology to do this integral.]
\end{enumerate}
\end{problem}

Your results from the problem above suggest the following definition.
\begin{definition}[Line Integral]\marginpar{The line integral is also called the path integral, contour integral, or curve integral.}%
 Let $f$ be a function and $C$ be a piecewise smooth curve given by the parametrization $\vec r(t)$ for $t\in[a,b]$. We require that the composition $f(\vec r(t))$ be continuous for all $t\in [a,b]$. Then we define the line integral
of $f$ over $C$ to be the integral 
$$\int_C f ds 
= \int_a^b f(\vec r(t))\frac{ds}{dt}dt
= \int_a^b f(\vec r(t))\left|\frac{d\vec r}{dt}\right|dt.$$
\end{definition}

Notice that this definition suggests the following four steps.  These four steps are the key to computing any line integral. \marginpar{When we ask you to set up a line integral, it means that you should do steps 1--3, so that you get an integral with a single variable and with bounds that you could plug into a computer or do in Calculus~2.}%
\marginpar{Please compute all integrals we ask you to compute to get a numeric answer.  Compute the integrals by hand to practice basic integration techniques unless we say to use technology (i.e., calculator or computer).}%
\begin{enumerate}
 \item Start by getting a parametrization $\vec r(t)$ for $a\leq t\leq b$ of the curve $C$. 
 \item Find the speed by computing $\dfrac{d\vec r}{dt}$ and then $\left|\dfrac{d\vec r}{dt}\right|$.
 \item Multiply $f$ by the speed, and replace each $x$, $y$, $z$ with what it equals in terms of $t$.
 \item Integrate the product from the previous step.
\end{enumerate}

\note{We commented out a few problems here to cut down on the problems in this chapter.  They were originally put here to practice (a) parametrizing curves, (b) basic integration techniques, and (c) computing line integrals.  We'll try to practice these things in the problems in the rest of the chapter, though.}
%%%%%%%%%%%%%%%%%%%%%%%%%%%%%%%%%%%%%%%%%%%%%%%%%%IMPORTANT. SEE THE NOTE BELOW.
\note{There are 3 problems below that I used the first semester, but not the second.  I think it went way better when I included these problems.  They are really fast (just plug in numbers and compute), but they help the students get the hand of line integrals.  I suggest including them next time. I'll probably include these in the exact same problem, just have parts a,b,c.}

\begin{problem}\marginparbmw{See 16.1: 9-32.  Some problems give you a parametrization, some expect you to come up with one on your own.}%
  Let $f(x,y,z)=x^2+y^2-2z$ and let $C$ be two coils of the helix $\vec r(t)=(3\cos t, 3\sin t, 4t)$, starting at $t=0$. Remember that the parameterization means $x=3\cos t$, $y=3\sin t$, and $z=4t$.  Compute $\int_Cf ds$. [You will have to find the end bound yourself. How much time passes to go around two coils?]
\end{problem}

\begin{problem}\marginparbmw{To practice matching parameterizations to curves, try 16.1:1-8.}%
 Consider the function $f(x,y)=3xy+2$. Let $C$ be a circle of radius 4 centered at the origin.  Compute $\int_C fds$.  [You'll have to come up with your own parameterization.]
\end{problem}


\begin{problem}\marginpar{If you've forgotten how to parametrize line segments, see \ref{first line between two points}.}%
 Let $f(x,y,z)=x^2+3yz$. Let $C$ be the straight line segment from $(1,0,0)$ to $(0,4,5)$. Compute $\int_C f ds$. 
\end{problem}

\begin{problem}\marginpar{See \ref{parameterizing plane curves} if you forgot how to parametrize plane curves.}%
\larsonfive{\marginpar{See Larson 15.2: 1--20, 63--70.  Some problems give you a parameterization, some expect you to come up with one on your own.}}%
 Let $f(x,y)=x^2+y^2-25$. Let $C$ be the portion of the parabola $y^2=x$ between $(1,-1)$ and $(4,2)$. We want to compute $\int_C fds$.  
\begin{enumerate}
\item Draw the curve $C$ and the function $f(x,y)$ on the same 3D $xyz$ axes.
\item Without computing the line integral $\int_C fds$, determine if the integral should be positive or negative. Explain why this is so by looking at the values of $f(x,y)$ at points along the curve $C$.  Is $f(x,y)$ positive, negative, or zero, at points along $C$?
 \item Parametrize the curve and set up the line integral $\int_C f ds$. [Hint: if you let $y=t$, then $x=$? What bounds do you put on $t$?]
 \item Use technology to compute $\int_C fds$ to get a numeric answer.  Was your answer the sign that you determined above?
\end{enumerate}
\end{problem}

\section{Average Value}

The concept of averaging values together has many applications.  In first-semester calculus, we saw how to generalize the concept of averaging numbers together to get an average value of a function.  We'll review both of these concepts. Later, we'll generalize average value to calculate centroids and center of mass.

\begin{problem}\label{average value methods}
\instructor{Throughout the section, I point out how each formula is a variation on one of the patterns below.}
  Suppose a class takes a test and there are three scores of 70, five scores of 85, one score of 90, and two scores of 95.  We will calculate the average class score, $\bar s$, four different ways to emphasize four ways of thinking about the averages.  We are emphasizing the pattern of the calculations in this problem, rather than the final answer, so it is important to write out each calculation completely in the form $\bar s = \blank{1cm}$ before calculating the number $\bar s$.
  \begin{enumerate}
  \item \marginpar{$\bar s=\frac{\sum \text{values}}{\text{number of values}}$}%
 Compute the average by adding 11 numbers together and dividing by the number of scores.   Write down the whole computation before doing any arithmetic.
  \item \marginpar{$\bar s=\frac{\sum (\text{value}\cdot\text{weight})}{\sum \text{weight}}$}%
Compute the numerator of the fraction in the previous part by multiplying each score by how many times it occurs, rather than adding it in the sum that many times.  Again, write down the calculation for $\bar s$ before doing any arithmetic.
  \item \marginpar{$\bar s=\sum (\text{value}\cdot\text{(\% of stuff)})$}%
Compute $\bar s$ by splitting up the fraction in the previous part into the sum of four numbers.  This is called a ``weighted average'' because we are multiplying each score value by a weight.
  \item \marginpar{$\text{(number of values)}\bar s = \sum \text{values}$\\ $(\sum \text{weight})\bar s = \sum (\text{value}\cdot\text{weight})$}%
Another way of thinking about the average $\bar s$ is that $\bar s$ is the number so that if all 11 scores were $\bar s$, you'd have the same sum.  Write this way of thinking about these computations by taking the formulas for $\bar s$ in the first two parts and multiplying both sides by the denominator.
  \end{enumerate}
\end{problem}

In the next problem, we generalize the above ways of thinking about averages from a discrete situation to a continuous situation.  You did this in first-semester calculus when you did average value using integrals.

\begin{problem}
 Suppose the price of a stock is \$10 for one day.  Then the price of the stock jumps to \$20 for two days.  Our goal is to determine the average price of the stock over the three days.
\begin{enumerate}
 %\item Find the average price of the stock using all of the methods from Problem~\ref{average value methods}.
 \item Let $f(t) = \begin{cases}10 &0<t<1\\20&1<t<3\end{cases}$, the price of the stock for the three-day period. Draw the function $f$, and find the area under $f$ where $t\in[0,3]$.
 \item Find a single constant $\bar f$ so that the areas under both $\bar f$ and $f$, above the interval $[0,3]$, are the same numbers.  [Hint: The area under $\bar f$ is just the area of a rectangle.]
 \item We found a constant $\bar f$ so that the area under $\bar f$ matched the area under $f$. In other words, we solved the equation below for $\bar f$: 
$$\int_a^b \bar f dx = \int_a^b f dx$$
  Solve for $\bar f$ symbolically, without doing any of the integrals. This quantity is called the average value of $f$ over $[a,b]$.%, and was crucial to proving the Fundamental Theorem of Calculus from first-semester calculus.
\item \instructor{I also write $\bar f=\int_a^b f \frac{dx}{\int_a^b dx}$ to emphasize the weighted average approach}%
 The formula for $\bar f$ in the previous part resembles at least one of the ways of calculating averages from Problem~\ref{average value methods}.  Which ones and why?
\end{enumerate}
\end{problem}

\instructor{I talk about ants building a mound. Then after removing the ants, so none get hurt, you shake their tank.  The average value is the height of the dirt. The mountains filled in the valleys.}%
Ask me in class about the ``ant farm'' approach to average value. 

\begin{problem}\label{Average Value intro}%
\marginpar{\href{http://www.youtube.com/watch?v=t7T0MzfgV0Q&list=PL04DF68E73B7ECD54&index=5&feature=plpp_video}{Watch a YouTube video.}}%
 Let the curve $C$ have the parametrization $\vec r(t) = (2\cos t, 3\sin t)$.  Let $f$ be the function $f(x,y)=9-x^2-y^2$.    \begin{enumerate}
  \item Draw the surface $f$ in 3D.  Add to your drawing the curve $C$ in the $xy$ plane. Then draw the sheet whose area is given by the integral $\int_C f ds$. 
  \item What's the maximum height and minimum height of the sheet? \marginpar{See problem \ref{horse track chain rule introduction}.}
  \item We would like to find a constant height $\bar f$ so that the area under $f$ above $C$, is the same as area under $\bar f$, above $C$. What integral gives us the area under $\bar f$ above $C$?  What integral gives us the area under $f$ above $C$?  
%  \item 
\marginparbmw{Please read 
\href{https://www.lds.org/scriptures/ot/isa/40.4?lang=eng\#3}{Isaiah 40:4} and
\href{https://www.lds.org/scriptures/nt/luke/3.5?lang=eng\#4}{Luke 3:5}. These scriptures should help you remember how to find average value.
}%
\instructor{Again, I emphasize how this relates to the ways of computing averages from Problem~\ref{average value methods}.}
Explain why the average value of $f$ along $C$ is 
$$\bar f = \frac{\int_C f ds}{\int_C ds}.$$
Connect this formula with the ways of thinking about averages from Problem~\ref{average value methods}.

  \item Use a computer to evaluate the integrals $\int_C f ds$ and $\int_C ds$, and then give an approximation to the average value of $f$ along $C$. Is your average value between the maximum and minimum of $f$ along $C$? Why should it be?
 \end{enumerate}
\end{problem}

\begin{problem}\instructor{After this problem, I like to emphasize that they should have noticed a linear growth rate, and then I show them how I would have guessed the answer.}%
 The temperature $T(x,y,z)$ at points on a wire helix $C$ given by $\vec r(t) = (\sin t, 2t, \cos t)$ is known to be $T(x,y,z)=x^2+y+z^2$. What are the temperatures at $t=0$, $t=\pi/2$, $t=\pi$, $t=3\pi/2$ and $t=2\pi$?  You should notice the temperature is constantly changing.  Make a guess as to what the average temperature is (share with the class why you made the guess you made---it's OK if you're wrong). Then compute the average temperature of the wire using the integral formula from the previous problem.
\end{problem}






















\section{Work, Flow, Circulation, and Flux}

We now look at an exciting application of line integrals.  This application helps us study the transfer of energy (work), as well as understanding the flow of air along a wing (circulation) and the flow of a fluid across a surface (flux).  

Let's start with a review of work. As an object moves through a vector field, energy transfer occurs.  When an object falls from high place, potential energy is transfered to kinetic energy. The gravitational vector field is the field which does work. Prior to problem \ref{first work problem} on page \pageref{first work problem}, we made the following statements. 
\begin{quote}If a force $F$ acts through a displacement $d$, then the most basic definition of work is $W=Fd$, the product of the force and the displacement.  This basic definition has a few assumptions.
\begin{itemize}
\item The force $F$ must act in the same direction as the displacement.
\item The force $F$ must be constant throughout the  displacement.
\item The displacement must be in a straight line.
\end{itemize}
\end{quote}
We used the dot product to remove the first assumption, and we showed in problem \ref{first work problem} that the work is simply the dot product $$W=\vec F\cdot \vec r,$$
where $\vec F$ is a force acting through a displacement $\vec r$. We now remove the other two assumptions so that we can deal with variable forces acting on objects moving along somewhat arbitrary curves.  We will use the unit tangent vector $\vec T$ to a curve $\vec r$ that was introduced in Definition~\ref{def unit tangent vector}.  Recall that 
$$\vec T = \ds\frac{d\vec r}{ds}=\ds\frac{d\vec r/dt}{ds/dt} = \frac{d\vec r/dt}{|d \vec r/dt|}.$$

\begin{problem}[Work]
\marginpar{\href{http://www.youtube.com/watch?v=9TGZIIpEaHw&list=PL04DF68E73B7ECD54&index=2&feature=plpp_video}{Watch a YouTube video}.}%
 Let $\vec F(x,y)=(M,N)$ be a vector field, where $M$ and $N$ are functions of $x$ and $y$.   Let $C$ be a curve parametrized by $\vec r(t)=(x,y)$, where $x$ and $y$ are functions of $t$ and $t\in[a,b]$. 
\begin{enumerate}
 \item Draw a random curve on your paper.  Cut the curve $\vec r(t)$ into lots of little segments. Each little segment has a length, which we call $ds$. If your segments are really small, then $\vec F$ is almost constant on this segment.  Explain why the work done by $\vec F$ along this tiny segment is approximately 
$$d(\text{Work}) = \vec F\cdot (\vec T ds).$$
 \item Explain why the work done by $\vec F$ along $C$ is 
$$
W=\int_C \vec F\cdot \vec T ds 
= \int_a^b\vec F\cdot \frac{d\vec r}{dt}dt 
= \int_a^b M\frac{dx}{dt}+N\frac{dy}{dt} dt.
$$
 \item If you are familiar with the units of energy, complete the following. What are the units of $\vec F$,  $\vec T$, $ds$, and $d\text{Work}$.
\end{enumerate}
\end{problem}



The work done by a vector field may show up in any of the following ways: \marginpar{When working with a vector field in space, we often use the notation $\vec F(x,y,z) = (M,N,P)$, so we often write work as $\int_C Mdx+Ndy+Pdz$.}
\begin{align*}
W
&=\int_C \vec F\cdot \vec T \ ds\\
&=\int_C \vec F\cdot \frac{d\vec r}{ds}\ ds \\
&= \int_C\vec F\cdot d\vec r \\
&=\int_a^b \vec F\cdot \frac{d\vec r}{dt}\ dt \\
&= \int_C Mdx+Ndy \\
&= \int_a^b \left( M\frac{dx}{dt}+N\frac{dy}{dt}\right) dt. 
\end{align*}
Notice that only two integrals above have the bounds $a$ and $b$.  These two integrals are the actual formula used to compute the integral. The others are just symbolic ways to remember the integral.

\begin{definition}[Flow and Circulation]
If the vector field $\vec F$ represents the velocity field of a fluid, such as airflow along a wing (so units are m/s), then the work integral is often called flow.  If the start and stop point for the curve are the same, then we'll call the the work integral {\it circulation}. In this case, we'll often add a circle to the integral, as in $\ds\oint_C \vec F\cdot d\vec r$, to emphasize that the integral is along a closed curve.  In most cases, we'll be computing circulation along curves in the counterclockwise direction.  If we want to emphasize the direction we are going along a closed curve, we'll use an arrow on the small circle on the integral sign.
\end{definition}

\begin{definition}[Simple Closed Curve]
 If $C$ is a smooth curve, and the start and end points of $C$ are the same, we call $C$ a closed curve.  If the closed curve does not intersect itself, we call the curve a simple closed curve.
\end{definition}


\begin{problem}
 Let $\vec F(x,y)=(M,N)$ be a vector field.  Let $C$ be a simple closed curve parametrized by $\vec r(t)=(x(t),y(t))$. Then $d\vec r = (dx, dy)$.
\begin{enumerate}
 \item Draw a simple closed curve to represent $\vec r$ (just draw any simple closed curve you like).  We know that the circulation (work) along $C$ is given by the integral $\int_C (M,N)\cdot (dx,dy)$.  Pick a spot on your curve, and draw a tangent vector to represent $(dx,dy)$ so that you traverse the curve using a counterclockwise orientation.
 
 \item What is the angle between $(dx,dy)$ and $(-dy,dx)$?  What is the angle between $(dx,dy)$ and $(dy,-dx)$? [Hint: see Problems~\ref{dot angle formula} and \ref{dot angle practice}.]
\item Which vector, $(-dy,dx)$ or $(dy,-dx)$, points towards the outside of the curve if we are going around the curve counterclockwise? Pick a few points on your curve and explain why $(dy,-dx)$ always points towards the outside the curve.
\end{enumerate}
\end{problem}

When we compute the work done by a vector field along a curve, we are focusing on how much of the vector field points in the direction of motion.  The integral $\oint_C \vec F \cdot \vec T ds$ measures the flow along a curve.  If we let $\vec n$ be the outward pointing normal vector to the curve, then the integral $\oint_C \vec F \cdot \vec n ds$ measures the flow outward across a curve. We'll define this outward flow as ``Flux.''

\begin{definition}[Flux]
\marginpar{\href{http://www.youtube.com/watch?v=5DNdI72XEYY&list=PL04DF68E73B7ECD54&index=4&feature=plpp_video}{Watch a YouTube video}.}%
 Let $\vec F(x,y)=(M,N)$ be a vector field.  Let $C$ be a simple closed curve parametrized by $\vec r(t)=(x,y)$, and oriented in the counterclockwise direction. 
 We know the circulation of $\vec F$ along $C$ (the flow of $\vec F$ along $C$) is the integral $\oint_C \vec F\cdot \vec T\ ds = \oint_C (M,N)\cdot(dx,dy)$. The outward flux of $\vec F$ across $C$ is the line integral
$$\text{Flux}=\Phi = \oint_C \vec F\cdot \vec n\ ds = \oint_C (M,N)\cdot(dy,-dx) = \oint_C Mdy-Ndx.$$
The flux of $\vec F$ measures the outward flow of $\vec F$ across $C$ instead of along $C$.
The vector $\vec n ds = (dy,-dx)$ is correct if the curve is oriented in the counterclockwise direction (which was shown in the previous problem).
\end{definition}

We're now prepared to compute both work (circulation, flow) and flux.  The next 4 problems ask you to do so. The most common way to remember these, provided the vector field is $\vec F(x,y) = (M(x,y),N(x,y))$, is 
$$\text{Work}=\int_C M dx+N dy\quad\quad \text{Flux}=\int_C M dy -N dx.$$

\note{After 3 semesters of trying this, I need to rewrite the section again.  I put lots of notes in my notebook for this semester. I'll rewrite the section soon.}

\begin{problem}
\marginpar{If you haven't yet, please watch the YouTube videos for 
\href{http://www.youtube.com/watch?v=9TGZIIpEaHw&list=PL04DF68E73B7ECD54&index=2&feature=plpp_video}{work} and 
\href{http://www.youtube.com/watch?v=5DNdI72XEYY&list=PL04DF68E73B7ECD54&index=4&feature=plpp_video}{flux}.}%
\instructor{See \href{http://aleph.sagemath.org/?z=eJy1VFFv2jAQfvevOFEkHGrYuqnVNMnSNmk8td1UVeKBIpTGDlgNdnR2KPn3O8dpC6x7XB6wuXzf3Xff2dnlyEd70YowythM7nQRHPLFpBX7ZcbwNXA5LpznIRNwOfbG0o5eBx9yDPIjC9oq-WlcG8YSYVUaXanVo2us8iCB872YXIkr4vM27TLG2Blc6zDysM2fNORQNrYIxllAs94EcGUJwYHe5VWTBwJYMDboNeYVuJ1GCBvjoWhwp6dM6RIqY_XqBcL7jVXZVwb0oA4NWri747bZajRFXr2BD9BT3zx6vpe4-LgU0NJ6sSTdqVlaqdeMXpF-hVKZsuQoQsbUXM5OWWOFcAY3TRVMXbXGriE8O0gWeeptrcOG-libnaa_Gw3KBajRqaYIbC6P-1FzKkl2kvfRURUrCJgo7MSoWdXs31FABNa9OkkWY2kE999_XP9km7CtpiF_rDRfsAUOhg8kEyifaLPhQMBsKd7CSBpiEPtgGAJZt9YU4uHEqh6iEjGyFB7mmvHXlKnQaQsvCeYy4eGhiDYdJpz3GIJQh6D69V_4-WF9C8rHYHL2pVhyMzEjghy_AdXCBG5B7buiEfLK6PDDVL3bd9vjBF3oRrWT25Qh4pYsHqQz-H396559i-PBnGYfT_OKk41By8WgrNzzQNDS7AfL_jTXsq5cWB1eNz4TMH7n_mWJcC6hzjHf6kBnfxXZdHD_npegW1U8We29_JyYpoSkRCYlSUB8Sodg6FKC78bPDxPF1LRMUiz7cPEleyN2goAU5YjumdPwpYnfBjyH6SXdmiwFClc5lCPUapSk6OpIDPnx38VEA4_UrFFr2-vpfvyGWLWA3Ndk_gpz-obJi-wPG3Sl-A}{Sage}.}%
 Consider the rotational field $\vec F=(-y,x)$ and the circle $C$ of radius 5 parametrized by $\vec r(t)=(5\cos t, 5\sin t)$ for $t\in[0,2\pi]$.  
\begin{enumerate}
 \item Draw the curve $C$ and vector field $\vec F$ on the same axes.
 \item Compute the circulation (work) of $\vec F$ along $C$.
 \item Compute the outward flux of $\vec F$ along $C$.
 \item Can you explain why one of these integrals must be zero, and the other must be positive? We'll answer this in class if you are unable.
\end{enumerate}

\end{problem}

\begin{problem}\instructor{See \href{http://aleph.sagemath.org/?z=eJy1VFFv2jAQfvevOFEkHGpYaTtpmmRpmzSe2m6qKvFAEUpjB6wGOzo7lPz7nWPaAuselwdsLt939913drY58sFOtCIMMjaVW10Eh3x-OdyJy2G7yBi-xa6GhfM8ZAKuht5Y2tHr4EOOQV6woK2Sl8PaMJYIy9LoSi2fXGOVBwmc78ToWlwTn7dplzHGzuBGh4GHTf6sIYeysUUwzgKa1TqAK0sIDvQ2r5o8EMCCsUGvMK_AbTVCWBsPRYNbPWZKl1AZq5evEL7fWJV9ZUAP6tCghft7bpuNRlPk1Tv4AD32zZPnO4nzi4WAltbJgnSnZmmlXjN6RfoVSmXKkqMIGVMzOT1lDRXCGdw2VTB11Rq7gvDiIFnkqbeVDmvqY2W2mv6uNSgXoEanmiKwmTzuR82oJNlJ3kdHVawgYKSwE6OmVbP7QAERWPfqJFmMpRE8fP9x85Otw6Yah_yp0nzO5tjrP5JMoHyizfo9AdOFeA8jaYhB3AdDH8i6laYQDydW7SEqESNL4WGuKX9LmQqdtvCaYCYTHh6LaNNhwtkeQxDqENR-_Rd-dljfgvIxmJx9LZbcTMyIIMdvQbUwgjtQu65ohLwxOnw_Ve_23fY4QRe6Ve3oLmWIuAWLB-kMft_8emDf4ngwp9nH07zkZGPQct4rK_fSE7Q0u95if5prWVcuLA-vG58KGH5w_7JEOJdQ55hvdKCzv4xsOrh_z0vQrSqerfZeXiWmKSEpkUlJEhCf0iEYupTgu_Hzw0QxNS2jFMs-Tb5k78ROEJCiHNG9cBq-NPHbgOcw_ky3JkuBwlUO5QC1GiQpujoSQ378dzHRwCM1K9Ta7vV0P35NrFpA7msyf4k5fcPkJPsD0wOmdw}{Sage}.}%
 Consider the radial field $\vec F=(2x,2y)$ and curve $C$ parametrized by $\vec r(t)=(3\cos t, 3\sin t)$ for $t\in[0,2\pi]$.  
\begin{enumerate}
 \item Draw the curve $C$ and vector field $\vec F$ on the same axes.
 \item Compute the circulation (work) of $\vec F$ along $C$.
 \item Compute the outward flux of $\vec F$ along $C$.
 \item Can you explain why one of these integrals must be zero, and the other must be positive? We'll answer this in class if you are unable.
\end{enumerate}

\end{problem}

From the previous two problems you might ask, ``Are there vector fields where the work and flux can both be nonzero?''  The next problems answer this in the affirmative. The previous two problems just dealt with a rotation field (where the vector field only rotates things, does not push in or out, so zero flux) and a radial field (which only pushes out, so no circulation).

\begin{problem}
\marginpar{\href{http://www.youtube.com/watch?v=6WcN36FbeWc&list=PL04DF68E73B7ECD54&index=3&feature=plpp_video}{Watch a YouTube video}.  Also, see \href{http://aleph.sagemath.org/?z=eJw9yjEKhDAQRuF-TzKDvxLTp80lgohodMMORjZBnNtvqm0-XvE8PVB21CueTvl1Sa7zHdeav_Oeomzk0ZZ-GGGHkUH6b-6uLHrkk0IwsBOChWma5oQ9iTi_SImo77R-zliKs_wDRwwhgQ}{Sage} for a picture.}%
\instructor{Answers---\href{http://aleph.sagemath.org/?q=38b590f1-dc52-4690-975f-ccde49161dcf}{Sage}.
%http://aleph.sagemath.org/?z=eJy9Vd9r2zAQfs9fcaSByK2c1S2DURBsK8tT241SyIMbghvJiakjh7Ps2P_9TlKa2GnL3paHSLof3313J53rBNm44S0342AwFbVamgJZHLa8uWjnwaA6iC75JZ3rw_nKnXcd_RWdB2dg1qgUyCxNFSptIM-0glKtNnQoBxiJmNVhFZybi4oDM_ySR8F8gFck34W1lddd-TXJq3Bn5buOfODjLtJM5XLxUlRaliCAsYaHEb8OyLT1OyKFeVYaEWPE8YrjNXl_z7RRmCzNQKoUFuxWOJvgZgD0Q05xTGkSNByM0jIQt05xBnfKjEvYJK8KEkgrvTRZoQGz1dpAkaZgClB1kleJIQMNNswKkxyKWiGVJithWWGtJg7OxrblWbyZsf2GInomjo0yFWp4fGS62ijMlkl-dOh4TMrqpWSNwPhyzqGlNZpTHXp5xLZpFtTHR2H7xJAbL5UzMT1FOJdIed9Xucm2eZvpFZhdAb78JeW7UmZNua2yWtFxTa0vDGyxkBVV14LORD9HOetQoLYxE9jOSRuNQyjxQFJO86r5gBE5Ob1Tn4BbWQd_37anHz_vfrn92mzyiUlecsViJ4hxOHqmdIDi8DYYDTlM57yvQuJoFdhRmBFQ2Vdq6G5lv8wdM-kBrLfEU9wpO8D7wKepdoFmwvvA89KWuAs869iRGVXDttKtn_nMTrlokKVV-I50A_sueARrRd26B9lCCA8gG0fAmvS8nM_IM3F7t-2DONG9bMMHj2Lt5g7AThIaJX_ufj-9e6pUZ6NEPNSFptoP07zYuaVqhvP9q9mKbV6YRXdEsCmH8w9mhr8qaYH2MdNzBTcHOo_PdXffXD8KyPCgfveC_nmnHb8LAdsEk40y9JwXlizDD64Rp4GxfNWqLMX10TtLwddA-OSPZN9SyWwipbudrAtoQ9ASelnwJfoW9J0dOSB2CWKxY3QvRWZHKV7A5CsNgsALlkVeoBijkuMjLZX3iFE7_hsx28cesxV9gvSeW7PJNG82SSM-aD-1JY5u_KVrrWH7mWF0NCzXRGHLISm3ZLjAhL4BIuI2kDhGs3_cYoojsP0L_gLPyDNp
}%
Let $\vec F=(-y,x+y)$ and $C$ be the triangle with vertices $(2,0)$, $(0,2)$, and $(0,0)$.
\begin{enumerate}
\item Look at a drawing of $C$ and the vector field (see margin for the Sage link).  If we go counterclockwise around the triangle, for each side of the triangle, guess the signs of the counterclockwise circulation and the flux (positive, negative, zero).
\item Find the counterclockwise circulation (work) done by $\vec F$ along $C$.  You'll have three separate calculations, one for each side.  You'll need to parametrize three line segments.
\end{enumerate}
\end{problem}

\begin{problem}\marginpar{See \href{http://aleph.sagemath.org/?z=eJxNjEEOQDAUBfdO8j-vSZWtrUs0ZUFFo0FopL09VmwnM9NSROKGVB5FQuRs91voLzuE7egnZ_1ILR5HKJQMSpComQvvVktaP1Qa6BKVMQizG5bVnmejuHg3VIvYqa_-CzduCCO3}{Sage}.}%
\instructor{Answers---\href{http://aleph.sagemath.org/?q=ec0be342-7afb-4cb7-8af1-c5da03352d28}{Sage}.
%http://aleph.sagemath.org/?z=eJy9VUuP2zYQvutXDJwFTO5Sm5XdAEUAAm2D-rSbFkEAHxTX4JqULUQWjRHllf59h6RjS84GucUHkZzHN988SB8VsmkneuGmPFnIo9k4iyyf3XZpL7oVT9qzLJ2JBxIcz4JMzP05nNZFaSq9frZtrRuQwFgn0rmYcQGsF2km3nGeJJjJnB3Tlt-6u5Y0TjyIjK-SN9BJ93YOjQWHqt4aKBtgXpfgTOZnBnPyy8RvKYs7_t9sxQcwCVZl42SOmcAZHf8oa2dQbVyiTQFr9kEGA_4-AfqhIEfXOIVOgDO15vJDULyBR-OmDezVVwMKirbeuNLWgOV258AWBTgL5qiqVjkyqMGH2aKqwB4NgtsR-U2LR3Mf4HzsqqzN-psZO20oYmQS2BjXYg2fPrG63RssN6q6OAw87pv2uWGdxPxhJaCnNfM1GOWR-z550BgfpS6LgqFwUaqXcnGNcKuR8n5qK1ceqr6st-BeLMTCN5Tv1rgd5bYtj4aOOwPaOjig1S1V14Mu5ThHvRxQoJlgjvux0D6agFTjmaReVG33CiNyCvqgvgL3sgH-qW2f__zr8e-w37l9de_Uc2VYHgQ5Tm6-UDpAcUTPbyYCFisxViFx9AocKNwNhIGchDEbl3lgpiOA99Z4jbtgZ_gY-DrVIdBSRh_4svElHgIvB3ZkRtXwrQzrj3yW11xq0I1XxI4MA8cuRARvRd16At1DCh9Bd4GANxl5BZ-byCTsw3YMEkRPuk8_RhRvtwoA1P-Erv6_j_98_u6qUp2dkfmktjXVflJU9iUsbTdZnW7NQR4q69bD94ctBNy-8iDFUSks-stM1xXCOzC4fKG7p-bGp4AMz-rvbtBPZzrwu5NwUKj2xtF1XnuyDF8ZI0EPxuZrbZpGzi_eZQGxBjImfyH7LZXSJ9KE6WRDQB-CljTK-Nvsdz52DuSA2ClE-8JoLmXpn1G8g_t39BDwKNjYyqKcotHTCy1TjYhRO34ZMd_HEbMtGlOfuHX7shbdXnXylfZTW_LsfRy63hv2PzLMLobNjigcBKjmQIZrVPQfIDPhA8lLNP8RHlNegP2H_w9xpCjx
Again, we'll discuss why some are positive, and some are negative. I want to emphasize flow in, and flow out. One is clearly positive, the other clearly negative.  The overall sum is positive. It should be obvious with a picture.}%
 Consider the vector field $\vec F=(2x-y,x)$. Let $C$ be the curve that starts at $(-2,0)$, follows a straight line to $(1,3)$, and then back to $(-2,0)$ along the parabola $y=4-x^2$.  

\begin{enumerate}
\item Look at a drawing of $C$ and the vector field (see margin for the Sage link).  If we go counterclockwise around $C$, for each part of $C$, guess the signs of the counterclockwise circulation and the flux (positive, negative, zero).
\item Find the flux of $\vec F$ across $C$.  There are two curves to parametrize. Make sure you traverse along the curves in the correct direction.
\end{enumerate}
\end{problem}

\note{In both of the preceding problems, both the circulation and flux are positive.  Everything is always positive. This would give a student a false impression that work and flux are always positive.  We need to change some computations.  Make sure to look into this. It needs to change.}





























\section{Physical Properties}

A number of physical properties of real-world objects can be calculated using the concepts of averages and line integrals.  We explore some of these in this section.  Additionally, many of these concepts and calculations are used in statistics.

\subsection{Centroids}%

\begin{definition}[Centroid]
  Let $C$ be a curve. If we look at all of the $x$-coordinates of the points on $C$, the ``center'' $x$-coordinate, $\bar x$, is the average of all these $x$-coordinates.  Likewise, we can talk about the averages of all of the $y$ coordinates or $z$ coordinates of points on the function ($\bar y$ or $\bar z$, respectively).  The \emph{centroid} of an object is the geometric center $(\bar x, \bar y, \bar z)$, the point with coordinates that are the average $x$, $y$, and $z$ coordinates.
\end{definition}

\begin{problem}[Centroid]\label{centroid of curve}\marginpar{\href{http://www.youtube.com/watch?v=t7T0MzfgV0Q&list=PL04DF68E73B7ECD54&index=5&feature=plpp_video}{Watch a YouTube video.}}%
  Notice the word ``average'' in the definition of the centroid. Use the concept of average value to explain why the coordinates of the centroid are %the formulas below. [Hint:  If we have a curve $C$ with parametrization $\vec r(t)$ and function $f$ so that $f(\vec r(t))$ is continuous, then we've developed in Problem~\ref{Average Value intro} a formula for the average value of $f$ along $C$.  What function $f(x,y,z)$ gives the $x$-coordinate of a point?]
\marginpar{Formulas for the centroid.}
$$
\bar x = \frac{\int_C x ds}{\int_C  ds},\quad
\bar y = \frac{\int_C y ds}{\int_C  ds},\quad 
\text{and}\quad
\bar z = \frac{\int_C z ds}{\int_C  ds}.
$$
Notice that the denominator in each case is just the arc length $s=\int_C ds$. 
\end{problem}


\begin{problem}\label{semicircle centroid}
 Let $C$ be the semicircular arc $\vec r(t)=(a\cos t, a\sin t)$ for $t\in[0,\pi]$. Without doing any computations, make an educated guess for the centroid $(\bar x, \bar y)$ of this arc.  Then compute the integrals given in problem \ref{centroid of curve} to find the actual centroid. Share with the class your guess, even if it was incorrect. 
\end{problem}

\subsection{Mass and Center of Mass}
\note{Jason made a comment that aluminum and copper won't combine.  So this should be changed eventually.}
Density is generally a mass per unit volume.  However, when talking about a curve or wire, as in this chapter, it's simpler to let density be the mass per unit length.  Sometimes an object is made out of a composite material, and the density of the object is different at different places in the object. For example, we might have a straight wire where one end is aluminum and the other end is copper. In the middle, the wire slowly transitions from being all aluminum to all copper.  The centroid is the midpoint of the wire.  However, since copper has a higher density than aluminum, the balance point (the center of mass) would not be at the midpoint of the wire, but would be closer to the denser and heavier copper end.  In this section, we'll develop formulas for the mass and center of mass of such a wire. Such composite materials are engineered all the time (though probably not our example wire).  \bmw{In future mechanical engineering courses, you would learn how to determine the density $\delta$ (mass per unit length) at each point on such a composite wire.}

\begin{problem}[Mass]\label{mass of curve}%
\marginpar{\href{http://www.youtube.com/watch?v=mz-Udq5TeS4&list=PL04DF68E73B7ECD54&index=6&feature=plpp_video}{Watch a YouTube video.}}%
 Suppose a wire $C$ has the parameterization $\vec r(t)$ for $t\in[a,b]$.  Suppose the wire's density at a point $(x,y,z)$ on the wire is given by the function $\delta(x,y,z)$. \bmw{You'll learn to calculate this function in a future class. For the purposes of our class, we'll just assume we know what $\delta(x,y,z)$ is.}
 \begin{enumerate}
  \item Consider a small portion of the curve at $t=t_0$ of length $ds$.  Explain why the mass of the small portion of the curve is $dm=\delta(\vec r(t_0)) ds$.
  \item Explain why the mass $m$ of an object is given by the formulas below (explain why each equals sign is true):
$$m=\int_C dm = \int_C \delta ds = \int_a^b \delta(\vec r(t)) \left|\frac{d\vec r}{dt}\right|dt.$$
 \end{enumerate}
\end{problem}

\begin{problem}\instructor{I found many students struggled with setting up a really simple sum.  In class, after they present this one, I would suggest actually taking time to show them how to write the problem in summation notation with 2 points.  It will prepare them for the proof of center of mass coming up.}%
\larsonfive{\marginpar{See Larson 15.2:23--26 for more practice.}}%
 A wire lies along the straight segment from $(0,2,0)$ to $(1,1,3)$.  The wire's density (mass per unit length) at a point $(x,y,z)$ is $\delta(x,y,z)=x+y+z$. 
 \begin{enumerate}
 \item Is the wire heavier at $(0,2,0)$ or at $(1,1,3)$?
 \item  What is the total mass of the wire?  [You'll need to parameterize the line as your first step---see Problem~\ref{first line between two points} if you need a refresher.]
 \end{enumerate}
\end{problem}

\instructor{Here I introduce the center of mass of an object, talk about moments, and talk about how the center of mass formula is also an averaging of the coordinates, where the weight is the amount of mass at a particular coordinate value (i.e., $xdm$).}

\marginpar{\href{http://en.wikipedia.org/wiki/Center_of_mass}{Wikipedia} has some interesting applications of center of mass.}%
The center of mass of an object is the point where the object balances.  In order to calculate the $x$-coordinate of the center of mass, we average the $x$-coordinates, but we weight each $x$-coordinate with its mass.  Similarly, we can calculate the $y$ and $z$ coordinates of the center of mass.

The next problem helps us reason about the center of mass of a collection of objects.  Calculating the center of mass of a collection of objects is important, for example, in astronomy when you want to calculate how two bodies orbit each other.

\begin{problem}\label{center of mass with two points}\instructor{After a student presents, this is a great time to connect the averages back to Problem~\ref{average value methods}.  I point out that we can think about the object as 2 points of the same mass at $P_1$ and 3 points of the same mass at $P_2$.  This suggests averaging 5 things with method 1.  Alternately, I suggest the approach $\frac{2}{5}P_1+\frac{3}{5}P_2$, suggesting a weighted average.}%
 Suppose two objects are positioned at the points $P_1=(x_1,y_1,z_1)$ and $P_2=(x_2,y_2,z_2)$.
 Our goal in this problem is to understand the difference between the centroid and the center of mass.
\begin{enumerate}
\item Find the centroid of two objects.
 \item Suppose both objects have the same mass of 2 kg.  Find the center of mass by averaging the $x$, $y$, and $z$ coordinates, weighted by how much mass is at each coordinate.
 \item If the mass of the object at point $P_1$ is 2 kg, and the mass of the object at point $P_2$ is 3 kg, will the center of mass be closer to $P_1$ or $P_2$? Give a physical reason for your answer before doing any computations.  Then find the center of mass $(\bar x, \bar y, \bar z)$ of the two points. [Hint: you should get $\bar x= \frac{2x_1+3x_2}{2+3}$.] 
\end{enumerate}
\end{problem}


\begin{problem}
 This problem reinforces what you just did with two points in the previous problem. However, it now involves two people on a seesaw. \marginpar{See \href{http://en.wikipedia.org/wiki/Seesaw}{Wikipedia} for a seesaw picture.}%
Ignore the mass of the seesaw in your work below (pretend it's an extremely light seesaw, so its mass is negligible compared to the masses of the people).
\begin{enumerate}
 \item 
 My daughter and her friend are sitting on a seesaw.  Both girls have the same mass of 30 kg. My wife stands about 1 m behind my daughter. We'll measure distance in this problem from my wife's perspective.  We can think of my daughter as a point mass located at $(1\text{m},0)$ whose mass is $30$ kg. Suppose her friend is located at $(5\text{m},0)$. Suppose the kids are sitting just right so that the seesaw is perfectly balanced.  That means the the center of mass of the girls is precisely at the pivot point of the seesaw. Find the distance from my wife to the pivot point by finding the center of mass of the two girls. 
 \item My daughter's friend has to leave, so I plan to take her place on the seesaw. My mass is 100 kg. Her friend was sitting at the point $(5,0)$. I would like to sit at the point $(a,0)$ so that the seesaw is perfectly balanced. Without doing any computations, is $a>5$ or $a<5$? Explain.
 \item Suppose I sit at the spot $(x,0)$ (perhaps causing my daughter or I to have a highly unbalanced ride). Find the center of mass of the two points $(1,0)$ and $(x,0)$ whose masses are $30$ and $100$, respectively (units are meters and kilograms). 
 \item Where should I sit so that the seesaw is perfectly balanced (what is $a$)?
\end{enumerate}
\end{problem}

\begin{problem}[Center of mass]\label{center of mass of curve}%
\marginpar{\href{http://www.youtube.com/watch?v=mz-Udq5TeS4&list=PL04DF68E73B7ECD54&index=6&feature=plpp_video}{Watch a YouTube video.}}%
In problem \ref{center of mass with two points}, we focused on a system with two points $(x_1,y_1)$ and $(x_2,y_2)$ with masses $m_1$ and $m_2$. The center of mass in the $x$ direction is given by  
$$
\bar x = \frac{x_1m_1+x_2m_2}{m_1+m_2}$$
\begin{enumerate}
\item If we consider a system with 3 points, what formula gives the center of mass in the $x$ direction?
\item Consider a system with $n$ points labeled $(x_1, y_1,z_1)$, $(x_2, y_2,z_2)$, \ldots, $(x_n, y_n,z_n)$, having masses $m_1$, $m_2$, \ldots, $m_n$ respectively. Give a formula for the center of mass in the $y$ direction (the $x$ and $z$ directions are similar).
 \item Suppose now that we have a wire located along a curve $C$. The density of the wire is known to be $\delta(x,y,z)$ (which could be different at different points on the curve).  Imagine cutting the wire into a thousand or more tiny chunks.  Each chunk would be centered at some point $(x_i,y_i,z_i)$ and have length $ds_i$. Explain why the mass of each little chunk is $dm_i\approx\delta ds_i$. 
 \item Give a formula for the center of mass in the $y$ direction of the thousands of points $(x_i,y_i,z_i)$, each with mass $dm_i$. [This should almost be an exact copy of the second part.] 
 Then explain why $$\bar y = \frac{\int_C y dm}{\int_C dm}=\frac{\int_C y \delta ds}{\int_C \delta ds}.$$
\end{enumerate}
\end{problem}

For quick reference, the formulas for the centroid of a wire along $C$ are
$$
\bar x = \frac{\int_C x ds}{\int_C  ds},\quad
\bar y = \frac{\int_C y ds}{\int_C  ds},\quad 
\text{and}\quad
\bar z = \frac{\int_C z ds}{\int_C  ds}.  \quad\text{(Centroid)}
$$
If the wire has density $\delta$, then the formulas for the center of mass are 
\marginpar{The quantity $\int_C x dm$ is sometimes called the first moment of mass about the $yz$-plane (so $x=0$). Notationally, some people write $M_{yz} =\int_C x ds$. Similarly, we could write $M_{xz}=\int_C y dm$ and $M_{xy}=\int_C zdm$.  With this notation, we could write the center of mass formulas as 
$$(\bar x,\bar y,\bar z) 
= 
\left(
\frac{M_{yz}}{m},
\frac{M_{xz}}{m},
\frac{M_{xy}}{m}
\right)
.
$$ }%
$$
\bar x = \frac{\int_C x dm}{\int_C  dm},\quad
\bar y = \frac{\int_C y dm}{\int_C  dm},\quad 
\text{and}\quad
\bar z = \frac{\int_C z dm}{\int_C  dm},  \quad\text{(Center of mass)}
$$
where $dm=\delta ds$. Notice that the denominator in each case is just the mass $m=\int_C dm$.

We'll often use the notation $(\bar x, \bar y,\bar z)$ to talk about both the centroid and the center of mass. If no density is given in a problem, then $(\bar x, \bar y,\bar z)$ is the centroid. If a density is provided, then $(\bar x, \bar y,\bar z)$ refers to the center of mass. If the density is constant, it doesn't matter (the centroid and center of mass are the same, which is what the seesaw problem showed).

\begin{problem}
\instructor{I purposefully put this problem in to show students how to generalize a formula from prime numbers to any number.  I mention how I use primes (5, 7, 11, 13) when I'm looking for a pattern.  I want to help them develop this skill a little. However, if you are pressed for time, then skip 5.}%
Suppose a wire with density $\delta(x,y)=x^2+y$ lies along the curve $C$ which is the upper half of a circle around the origin with radius $7$.
\begin{enumerate}
\item Parametrize $C$ (find $\vec r(t)$ and the domain for $t$).
 \item Where is the wire heavier, at $(7,0)$ or at $(0,7)$?
 \item In problem \ref{semicircle centroid}, we showed that the centroid of the wire is $(\bar x, \bar y) = \left(0,\frac{2(7)}{\pi})\right)$.  We now seek the center of mass. Before computing, will $\bar x$ change?  Will $\bar y$ change?  How will each change? Explain.
 \item Set up the integrals needed to find the center of mass. Then use technology to compute the integrals. Give an exact answer (involving fractions), rather than a numerical approximation.
% \item Change the radius from 7 to 13, and use technology to compute the integrals again.  Generalize what you see to give a formula for the center of mass if the radius of the semicircle is $a$.
\end{enumerate}
\end{problem}

%\note{To save time, I axed this problem.  It is basically already discussed in the examples above.}
%\begin{problem}
% Suppose a wire lies along the smooth curve $C$. Your explanation in each case below should use the formula from problem \ref{center of mass of curve}. With each part, just start with the formulas for center of mass, and then simplify it to obtain the centroid formulas.
%\begin{enumerate}
% \item If the density of the wire is $\delta =1$, explain why the center of mass is the centroid. 
% \item If the density of the wire is $\delta =7$, explain why the center of mass is the centroid.
% \item If the density of the wire is constant, so $\delta =c$ for some constant $c$, explain why the center of mass is the centroid.
%\end{enumerate}
%\end{problem}

\begin{problem}\label{center of mass alternate approach}
\instructor{I don't have enough here to help them see why we use $xm$ for moments.  It really has to do with levers.  I'm not sure I want to add more.  The engineers pound this to death in all their future classes.  This problem is actually just EXTRA, but it connects the idea with average value, to center of mass, to radius of gyration.  I like to use this problem as a ``hint, this is how you do the radius of gyration problem coming up, and it's the exact same as what we did with the ant farm approach.''}%
 The quantity $M_{yz}=\int_C x dm$ is sometimes called the first moment of mass about the $yz$-plane (the plane $x=0$). It adds up the weighted distances from the plane $x=0$.  
 One way to view the center of mass is to ask yourself the following question.
\begin{quote}
 The mass $m$ of a curve $C$ is known. If you could place all the mass at one single spot, called $(\bar x,\bar y, \bar z)$, what should $\bar x$ be so that the first moment of mass about the $yz$-plane does not change. 
\end{quote}
We want the moment $\int_C \bar x dm$ (all mass at one point) and the moment $\int_C x dm$ (the mass is spread across infinitely many points) to be exactly the same.
Use this idea to solve for $\bar x$ in the equation $$\int_C \bar xdm = \int_C x dm.$$ 
Then similarly obtain $\bar y$ and $\bar z$. [Hint: the number $\bar x$ is a constant, whereas $x$ is not. Does $\int 2fdx = 2\int fdx$?]
\end{problem}


%\begin{problem}[Optional - Parallel Axis Theorem]
% This is currently a place holder for the parallel axis theorem. You'll use it extensively in future courses.  We may or may not have time for it. We'll probably come back to this when we get to double integrals.
%\end{problem}

\subsection{Inertia and Radii of Gyration}

\note{I used the variable $d$ to stand for radius of rotation.  I DO NOT use $r$ because too many students replace it with the polar coordinate $r$, especially when we get to double integrals.  I tried $d$, but now students were thinking it was a differential.  I need a different variable.  I've used $(rad)^2$ before, and it works, it's just awkward.  Jason, if you have a good idea, email me.  I would like to discuss this. We could use $(\text{dist})^2$ or $(\text{radius of rotation})^2$. Maybe the last is the best.  However, I would really like to write $I=\int ?^2 dm$ without it taking up half a board.  A variable would be good.  Capital $R$ doesn't work. }

Some of you may have already had a physics class, in which you learned that the kinetic energy of an object with mass $m$ moving at speed $v$ is $$KE = \frac{1}{2}mv^2.$$
One of the main reasons we are studying mass, center of mass, centroids, etc., is so that we can understand energy. The transfer of energy (for example from kinetic to electrical and then back from electrical to kinetic) is one of the most important ideas in modern innovations. Our goal in this unit it to help us understand rotational kinetic energy. We'll show that the kinetic energy of an object that is rotating about a line $L$, and has an angular velocity of $\omega$ radians per second about the line, is precisely $$KE = \frac{1}{2}I \omega^2,$$
where $I$ is the (second) moment of inertia. The moment of inertia can be obtained by integrating $I=\int_C (d)^2 dm$ where $d$ is the radius of rotation about $L$, i.e. the distance from a point $(x,y,z)$ to the axis of rotation $L$. If the line $L$ is one of the coordinate axes, then we obtain the key formulas 
$$
I_x = \int_C (y^2+z^2)dm,\quad
I_y = \int_C (x^2+z^2)dm,\quad
I_z = \int_C (x^2+y^2)dm
.$$
If you have never worked with kinetic energy before, you may skip the next problem and then just practice using these formulas.

\begin{problem}%
\marginpar{\href{http://www.youtube.com/watch?v=Zyqk9SWlTyQ&list=PL04DF68E73B7ECD54&index=7&feature=plpp_video}{Watch a YouTube video.}}%
\instructor{One student asked if this had to do with figure skating.  OF course.  I spun around in a circle with my arms out, and then quickly brought them in. I also like to pick up a table/desk, and show them how easy it is to rotate the object if I use an axis near the center.  Then I try to grab the edge of the desk and rotate it, it doesn't work.  The only real thing I want them to master is that the inertia gets really large (grows quadratically) with distance to an axis.  So I do something memorable to help them remember that.  We just did the normal acceleration problem in the motion unit, so I try to connect it to that.}%
 Suppose that an object, whose mass is $m$, is attached to a string (whose mass is so small we'll ignore it). The object is rotated about a point, where the angular velocity is $\omega$ radians per second. The length of the string (distance from the point to the center of rotation) is $d$.
 \begin{enumerate}
  \item  We know kinetic energy is $KE=\frac{1}{2}mv^2$. If a string is being rotated with with angular velocity $\omega$, why is the velocity of an object, that is located $d$ units away from the axis of rotation, equal to $v=d\omega$? Show that the kinetic energy of this object in rotational motion is $KE = \frac{1}{2}(d^2m)\omega^2$.  The quantity $I=d^2m$ is called the moment of inertia. This problem only applies if you have a single point.   
  \item Suppose the point $P=(x,y,z)$, which has mass $m$, is attached to a negligible mass string. The point is rotated about the $x$-axis with angular velocity $\omega$. Find the kinetic energy, using the results from the previous problem. [So what's the distance from $(x,y,z)$ to the $x$-axis.]
  \item We can think of a curve as thousands of points $(x,y,z)$, each with mass $dm=\delta ds$. As we rotate an entire curve about the $x$-axis with angular velocity $\omega$, each little piece contributes small amount of kinetic energy, which we'll call $dKE$.  Explain why $dKE = \frac{1}{2}(y^2+z^2) \omega^2 dm$.
  \item Explain why the kinetic energy of the curve (when rotated about the $x$ axis) is
$$KE= \frac{1}{2}\left(\int_C (y^2+z^2)dm\right)\omega^2=\frac{1}{2}I_x\omega^2.$$
  \item If we rotated about the $y$-axis instead, how does this formula change?
 \end{enumerate}
\end{problem}

\note{Perhaps more applications would be good here.} 

\begin{problem}
 A wire follows the helix $\vec r(t) = (3\cos t, 4t, 3\sin t)$ for $t\in[0,4\pi]$. The density is $\delta(x,y,z) = x^2+y+2z^2$. 
%\begin{enumerate}
 %\item 
 Set up formulas to compute $I_x$, $I_y$, and $I_z$. Use software to compute the integrals. In your presentation, show us the set up you used, and then just give us the numerical solutions. 
% \item Which is larger, $I_x$ or $I_z$? Can you explain why without having done any integrals?
% \item \instructor{The idea here is that since the density is greater on $z$, it makes $I_x$ larger (as $I_x$ depends on the $z$ values).  It's harder to rotate the heavier spots.}%
%If the wire had constant density and followed the elliptical helix $\vec r(t) = (2\cos t, 4t, 3\sin t)$ instead, which would be bigger, $I_x$ or $I_z$?  Feel free to use software to check your guess. 
%\end{enumerate}
\note{This problem would most likely go better if I started with density $\delta(x,y,z) = x^2+y+z^2$ and had them compute the integrals, showing that $I_x$ and $I_z$ are equal. Then change the density to $\delta(x,y,z) = x^2+y+2z^2$.  Then change the curve as in the third problem. The 3rd part tries to help them see that if the helix is elliptical, then remember that increasing $z$ doesn't affect $I_z$, rather it affects $I_y$ and $I_z$. A large $x$ value affects the inertia about the other axes.}
\end{problem}


 In problem \ref{center of mass alternate approach}, we showed how to find the center of mass by replacing the variable distance $x$ in $\int_C x dm$ with the constant distance $\bar x$, and then solving for $\bar x$ in the equation $\int_C \bar xdm = \int_C x dm$. The idea is simple; if all the mass were located at one spot, what would that spot have to be for the moment of mass to be the same.  The radii of gyration are obtained in the exact same manner.  They can be thought of as a rotational center of mass.
\begin{problem}[Radii of Gyration]%
\marginpar{\href{http://www.youtube.com/watch?v=dsVtOw09StM&list=PL04DF68E73B7ECD54&index=8&feature=plpp_video}{Watch a YouTube video.}}%
Suppose a wire lies on the curve $C$ and has density $\delta$. The inertia about a line $L$ we know is $I_x=\int_C d^2 dm$, where $d$ is the radius of rotation (distance to the line $L$).  What constant radius $R$ should we replace the variable radius $d$ with so that $\int_C d^2 dm = \int_C R^2 dm$.  Explain how to obtain the radii of gyration about the $x$ axis. 
\end{problem}

You only needed to show how to obtain the radius of gyration about the $x$ axis. All three radii of gyration are found using the formulas
$$
R_x = \sqrt{\frac{\int_C (y^2+z^2)dm}{\int_C dm}},\quad
R_y = \sqrt{\frac{\int_C (x^2+z^2)dm}{\int_C dm}},
\text{ and }
R_z = \sqrt{\frac{\int_C (x^2+y^2)dm}{\int_C dm}}.
$$

\note{Add an example in \href{http://aleph.sagemath.org/?q=e5572184-0e48-42c1-8d9b-05d2a0eee400}{Sage} so they can check their work.
%Long URL that contains the code, but doesn't work since it's too long: http://aleph.sagemath.org/?z=eJyFVU1z2jAQvetXaDLMIBORONBeMqNTe-GQi69APMIW4Kllu5KgVhn-e1eybD6atFxs6e2-fX7aFUeuyLillppxhHJR6cJYAuuIte-zR4sUO4rM1Ios55Os1sREFM8nuqjgbR0how1XhsXIiCpnTYFMqni1E4wY2mHUIRFCSKWNKqTADOfFdksUNVBQs2HxVNVKkghpVlRG7BQvSa7pJBBGKBOVUXWRpy2rSHWQQhUZL9MhmLSTXEekZWoZr0FlKI99fdh61lcc9hMO6zkscLx8xpFLRq6NuqpKcchEkgqlasU-qJHLO17Ej7s0MII78lkjOIktLotKXNLCC8S_Igw_JcxBVThJPvyOS_STPmz0vb6_vwxO6K1ltyXBUBmhN3u3bf22s1KoVHKt0xZkk7f2WcIhklvIesj20KKlC0sXMewuP_s-vK0VHpbwhol9n02cb233JL4332eRU7JGiauvfypDFk4DSuywtn4dD-vYrdHeyPLJ8E0pyBIt1cP3YP5olYvScH-soweKr495TV3kt4M6CoiDocAKRgCiVIeMzAj7ToW80LMBWG0Vz055l3M-5ebsuMM4hJhcjyjmcOTGlAJvCoPrLeYqK0W1M3snRYdIPcLTGyQA30Jrg7hSbA3pqr6cT_q8AjNxu6IwTve71u2uVLHb-2_B5DJl9DIsUa9Sss4hkONl3Al2J-60yhAP4a7ITU6ICSH8KBTfid5pPBoEyk4gJD1gsOYyIq6Pemf9h0E9J_2tvd61_a692CNUL_LGpBeJOyLXYTc7Npf33gwtT_FNm_eSFn4Yuuz3WRCxAGkdaDtH2gtkeyjuoKG3Ax4H4gTuvZVr4hOUeJa-h5LAmzjegEHHnx1keygeoNhDSbxGbtzrxmh2Grsp1MaWYvw6zrnei3x8Rg1ruOJSGLhX0qasDdzQfVfDy77IfkCWZl8i1Dyypgbh5N_20Kwu4T4cK-CnPkEXvwX7GlNcih1cQmnJN6Jkdwf1cF8g9Ca-ak5YdNw7JUQ1pvh_9C6vI84KN0iExDSOwBg6mThXPOR8IcvldE69y_6x7gMeL-g0wNNr_IogsXQ6hwh4zj8g6OHpLY70vv5FGmh83cD_Lzhvipq9RH8Ah12GVw%3D%3D

\href{http://aleph.sagemath.org/?q=236077b6-aced-48e2-8a2e-1899d403b6e8}{3D Sage}
% http://aleph.sagemath.org/?z=eJx9Vk2PqzYU3ftXWNNIMRNPJslrN5VYTTcjtRvUXZKHPOAk6GGgtpNij-a_99p8BcJrNsH3Hh_fD58LkiuuSYBuTJJlTQ3VywClvFCZNsStbRDWK7OySIY3nuhSkv3uOSkV0QHFu2eVFf5JHwOEfpEh6UAbuqObY_DSrbd0S78dg-BZryYmpJVmUocbpHmRht-eqwzpWLLizEOiaeOlzueOwH9yvVRYsB8cM3y6FonOygLL7HzRuDydsC4xv7H8yjQACpwVmp8ly3F54xLrS6ZwcpU3voYsTzjPCh53ENI-wEG_Iww_yfVVFjiKSHEVXGYJywfwHXqtrh-K1KHcb44UG_jfHqmFv93RlaZJAPsMAAFZyLiSmeA4xGl2OhFJNRRdhf1iXZRSQFdUOA4wVQFKeKFlmaVxPXHWz-B-VQPATABmCrATgG0BqQjHVwDsSExjEQFit3PcIiEXAVuRKAXQx-yjvOrY2LjKWcEfI4XNI2Q9jzQzSDOLtB7pUuMyFkypuHYhzUXzKkY4M8XV8zj7gDM9Dr1Pu0HM993Kft8FPq73aS9IPXJPO-HdpnNHLhX1j9TkvX51a9OvjV_bfm3dGp1kKbBiZ74WmUrWFy1ynImqlBqEoy9xxaQCdYCMKoScd63ZR87JHu1FReTTH21PF4eU55q5WxAsnuAut80-0hb45qQEMJA0ljAIHEj23oVeYK9jt7WV9OBsNi3bXbjVxOBP1YKCxPNM65zjj8zJGzOZ5Lw464uPRg1otcAvY-_gfGsvPCZJWYAaC90lEkDsOT9pcjhJlnxuvz7F1wG6gOsDTRWdWo234qnZOvPBj6AmFzKIlA5yHB5tcJemCJsqQzo-h0nG7u75ZMWwB7a4c0f7OtwAYzDx4A50uUKqk7hh41OAociDjNcFuYvtr_jT2K_msBoi8Od4CWAvAbwwdoG9CPzRc2obsdUdm5llq3_KVs-ymZbNzrOZn7KZCdub13lX7tGl2Arc5J-KtvOdxTxYXBwP96CfRxSPxs54eX8l3v3s8oz9GHEZOsr3-kh7mGnS74dJDzJ3IHsHciOlw9jhwAheJgc3QD7h6Ffx5QHRcFLkTmr9MHG-vNvcuW3vto07skfk3nSoCmHYMME1vD6h6KWGF1w3C6h7Hyc_Cq5U-GuAqlVYlRAq-f-yjatGkzIvZbiUPF1Sv11lloe_baZ8rRzxvB7B3vCcJefFkuIHKjegyR6-azb0xc_ayBU0chWFQjjz1HqkkqXZVYWRpWXFEpBXuFlv79laJtfkyDUQWOATYj9nHdjqezY6insc6X2cDY-P9NE6cJs5bldb6KW6lP-SCsaFquALDloIX17hlrKaq_BveeUB-g_6R3nE
}

For the remaining 2 problems, you are asked to review the key ideas in this section.  You have to obtain a parametrization of the curve, and then just set up the appropriate integrals.

\begin{problem}
Consider the curve $y=4-x^2$ for $x\in[-1,2]$, with $\delta (x,y) = y$.  Set up integral formulas which would give (1) the $x$ coordinate $\bar x$ of the centroid, (2) the $y$ coordinate $\bar y$ of the center of mass, (3) the moment of inertia $I_x$ about the $x$-axis, and (4) the radius of gyration $R_y$ about the $y$ axis. %Use software to compute the integrals. \note{I don't really need them to perform the integral, so I'm dropping this.}
\end{problem}

\begin{problem}
Consider a straight wire which lies on the line segment between $(-2,1,0)$ and $(0,-1,2)$. The density of the wire is known to be $\delta(x,y,z) = x+y+z+2$. Set up integral formulas which would give (1) the $x$ coordinate $\bar x$ of the centroid, (2) the $z$ coordinate $\bar z$ of the center of mass, (3) the moment of inertia $I_y$ about the $y$-axis, and (4) the radius of gyration $R_x$ about the $x$-axis. %Compute the integrals.
\end{problem}






















\section{The Fundamental Theorem of Line Integrals}

In this final section we'll return to the concept of work. Many vector fields are actually the derivative of a function.  When this occurs, computing work along a curve is extremely easy.  All you have to know is the endpoints of the curve, and the function $f$ whose derivative gives you the vector field. This function is called a potential for a vector field.  Once we are comfortable finding potentials, we'll show that the work done by such a vector field is the difference in the potential at the end points.  This makes finding work extremely fast.

\begin{definition}[Gradients and Potentials]
\marginpar{\href{http://www.youtube.com/watch?v=8Tk2pEIOnwg&list=PL04DF68E73B7ECD54&index=9&feature=plpp_video}{Watch a YouTube Video}.}%
 Let $\vec F$ be a vector field.  A potential for the vector field is a function $f$ whose derivative equals $\vec F$. So if $Df=\vec F$, then we say that $f$ is a potential for $\vec F$. When we want to emphasize that the derivative of $f$ is a vector field, we call $Df$ the gradient of $f$ and write $Df = \vec \nabla f$.
\marginpar{The symbol $\vec \nabla f$ is read ``the gradient of $f$'' or ``del f.''}
 If $\vec F$ has a potential, then we say that $\vec F$ is a gradient field. 
\end{definition}

We'll quickly see that if a vector field has a potential, then the work done by the vector field is the difference in the potential.  If you've ever dealt with kinetic and potential energy, then you hopefully recall that the change in kinetic energy is precisely the difference in potential energy.  This is the reason we use the word ``potential.''

\begin{problem}
\marginpar{\href{http://www.youtube.com/watch?v=8Tk2pEIOnwg&list=PL04DF68E73B7ECD54&index=9&feature=plpp_video}{Watch a YouTube Video}.}%
Let's practice finding gradients and potentials.
\begin{enumerate}
 \item  Let $f(x,y) = x^2+3xy+2y^2$. Find the gradient of $f$, i.e. find $Df(x,y)$. Then compute $D^2f(x,y)$ (you should get a square matrix). What are $f_{xy}$ and $f_{yx}$?
 \item Consider the vector field $\vec F(x,y)=(2x+y,x+4y)$. Find the derivative of $\vec F(x,y)$ (it should be a square matrix). Then find a function $f(x,y)$ whose gradient is $\vec F$ (i.e. $Df=\vec F$). What are $f_{xy}$ and $f_{yx}$?
 \item \marginpar{See problem \ref{second partials agree}.}%
Consider the vector field $\vec F(x,y)=(2x+y,3x+4y)$.  Find the derivative of $\vec F$.  Why is there no function $f(x,y)$ so that $Df(x,y)=\vec F(x,y)$? [Hint: what would $f_{xy}$ and $f_{yx}$ have to equal?] 
\end{enumerate}
\end{problem}

Based on your observations in the previous problem, we have the following key theorem.

\begin{theorem}
 Let $\vec F$ be a vector field that is everywhere continuously differentiable. Then $\vec F$ has a potential if and only if the derivative $D\vec F$ is a symmetric matrix. We say that a matrix is symmetric if interchanging the rows and columns results in the same matrix (so if you replace row 1 with column 1, and row 2 with column 2, etc., then you obtain the same matrix).  
\end{theorem}

\begin{problem}
\marginpar{If you haven't yet, please watch this \href{http://www.youtube.com/watch?v=8Tk2pEIOnwg&list=PL04DF68E73B7ECD54&index=9&feature=plpp_video}{YouTube video}.}%
For each of the following vector fields, find a potential, or explain why none exists.
\begin{enumerate}
 \item $\vec F(x,y)=(2x-y, 3x+2y)$
 \item $\vec F(x,y)=(2x+4y, 4x+3y)$
 \item $\vec F(x,y)=(2x+4xy, 2x^2+y)$
 \item $\vec F(x,y,z)=(x+2y+3z,2x+3y+4z,2x+3y+4z)$
 \item $\vec F(x,y,z)=(x+2y+3z,2x+3y+4z,3x+4y+5z)$
 \item $\vec F(x,y,z)=(x+yz,xz+z,xy+y)$
 \item $\ds \vec F(x,y) = \left(\frac{x}{1+x^2}+\arctan (y),\frac{x}{1+y^2}\right)$
\end{enumerate}
\end{problem}


If a vector field has a potential, then there is an extremely simple way to compute work. To see this, we must first review the fundamental theorem of calculus. The second half of the fundamental theorem of calculus states,
\begin{quote}
 If $f$ is continuous on $[a,b]$ and $F$ is an anti-derivative of $f$, then $F(b)-F(a) = \int_a^b f(x) dx$.
\end{quote}
If we replace $f$ with $f'$, then an anti-derivative of $f'$ is $f$, and we can write,
\begin{quote}
 If $f$ is continuously differentiable on $[a,b]$, then $f(b)-f(a)=\int_a^b f'(x) dx$.
\end{quote}
This last version is the version we now generalize.

\begin{theorem}[The Fundamental Theorem of Line Integrals]
\marginpar{\href{http://www.youtube.com/watch?v=5ZsCN6NN3yg&list=PL04DF68E73B7ECD54&index=11&feature=plpp_video}{Watch a YouTube video}.}%
 Suppose $f$ is a continuously differentiable function, defined along some open region containing the smooth curve $C$. Let $\vec r(t)$ be a parametrization of the curve $C$ for $t\in[a,b]$. Then we have
$$f(\vec r(b))-f(\vec r(a))=\int_a^b Df(\vec r(t))D\vec r(t)\ dt.$$
\end{theorem}
Notice that if $\vec F$ is a vector field, and has a potential $f$, which means $\vec F = Df$, then we could rephrase this theorem as follows. 
\begin{quote}
 Suppose $\vec F$ is a a vector field that is continuous along some open region containing the curve $C$. Suppose $\vec F$ has a potential $f$. Let $A$ and $B$ be the start and end points of the smooth curve $C$.  Then the work done by $\vec F$ along $C$ depends only on the start and end points, and is precisely
$$f(B)-f(A)=\int_C \vec F\cdot d\vec r = \int_C Mdx+Ndy.$$
 The word done by $\vec F$ is the difference in potential.
\end{quote}
If you are familiar with kinetic energy, then you should notice a key idea here.  Work is a transfer of energy. As an object falls, energy is transferred from potential energy to kinetic energy.  The total kinetic energy at the end of a fall is precisely equal to the difference between the potential energy at the top of the fall and the potential energy at the bottom of the fall (neglecting air resistance). So work (the transfer of energy) is exactly the difference in potential energy.  

\begin{problem}[Proof of Fundamental Theorem]\marginpar{The proof of the fundamental theorem of line integrals is quite short. All you need is the fundamental theorem of calculus, together with the chain rule (\ref{chain rule def}).}
 Suppose $f(x,y)$ is continuously differentiable, and suppose that $\vec r(t)$ for $t\in[a,b]$ is a parametrization of a smooth curve $C$. Prove that $f(\vec r(b))-f(\vec r(a)) = \int_a^b Df(\vec r(t))D\vec r(t)\ dt$. [Let $g(t) = f(\vec r(t))$. Why does $g(b)-g(a) = \int_a^b g'(t)dt$? Use the chain rule (matrix form) to compute $g'(t)$. Then just substitute things back in.]  
\end{problem}

\begin{problem}
\marginpar{\href{http://www.youtube.com/watch?v=5ZsCN6NN3yg&list=PL04DF68E73B7ECD54&index=11&feature=plpp_video}{Watch a YouTube video}.}%
For each vector field and curve below, find the work done by $\vec F$ along $C$. In other words, compute the integral $\int_C Mdx+Ndy$ or $\int_C Mdx+Ndy+Pdz$. [Hint: if you parametrize the curve, then you've done the problem the HARD way. You don't need any parameterizations.]
\begin{enumerate}
 \item\marginpar{See \href{http://aleph.sagemath.org/?z=eJxz06jQqdS01TDSqtCu1KnQNtGq1OQqyMkviS9LTS7JL4pPy0zNSdFw01EAKtQ11jHSBLIqdQx0LDU1tUHqNCx1K-KMdGCymgDJPhZ0}{Sage}.}%
 Let $\vec F(x,y) = (2x+y,x+4y)$ and $C$ be the parabolic path $y=9-x^2$ for $x$ from $-3$ to $2$.
 \item\marginpar{See \href{http://aleph.sagemath.org/?z=eJwVi7EKgDAMRPd-hWPSRtCIo6s_IdJBKwhFRYo0_XrT6R5372bIJFRwArbZiS3Etris2bAVBUHzxDv5L2zpfv1xhrgPO8zU6LMnRgWhdqwEhToaEF08rwALcO27aqiow4rmBwN1HWM}{Sage}.}%
 Let $\vec F(x,y,z) = (2x+yz,2z+xz,2y+xy)$ and $C$ be the straight segment from $(2,-5,0)$ to $(1,2,3)$. 
\end{enumerate}
\end{problem}

\begin{problem}\marginpar{See \href{http://aleph.sagemath.org/?z=eJytkM1uhCAUhfc-hTsueMmMYNOV23mJydQYpR1SK0TItPr0hVFbm5kumnQBl597zvngUg9APKHJAT5wxImWoU440sRY78pWNx78WTevvXKulDSxpe2Mry6q8WaonrXqWtnCAdOg4zkKGlYj8gILijAhl_hIgyorbT3Ub8oPuqmiA0BjHPjQ7nR_rXGAxz0KZjVFxiLBVdrpXsER8q-7FAQWKOkJ07UtzkfSmM4M5FSSQbVkI53bgy567Le6e2SCr0wFmykZ5Dzu5bJfUa3eiR-ov3oxwfzO6tVSsm-7B_YPhuFgLllnXiDPIrPns92SEzPnhzyJP-RlMe-qvX3CbWa-4G0-a3N4P9adzTtYmnwCd8u7lQ}{Sage}---$C_1$ and $C_2$ are in blue, and several possible $C_3$ are shown in red.}%
  Let $\vec F = (x,z,y)$. Let $C_1$ be the curve which starts at $(1,0,0)$ and follows a helical path $(\cos t, \sin t, t)$ to $(1,0,2\pi)$. Let $C_2$ be the curve which starts at $(1,0, 2\pi)$ and follows a straight line path to $(2,4,3)$. Let $C_3$ be any smooth curve that starts at $(2,4,3)$ and ends at $(0,1,2)$.
 \begin{itemize}
  \item Find the work done by $\vec F$ along each path $C_1$, $C_2$, $C_3$. \marginpar{If you are parameterizing the curves, you're doing this the really hard way. Are you using the potential of the vector field?}
  \item Find the work done by $\vec F$ along the path $C$ which follows $C_1$, then $C_2$, then $C_3$.  
  \item If $C$ is any path that can be broken up into finitely many smooth sub-paths, and $C$ starts at $(1,0,0)$ and ends at $(0,1,2)$, what is the work done by $\vec F$ along $C$?
 \end{itemize}
\end{problem}

\begin{definition}
 We say that a vector field is conservative if the integral $\int_C \vec F\cdot d\vec r$ does not depend on the path $C$. We say that a curve $C$ is piecewise smooth if it can be broken up into finitely many smooth curves.
\end{definition}

\begin{problem}
 The gravitational vector field is directly related to the radial field $\ds\vec F = \frac{\left(-x,-y,-z\right)}{(x^2+y^2+z^2)^{3/2}}$. Find a potential for $\vec F$, and then compute the work done by an object that moves from $(1,2,-2)$ to $(0,-3,4)$ along ANY path that avoids the origin. 
 %[While $\vec F$ is not continuously differentiable everywhere (it's not at the origin), we can still use the fundamental theorem of line integrals if we stay away from the origin. This shows that $\vec F$ is a conservative vector field, provided we dodge the origin.]
\end{problem}

%If a vector field is everywhere continuously differentiable, then the last problem shows that conservative vector fields and gradient fields are exactly the same.

\begin{problem}
 Suppose $\vec F$ is a gradient field.  Let $C$ be a piecewise smooth closed curve. What is $\int_C \vec F\cdot d\vec r$? Explain.
\end{problem}


\subsection{``Nice'' sets}
\label{sec:nice-sets}



Here we will discuss in (brief) technical detail what a ``nice'' set
is and what goes wrong when you have a set that is not so nice.  This
would all be explored in much more detail in an advanced calculus, real
analysis, or algebraic topology class.


\begin{figure}[t]
  \centering
 \begin{tikzpicture}[description/.style={fill=white,inner sep=2pt}]
    \matrix (m) [matrix of math nodes, row sep=8em,
    column sep=16em, text centered]
    { F=\nabla f & \node{\parbox{2.5cm}{\centering $\displaystyle\int_C\vec F\cdot
          d\vec r$ is \\ independent of path $C$}}; \\
     \parbox{1.5cm}{\centering $M_y=N_x$\\ $M_z=P_x$\\ $N_z=P_y$}  &
     \node{\parbox{2.5cm}{\centering $\displaystyle\oint_C \vec F\cdot d\vec r=0$ \\(closed path $C$)}};\\ };
    \path[>=latex,->,line width=1pt]
    (m-1-1) edge [bend left=5] (m-1-2)
    (m-1-1) edge  node[auto] {$\vec F$ has continous partials} (m-2-1)
    (m-1-2) edge [bend left=5] node[auto] {$D$ connected} (m-1-1)
    (m-1-2) edge [bend left=10] (m-2-2)
    (m-2-1) edge node[above] {$\vec F$ has continous partials} node[below] { $D$ connected and simply connected} (m-2-2)
    (m-2-2) edge [bend left=10] (m-1-2);
    
\end{tikzpicture}
  \caption{Equivalences for nice sets}
  \label{fig:equivalences}
\end{figure}


Let $\vec F=\langle M,N,P\rangle$ be continuous on an open set $D$.
In the diagram in Figure~\ref{fig:equivalences}, the arrows represent
implications (i.e., if there is an arrow from one statement pointing
to another, then the first statement being true implies that the
second statement is true).  A label on the arrow means that the
condition has to be satisfied for the implication to be hold.  The
definitions of the terms are in the book (or you can ask me).



Using terms from the book, a ``nice'' set is a set that is open,
connected, and simply connected.  Roughly, this means that $D$ does
not include its boundary and is a single region without any holes going
through it.  For such a ``nice'' set, all four statements in
Figure~\ref{fig:equivalences} are equivalent (i.e., either they are
all true or they are all false for a given vector field $\vec F$).

If the set $D$ is not connected, then path-independence does not imply
the vector field is a gradient field.  In this case, discrepancies can
occur between the two definitions for ``conservative vector field''.

There is a homework problem that shows that the bottom arrow
does not hold when $D$ is not simply connected.  When $D$ is not
connected and simply  connected, we lose the nice, easy test in the
lower left of the figure for determining exactly when a vector field
is a gradient field.





\begin{table}[h]
 \begin{center}
\begin{tabular}{|c|c|}
 \hline
 Surface Area& 
     $\sigma = \int_C d\sigma=\int_C f ds = \int_a^b f \left|\frac{d\vec r}{dt}\right|dt$\\
 \hline
 Average Value& 
     $\bar f = \frac{\int f ds}{\int ds}$\\
 \hline
 Work, Flow, Circulation &
     $W=\int_C d\text{Work} = \int_C (\vec F\cdot \vec T) ds = \int_C \vec F\cdot d\vec r = \int_C Mdx+Ndy$\\
 \hline
 Flux & 
     $\text{Flux} = \int_C d\text{Flux} = \int_C \vec F\cdot \vec n ds = \oint_C Mdy-Ndx$\\
 \hline
 Mass& 
     $m=\int_C dm = \int_C \delta ds $\\
 \hline
 Centroid& 
     $\left(\bar x,\bar y,\bar z\right) =\left(\frac{\int x ds}{\int_C ds},\frac{\int y ds}{\int_C ds},\frac{\int z ds}{\int_C ds}\right)$\\
 \hline
 Center of Mass & 
     $\left(\bar x,\bar y,\bar z\right) =\left(\frac{\int x dm}{\int_C dm},\frac{\int y dm}{\int_C dm},\frac{\int z dm}{\int_C dm}\right)$\\
 \hline
% First moment of mass & 
%     $M_{yz}=\int x dm$, $M_{xz} = \int x dm$, $M_{xy}=\int x dm$\\
% \hline
 (Second) Moment of Inertia & 
     $I_x = \int (y^2+z^2) dm$, $I_y = \int (x^2+z^2) dm$, $I_z = \int (x^2+y^2) dm$ \\
 \hline
 Radius of Gyration &
     $R = \sqrt{I/m}$\\
% \hline
% Fund. Thm. Calculus & 
%     $f(b)-f(a)=\int df = \int \frac{df}{dx}dx = \int_a^b f^\prime(x)dx$\\
 \hline
 Fund. Thm of Line Int. &
    $f(B)-f(A)=\int_C \vec \nabla f \cdot d\vec r$ 
    %\int_C df = \int_C \frac{df}{ds}ds = \int_a^b \frac{df/dt}{ds/dt}ds = 
    %\int_a^b \vec \nabla f \cdot \frac{d\vec r}{dt}dt 
    %= \int_C \vec F \cdot d \vec r$
    \\
\hline
\end{tabular}
\caption{A summary of the ideas in this unit.\label{line integral summary}}
\end{center}
\end{table}

%%% Local Variables: 
%%% mode: latex
%%% TeX-master: "215-problems"
%%% End: 
