\documentclass[11pt]{article}
\usepackage[margin=1in]{geometry}
\newcommand{\sstop}{\smallskip\begin{center}STOP\end{center}\smallskip}
\begin{document}
\begin{center}Parametric Curves\end{center}

\begin{enumerate}
\item First we'll find some parametric equations
\begin{enumerate}
\item Write $x$ and $y$ as functions of $t$ to describe the curve $y=2x-3$ in the domain $-3\leq x \leq 5$.
\item Write $x$ and $y$ as functions of $t$ to describe $y=x^2$ in the domain $-\infty\leq x\leq \infty$.
\item Write $x$ and $y$ as functions of $t$ to describe $y=\pm \sqrt{x}$ (you might find it helpful to draw the curve first).
\end{enumerate}

\sstop

\item Next we'll look at a very common parametric equation
\begin{enumerate}
\item What does the graph $\vec r(t)=(x,y)=(\cos t, \sin t)$, $0\leq t\leq 2\pi$ look like?  [Hint: try plotting a bunch of points]
\item What does the graph $\vec r(t)=(x,y)=(3\cos t, 3\sin t)$, $0\leq t\leq 2\pi$ look like?  [Hint: try plotting a bunch of points]
\item If $R$ is a constant, what does the graph $\vec r(t)=(x,y)=(R\cos t, R\sin t)$, $0\leq t\leq 2\pi$ look like?  [Hint: try plotting a bunch of points]
\item Find a relationship between $x$ and $y$ if $x=\cos t$ and $y=\sin t$ (this is an equation with just $x$ and $y$ in it).
\item Solve the equation you just found for $y$.  If you graphed this, it should give you the entire graph you got above.
\item Which is simpler?  The parametric equation, or the $y=$ equation?
\item Find a relationship between $x$ and $y$ if $x=R\cos t$ and $y=R\sin t$ (this is an equation with just $x$ and $y$ and $R$ in it).
\end{enumerate}


\sstop

\item Now we'll look at some derivatives
  \begin{enumerate}
  \item If $x=\cos t$, $y=\sin t$, how fast does $x$ change with respect to $t$?  Write this as a derivative using the definition of the derivative.
  \item If $x=\cos t$, $y=\sin t$, how fast does $y$ change with respect to $t$? Write this as a derivative using the definition of the derivative.
\sstop
  \item If $x=\cos t$, $y=\sin t$, how fast does $y$ change with respect to $x$?  In other words, what is $\lim_{\Delta x\to 0}(\Delta y)/(\Delta x)$?  [Hint: Use what we just did in the STOP and divide the top and bottom by $\Delta t$.]
\sstop

  \item What is the slope of the tangent line at the point $t=\pi/3$?
  \item Check the sign of the slope of the tangent line in each quadrant of the circle.  Is it what you expect?
  \item Above, you found an equivalent equation to this parametric function in terms of just $x$ and $y$.  Find the derivative $dy/dx$ using that equation.  Which is simpler?
  \item Once you have the first derivative $dy/dx$, you can find the concavity by taking its derivative.  What is the second derivative of $y$ with respect to $x$?
  \item Check the concavity in each quadrant of the circle with your second derivative.  Is it what you expect?
  \end{enumerate}

\sstop

\item Now we'll generalize to the next dimension
  \begin{enumerate}
  \item Now suppose we have $x=\cos t$, $y=\sin t$, and $z=t$ for $0\leq t\leq2\pi$ (so we get three coordinates associated with time $t$).  Write this fact as a function $\vec r(t)$.
  \item What does the graph of $\vec r(t)$ look like? [Hint: try plotting some points]
  \item What is the derivative of each part $\vec r^{\thinspace\prime}(t)$?  (This is also called the velocity vector).
  \end{enumerate}
\item What does the graph of $\vec r(t)=(t,t^2,\sin t)$, $-2\pi\leq t\leq 2\pi$ look like?  Describe it as best as you can.  What is the velocity vector $\vec r^{\thinspace\prime}(t)$?

\end{enumerate}
\end{document}
