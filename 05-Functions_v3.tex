\stepcounter{unitday}

\noindent 
When you have finished this unit you should be able to...
\begin{enumerate}
\item Describe uses for, and construct equations or graphs of, space curves and parametric surfaces. 
\item Find derivatives of space curves, and use this to find velocity, acceleration, and find equations of tangent lines.
\item Describe uses for, construct graphs of, and find the domain or range of functions of several variables. 
\begin{enumerate}
	\item For functions of the form $z=f(x,y)$, this includes both 3D surface plots and 2D level curve plots.
	\item For functions of the form $w=f(x,y,z)$, construct plots of level surfaces.
\end{enumerate}
\item Describe uses for, and construct graphs of, vector fields and transformations.
\item Explain how to obtain a function for a vector field, or a parametrization for a curve or surface if you are given a description of the vector field, curve, or surface (instead of a function or parametrization), 
\end{enumerate}

\clearpage


\section{Function Terminology}
In this section you will learn how to:
\begin{itemize}
\item identify the domain and range of a multivariable or vector-valued function
\end{itemize}


A function is a set of instructions involving two sets (called the domain and codomain).  A function assigns to each element of the domain $D$ exactly one element in the codomain $R$. We'll often refer to the codomain $R$ as the target space.  We'll write $$f\colon D\to R$$ when we want to remind ourselves of the domain and target space.  In this class, we will study what happens when the domain and target space are subsets of ${\mathbb{R}}^n$ (Euclidean $n$-space).  In particular, we will study functions of the form $$f\colon {\mathbb{R}}^n\to {\mathbb{R}}^m,$$ when $m$ and $n$ are 3 or less. The value of $n$ is the dimension of the input vector (or number of inputs).  The number $m$ is the dimension of the output vector (or number of outputs).  Our goal is to understand uses for each type of function, and be able to construct graphs to represent the function.\\

We will focus most of our time this semester on two- and three-dimensional problems. However, many problems in the real world require a higher number of dimensions. When you hear the word ``dimension'', it does not always represent a physical dimension, such as length, width, or height.  If a quantity depends on 30 different measurements, then the problem involves 30 dimensions.  As a quick illustration, the formula for the distance between two points depends on 6 numbers, so distance is really a 6-dimensional problem.  As another example, if a piece of equipment has a color, temperature, age, and cost, we can think of that piece of equipment being represented by a point in four-dimensional space (where the coordinate axes represent color, temperature, age, and cost).

% As we introduce each type of function, we'll introduce it in the context of a somewhat realistic setting.  
% After introducing each type of surface, we'll end this chapter with an assortment of functions to practice graphing.
\hrule

\begin{problem}\label{prob:pebble}\instructor{Recommended as a pre-class problem}
\marginpar{See 
\href{http://aleph.sagemath.org/?z=eJwrsS1LLNJQL1HXtOYqyMkv0TAz0TU00yqJM9JR0CjRMdAx0tQEAL5TCVo}{Sage}
or 
\href{http://wolfr.am/xoc07E}{Wolfram Alpha}. %http://www.wolframalpha.com/input/?i=f%28t%29%3D64-16t^2
Follow the links to Sage or Wolfram Alpha in all the problems below to see how to get the computer to graph the function.}%

A pebble falls from a 64 ft tall building.  Its height (in ft) above the ground $t$ seconds after it drops is given by the function $y=f(t)=64-16t^2$.
\begin{enumerate}
	\item What is $n$ (the number of inputs)?
	\item What is $m$ (the number of outputs)?
	\item Construct a graph of this function.
	\item How many dimensions do you need to graph this function?
\end{enumerate}

\end{problem}

The next several sections will explore different sorts of functions with a variety of $n$ and $m$ combinations. We will start with these same questions as we work to understand the functions. Later on in the semester, you can use these questions to help understand what sort of function you are studying.

\section{Parametric Curves:} %$\vec f\colon \RR \to \RR^m$}
\note{I commented out the `function' form, with the hope/intent for students to discover this themselves in the problems. This might not work so well...}
In this section you will learn how to:
\begin{itemize}
	\item express curves and surfaces with parametric equations
	\item compute rates of change along space curves
\end{itemize}

\begin{problem}\label{prob:parametric curve in plane}\instructor{Recommended as a pre-class problem}
\marginpar{
See \href{http://aleph.sagemath.org/?z=eJwrsS1LLNJQL1HX5CpILErMTS0pykyOL8jJL9GINtJKzi_WKNHUUTDWKs7MA7JidRQ0SnQMdIy0CjI1NQFWdRJT}{Sage} or \href{http://wolfr.am/wAkR8l}{Wolfram Alpha}. %http://www.wolframalpha.com/input/?i=parametric+plot+%282+cos+t,+3+sin+t%29}
}
\marginpar{%See also Chapter 3 of this problem set.
	\thomasee{There's a lot more practice of this idea in 11.1. You'll also find more practice in 13.1: 1-8.}%
	\larsonfive{See also Larson 10.2.  You can also find more practice in 12.1 and 12.3.}
	\stewarts{For review on parameterizing see 10.1 and 10.2. Find more practice in 13.4:3-6 }
	}%
A horse runs around an elliptical track. Its position at time $t$ is given by the  function $\vec r(t)=(2\cos t, 3\sin t).$ We could alternatively write this as $x=2\cos t, y=3\sin t$. 
 \begin{enumerate}
  \item What are $n$ and $m$ when we write this function in the form  $\vec r\colon {\mathbb{R}}^n\to {\mathbb{R}}^m$?
  \item Construct a graph of this function. 
  \item Next to a few points on your graph, include the time $t$ at which the horse is at this point on the graph. Include an arrow for the horse's direction.
  \item How many dimensions do you need to graph this function?
	\item If we wanted to plot time (t) on its own axis, how many dimensions would we need?
 \end{enumerate}
\end{problem}


Notice in the problem above that we placed a vector symbol above the function name, as in $\vec r\colon {\mathbb{R}}^n\to {\mathbb{R}}^m$.  When the target space (codomain) is 2-dimensional or larger, we place a vector above the function name to remind us that the output is more than just a number.

In the next problem, we keep the input as just a single number $t$, but the output is now a vector in $\mathbb{R}^3$.

\begin{problem}\label{prob:jet intro for space curves}%
\label{space curve example}%
\marginpar{
See \href{http://aleph.sagemath.org/?z=eJxL0yjRtNUw0krOLwaydBSMtIoz88CsEk2ugsSixNzUkqLM5PiCnPwSjTQdBY0SHQMdBROtgkxNTQAYOxGO}{Sage} or \href{http://www.wolframalpha.com/input/?i=parametric+plot+3D++\%282+cos+t\%2C+2+sin+t\%2C+t\%29+for+t+from+0+to+4+pi}{Wolfram Alpha}. 
	\thomasee{The text has more practice in 13.1: 9-14.}%
	\larsonfive{More practice is in Larson 12.1:9--12, 21--24, 27--32.}
	\stewarts{More practice in 13.4:7-8, 15-18}
	}%
 A jet begins spiraling upwards to gain height. The position of the jet after $t$ seconds is modeled by the equation 
$\vec r_2(t)=(2\cos t, 2\sin t, t).$ We could alternatively write this as $x=2\cos t,\, y=2\sin t,\, z=t$. 
\begin{enumerate}
 \item What are $n$ and $m$ when we write this function in the form  $\vec r\colon {\mathbb{R}}^n\to {\mathbb{R}}^m$? 
 \item Construct a graph of this function by picking several values of $t$ and plotting the points resulting from $(2\cos t, 2\sin t, t)$. 
 \item Next to a few points on your graph, include the time $t$ at which the jet is at this point on the graph. Include an arrow for the jet's direction.
 \item  How many dimensions do you need to graph this function?
\end{enumerate}
\end{problem}

\begin{problem}\label{prob:function table}
On a separate piece of paper (you'll be expanding this later) create a table with columns for:
\begin{itemize}
	\item problem number
	\item function
	\item $n$
	\item $m$
	\item number of dimensions required to graph
\end{itemize}
Go back over the previous problems in this unit and fill in the table.
\end{problem}

\section{Parametric Surfaces:} %$\vec f\colon  \RR^2 \to \RR^3$}
\note{I commented out the `function' form, with the hope/intent for students to discover this themselves in the problems. This might not work so well...}
We now increase the number of inputs from 1 to 2.  This will allow us to graph many space curves at the same time.

\begin{problem} \label{prob:parametric surface example}%
\marginpar{See \href{http://aleph.sagemath.org/?z=eJxL0yjRUUjUtNVI1ErOL9Yo0QTytIoz88CsEk2ugsSixNzUkqLM5PiCnPwSjTQdBaAOAx0FE62CTKASjUQdBSMgT1OTCwBCiRSf}{Sage} or \href{http://www.wolframalpha.com/input/?i=parametric+plot+3D++\%28a+cos+t\%2C+a+sin+t\%2C+t\%29+for+t+from+0+to+4+pi+and+a+from+2+to+4}{Wolfram Alpha}.
\larsonfive{More practice in Larson 15.5:1--6.}
}%
 The jet from problem \ref{prob:jet intro for space curves} is actually accompanied by several jets flying side by side. As all the jets fly, they leave a smoke trail behind them (it's an air show). The smoke from one jet spreads outwards to mix with the neighboring jet, so that it looks like the jets are leaving a rather wide sheet of smoke behind them as they fly. \\
The position of two of the many other jets is given by $\vec r_3(t)=(3\cos t, 3\sin t, t)$ and $\vec r_4(t)=(4\cos t,4\sin t,t)$.  A function which represents the smoke stream is $\vec r(a,t)=(a\cos t, a\sin t, t)$ for $0\leq t\leq 4\pi$ and $2\leq a\leq 4$.
 \begin{enumerate}
  \item What are $n$ (inputs) and $m$ (outputs) when we write the function $\vec r(a,t)=(a\cos t, a\sin t, t)$ in the form  $\vec r\colon {\mathbb{R}}^n\to {\mathbb{R}}^m$?
  \item Start by graphing the position of the three jets. For $t=0, \frac{\pi}{2}, \pi, \frac{3\pi}{2}$ plot the position of each jet:
	\begin{enumerate}
		\item $\vec r_2(2,t)=(2\cos t, 2\sin t, t)$
		\item $\vec r_3(3,t)=(3\cos t, 3\sin t, t)$
		\item $\vec r_4(4,t)=(4\cos t, 4\sin t, t)$
	\end{enumerate}
	\item We said in the initial problem statement that the smoke spreads and merges. So we really want to plot $\vec{r}(a,t)=(a\cos t, a\sin t, t)$ for $2\leq a \leq 4$ at each $t$ value. 
	\begin{enumerate}
		\item Describe how would this modify your graph from the previous part?
		\item Let $t=0$ and graph the curve $r(a,0)=(a,0,0)$ for $a\in[2,4]$.
		\item Repeat this for $t=\pi/2,\pi,3\pi/2$
	\end{enumerate}
  \item Describe the resulting surface.
 \end{enumerate}
\end{problem}

\textbf{Contexual Aside:} The function above is called a parametric surface.  Parametric surfaces are formed by joining together many parametric space curves. Most of 3D computer animation is done using parametric surfaces. Woody's entire body in {\it Toy Story} is a collection of parametric surfaces. Car companies create computer models of vehicles using parametric surfaces, and then use those parametric surfaces to study collisions. Often the mathematics behind these models is hidden in the software program, but parametric surfaces are at the heart of just about every 3D computer model.

\vskip0.1in
\hrule
\vskip0.1in

\begin{problem}
\marginpar{See Section~\ref{differentialtangentline} and Definition~\ref{def:velocity acceleration}.}%
\marginpar{
	\thomasee{The text has more practice in 13.1: 19-22.}
	\larsonfive{More practice in Larson 12.2:23--30.} 
	\stewarts{The text has more practice in 13.2:3-8 }
	}%
 Use the same set up as problem \ref{space curve example}, namely $$\vec r(t)=(2\cos t, 2\sin t, t).$$  You'll need a graph of this function to complete this problem.
 \begin{enumerate}
  \item Find the first and second derivative of $\vec r(t)$. \\
	Note that $\vec{v}(t) = \vec{r}'(t)$ and $\vec{a}(t)=\vec{r}''(t)$.
  \item Compute the velocity and acceleration vectors at $t=\pi/2$. Place these vectors on your graph with their tails at the point corresponding to $t=\pi/2$.
  \item Give an equation of the tangent line to this curve at $t=\pi/2$.
 \end{enumerate}
\end{problem}


\begin{problem}%
\marginpar{
See \href{http://aleph.sagemath.org/?z=eJwNxksKgCAUBdB5q3Dmh2eQfWZuJXmIgpAodtt_Dg6crKC92g1IXIfdLoPb6aXz4JowSgz9aVCZhAJZt540aRL89hQRBqM0L_lDq7NR6h-iAhhm}{Sage} or \href{http://wolfr.am/ynm3kD}{Wolfram Alpha}. %http://www.wolframalpha.com/input/?i=parametric+plot+%283t,+64-16t^2%29
	\thomasee{The text has more practice in 13.1: 1-8.}
	\larsonfive{See also Larson 12.3.}
	\stewarts{The text has more practice in 13.4:3-6}
	}%
Consider the pebble from problem \ref{prob:pebble}. The pebble's height was given by $y=64-16t^2$.  The pebble also has some horizontal velocity (it's moving at 3 ft/s to the right).  If we let the origin be the base of the 64 ft building, then the position of the pebble at time $t$ is given by $\vec r(t) = (3t, 64-16t^2)$.
 \begin{enumerate}
  \item What are $n$ and $m$ when we write this function in the form  $\vec r\colon {\mathbb{R}}^n\to {\mathbb{R}}^m$?
  \item At what time does the pebble hit the ground (the height reaches zero)?
	\item Construct a graph of the pebble's path from when it leaves the top of the building until it hits the ground.
  \item\marginpar{See Section~\ref{differentialtangentline} and Definition~\ref{def:velocity acceleration}.}%
 Find the pebble's velocity and acceleration vectors at $t=1$? Draw these vectors on your graph with their base at the pebble's position at $t=1$. 
  \item At what speed is the pebble moving when it hits the ground?
 \end{enumerate}
\end{problem}

\begin{problem}
Go back to the table you started in problem \ref{prob:function table}.
\begin{enumerate}
	\item Add in the new problems you've completed.
	\item What do you observe about the domain ($n$), co-domain ($m$) and the dimensions required for plotting?
\end{enumerate}
\end{problem}

In all the problems above, you should have noticed that in order to draw a function (provided you include arrows for direction, or use an animation to represent ``time''), you can determine how many dimensions you need to graph a function by just summing the dimensions of the domain and codomain. This is true in general.

\begin{challenge}[More Parametric Surfaces]\label{second parametric surface example}%
\marginpar{See \href{http://aleph.sagemath.org/?z=eJxL0yjVUSjTtNUo1UrOL9Yo09RRKNUqzsyDsOKMNLkKEosSc1NLijKT4wty8ks00nQUQHoMdBRMgEo0ynQMdIy0CjI1NQFPyxVa}{Sage} or \href{http://wolfr.am/A90cfW}{Wolfram Alpha}.% http://www.wolframalpha.com/input/?i=parametric+plot+3d+%28u+cos+v,+u+sin+v,+u^2%29
}%
 Consider the parametric surface $\vec r(u,v)=(u\cos v, u\sin v, u^2)$ for $0\leq u\leq 3$ and $0\leq v\leq 2 \pi$.
 Construct a graph of this function. To do so, let $u$ equal a constant (such as 1, 2, 3) and then graph the resulting space curve.  Then let $v$ equal a constant (such as 0, $\pi/2$, etc.) and graph the resulting space curve until you can visualize the surface. [Hint: Think satellite dish.] 
\end{challenge}
%
%\newpage
%
%\uday
%\normalsize

\section{Functions of Several Variables:} %$f\colon \RR^n \to \RR$}
Section Objectives:
\begin{itemize}
\item identify the domain and range of a multivariable or vector-valued function
\item express curves and surfaces with parametric equations
\end{itemize}

In this section we'll focus on functions of the form $f\colon \mathbb{R}^2\to\mathbb{R}^1$ and $f\colon \mathbb{R}^3\to\mathbb{R}^1$; we'll keep the output as a real number. In the next problem, you should notice that the input is a vector $(x,y)$ and the output is a number $z$. There are two ways to graph functions of this type.  The next two problems show you how. 

\subsection{3-D Surface Plots}

\begin{problem}\label{prob:3dsurface plot}\instructor{Recommended as a pre-class problem}
\marginpar{See \href{http://aleph.sagemath.org/?z=eJxL06jQqdS0tdStiDPSrYwz4irIyS8xTtFI01EAyuga6xhrAlmVEJYmADAVC84}{Sage} or 
\href{http://wolfr.am/wny0IF}{Wolfram Alpha}.%http://www.wolframalpha.com/input/?i=plot+3d+9-x^2-y^2
}%
\larsonfive{\marginpar{See Larson 13.1:33--40.}}%
 A computer chip has been disconnected from electricity and sitting in cold storage for quite some time.  The chip is connected to power, and a few moments later the temperature (in Celsius) at various points $(x,y)$ on the chip is measured. From these measurements, statistics is used to create a temperature function $T=z=f(x,y)$ to model the temperature at any point on the chip. Suppose that this chip's temperature function is given by the equation $T=z=f(x,y)=9-x^2-y^2$. Eventually we'll be creating a 3D model of this function in this problem, so you'll need to place all your graphs on the same $x,y,z$ axes. However, to begin with you may find it easier to start with several 2-D plots.
\begin{enumerate}
	\item What is the temperature at $(0,0)$, $(1,2)$, and $(-4,3)$? \marginpar{\thomasee{See 14.1: 1-4.} \stewarts{See 14.1: 2, 7}}
	\item Let $y=0$, Substitute this value into the temperature function $T=z=f(x,y)=9-x^2-y^2$. Plot the resulting function in the $xz$-plane. [Hint: Treat $z$ as ``y''. It should be a (upside-down) parabola]
%construct a graph of the temperature $z=f(x,0) = 9-x^2-0^2$, or just $z=9-x^2$. In the $xz$ plane (where $y=0$) draw this upside down parabola.
	\item Now let $x=0$. Substitute this value into the temperature function and simplify. Plot the result. This plot should appear in the $yz$ plane and be another parabola.
	\item Now let $z=0$. Draw the resulting curve in the $xy$ plane.\\
	If you drew the above curves in 2-D it is time to combine them. You should end up with 3 lines or a basic wire frame. To make a more complete wireframe we need to add more curves.
	\item State the simplified equation, then plot and label the curves for:
	\begin{enumerate}
		\item $x=-2$
		\item $x=-1$
		\item $x=1$
		\item $x=2$
	\end{enumerate}
	Note: You could have done this with $y$ also. 
	%\item Once you've drawn a curve in each of the three coordinate planes, it's useful to pick an input variable (either $x$ or $y$) and let it equal various constants. So now let $x=1$ and draw the resulting parabola in the plane $x=1$.  Then repeat this for $x=2$.
	\item Describe the shape. Add any extra features to your graph to convey the 3D image you are constructing. \marginpar{\thomasee{See 14.1: 37-48.}\stewarts{See 14.1: 23-31}}
\end{enumerate}
\end{problem}

\subsection{2-D Contour Plots}

\begin{problem}\label{prob:intro to contour plots}\instructor{Recommended as a pre-class problem}
\marginpar{See \href{http://aleph.sagemath.org/?z=eJydj81qwzAQhO9-iiWXWCCHJqbQHHTtExR6KI1RnBVWWWuNVibR21fOH_Ta2-7M7PCtqy86K7NvLoddkw-7aiJO7al2GorTtLpVZcq3SW1k4HOtqiGNVK8-EZY0pDPDyTuHEUMCSZlQgB30HBLP8RoSEIbMM_Q2gCCWdcQllAb8EwSekucgm5Wq7nq36IXoCfTg0Y9LMV8veqtf9Zvefy8qcTzaaD7ijLof7WTWP5jWT_7_FpM9IsmtFpwnMu-WBPVV73wgH_DuWpmwT1205RuzVdUvSwN0NA}{Sage} or 
\href{http://wolfr.am/wny0IF}{Wolfram Alpha}.%http://www.wolframalpha.com/input/?i=plot+3d+9-x^2-y^2
}%
\larsonfive{\marginpar{See Larson 13.1:45--56.}}%
We'll be using the same function $T=z=f(x,y)=9-x^2-y^2$ as the previous problem.  However, this time we'll construct a graph of the function by only studying places where the temperature is constant.  We'll do this by creating a graph in 2D of the surface (similar to a topographical map). \marginpar{
	\thomasee{See 14.1: 13-16 and 31-36.}
	\stewarts{See 14.1: Examples 9-13, for more practice problems 14.1:43-50}
	}%

 \begin{enumerate}
  \item How can we find all the points which have zero temperature? [Hint: Think about what a temperature of zero means in the function]
	\item Use your process to find the curve corresponding to a temperature of zero. Plot this curve in the $xy$-plane. Be sure to label it with $T=0$\\
	This curve is called a level curve. As long as you stay on this curve, your temperature will remain level, it will not increase nor decrease. 
	%\item Which points in the plane have zero temperature? Just let $z=0$ in $z=9-x^2-y^2$. Plot the corresponding points in the $xy$-plane, and write $z=0$ next to this curve. This curve is called a level curve. As long as you stay on this curve, your temperature will remain level, it will not increase nor decrease. 
  \item Which points in the plane have temperature $z=5$?  Add this level curve to your 2D plot and write $z=5$ next to it.
  \item Repeat the above for $z=8$, $z=9$, and $z=1$. \marginpar{\thomasee{See 14.1: 37-48.}}
	\item What's wrong with letting $z=10$? 
  \item Using your 2D plot, construct a 3D image of the function by lifting each level curve to its corresponding height.
 \end{enumerate}
\end{problem}

\begin{definition}
 A level curve of a function $z=f(x,y)$ is a curve in the $xy$-plane found by setting the output $z$ equal to a constant. Symbolically, a level curve of $f(x,y)$ is the curve $c=f(x,y)$ for some constant $c$.  A 2D plot consisting of several level curves is called a contour plot of $z=f(x,y)$.
\end{definition}

\vspace{0.1in}
\hrule
\vspace{0.1in}

\subsection{Parametric Surfaces Continued}

\begin{problem}\instructor{Recommended as a pre-class problem}

\marginpar{See \href{http://aleph.sagemath.org/?z=eJxL06jQqdS0rdCtjDPiKs7IL9coyMkvMU7RSNNRAErpGusYawJZlRCWpiZETXJ-Xkl-aVE8SC12lToKybmJBbbqWakl6kB2fk5-UVJikW1IUWmqTk5iUmpOMZitqQkAhh0mmg}{Sage} or 
\href{http://wolfr.am/wBOk1b}{Wolfram Alpha}.}%http://www.wolframalpha.com/input/?i=plot+3d+x-y^2
\marginpar{
		\thomasee{More practice is in 14.1: 37-48.}%
		\larsonfive{See Larson 13.1:45--56.}
		\stewarts{More practice is in 14.1:51-52, 55-58}}%
 Consider the function $f(x,y)=x-y^2$.
\begin{enumerate}
 \item Use the same process as problem \ref{prob:3dsurface plot} to construct a 3D surface plot of $f$. [So just graph in 3D the curves given by $x=0$ and $y=0$ and then try setting $x$ or $y$ equal to some other constants, like $x=1$, $x=2$, $y=1$, $y=2$, etc.]
 \item Use the same process as problem \ref{prob:intro to contour plots} to construct a contour plot of $f$. [So just graph in 2D the curves given by setting $z$ equal to a few constants, like $z=0$, $z=1$, $z=-4$, etc.]
 \item% 
\marginpar{\thomasee{See 14.1: 49-52.}}%
Which level curve passes through the point $(2,2)$?  Draw this level curve on your contour plot.
\end{enumerate}
\end{problem}

Notice that when we graphed the previous two functions (of the form $z=f(x,y)$) we could either construct a 3D surface plot, or we could reduce the dimension by 1 and construct a 2D contour plot by letting the output $z$ equal various constants. \\

The next function is of the form $w=f(x,y,z)$, so it has 3 inputs and 1 output.  We could write $f\colon \mathbb{R}^3\to\mathbb{R}^1$. We would need 4 dimensions to graph this function, but graphing in 4D is not an easy task.  Instead, we'll reduce the dimension and create plots in 3D to describe the level surfaces of the function.

\vskip0.1in
\hrule
\vskip0.1in

\begin{problem}%
\marginpar{See \href{http://aleph.sagemath.org/?z=eJwrSyzSUK_QqdSpUtfkCtEAszRtDQ0MdCvijHQrgbgqzogrM7cgJzM5syS-ICe_xDhFA67Q1tJSRwHIUdA11lEw1gSyK8FsMLMKytRUAADtWRrw}{Sage}.  \\
Wolfram Alpha currently does not support drawing level surfaces.  You could also use Mathematica or \href{http://demonstrations.wolfram.com/LevelSurfacesAndQuadraticSurfaces/}{Wolfram Demonstrations}.}
\marginpar{
		\thomasee{You can access more problems on drawing level surfaces in 12.6:1-44 or 14.1:53-60.}%
		\larsonfive{See Larson 11.6 and 13.1:69--74, as well as 13.1, Example 6.}
		\stewarts{For more practice see 14.1:65-68 }}%
Suppose that an explosion occurs at the origin $(0,0,0)$. Heat from the explosion starts to radiate outwards.  Now suppose that a few moments after the explosion, the temperature at any point in space is given by $w=T(x,y,z)=100-x^2-y^2-z^2.$ 
\begin{enumerate}
 \item Which points in space have a temperature of 99? Use algebra to simplify this to $x^2+y^2+z^2=1$. [Hint: What does the 99 replace in your function?]
%Used to include: To answer this, replace $T(x,y,z)$ by $99$ to get $99=100-x^2-y^2-z^2$.
 \item Draw the resulting object/function.
 \item Which points in space have a temperature of 96? of 84? Draw the surfaces. 
 \item What is your temperature at $(3,0,-4)$? Draw the level surface that passes through $(3,0,-4)$.
\item The 4 surfaces you drew above are called level surfaces. If you walk along a level surface, what happens to your temperature?
 \item As you move outwards, away from the origin, what happens to your temperature?
\end{enumerate}
\end{problem}

\instructor{Talk about graphing functions with 4 or more variables.  Show the class \href{http://www.osirix-viewer.com/}{OsiriX} as an example of graphing a 4d function (where opacity is the density of material.  Also, practice sliding a plane through a 3d object to get an idea of what the contour plots are telling us. One example could be a play-do object, and actually taking slices of it. A fun play-do activity from `The Art of Mathematics' is included in the GitHub project to explore this idea.}
 

\begin{problem}
\marginpar{See \href{http://aleph.sagemath.org/?z=eJwrSyzSUK_QqdSpUtfkStMAszRtK-KMtKvijLgycwtyMpMzS-ILcvJLjFM04ApsTXQUgGwFXWMdBWNNILsSzAYzq6BMTQUAEvYY4A}{Sage}.}%
\marginpar{ \larsonfive{See Larson 11.6:7--16.}}%
Consider the function $w=f(x,y,z)=x^2+z^2$. This function has an input $y$, but notice that changing the input $y$ does not change the output of the function. This means that when $y=0$ you can draw one curve.  When $y=1$, you should draw the same curve.  When $y=2$, again you draw the same curve, etc.
 \begin{enumerate}
  \item Draw a graph of the level surface $w=4$.
  \item Graph the surface $9=x^2+z^2$ (so the level surface $w=9$).
  \item Graph the surface $16=x^2+z^2$.
 \end{enumerate}
This kind of graph is called a cylinder, and is important in manufacturing where you extrude an object through a hole.
\end{problem}

Most of our examples of function of the form $w=f(x,y,z)$ can be drawn by using our knowledge about conic sections.\marginpar{\valpo{If you feel a little lost and skipped the review unit on conic sections, now would be a good time to go back and go through it.}\instructor{If the conic section was skipped, now's a good time to review}} We can graph ellipses and hyperbolas if there are only two variables. So the key idea is to set one of the variables equal to a constant and then graph the resulting curve.  Repeat this with a few variables and a few constants, and you'll know what the surface is. Sometimes when you set a specific variable equal to a constant, you'll get an ellipse. If this occurs, try setting that variable equal to other constants, as ellipses are generally the easiest curves to draw.

\begin{problem}
Go back to the table you started in problem \ref{prob:function table}.
\begin{enumerate}
	\item Add in the new problems you've completed.
	\item What do you observe about the domain ($n$), co-domain ($m$) and the dimensions required for plotting?
\end{enumerate}
\end{problem}

\begin{challenge}[More Surfaces]
\marginpar{See \href{http://aleph.sagemath.org/?z=eJwrSyzSUK_QqdSpUtfkStMAszRtK-KMdCvjjLSr4oy4MnMLcjKTM0viC3LyS4xTNOCKbA11FIBsBV1jHQVjTSC7EswGM6ugTE0FAIXAGhM}{Sage}.
	\thomasee{Remember you can find more practice in 12.6:1-44 or 14.1: 53-64.}%
	\larsonfive{See Larson 11.6 and 13.1:69--74, as well as 13.1, Example 6.}
	\stewarts{Remember you can find more practice in 14.1:65-68}
	}%
Consider the function $w=f(x,y,z)=x^2-y^2+z^2$.\marginparbmw{We'll have a few people present this problem.}
 \begin{enumerate}
  \item Draw a graph of the level surface $w=1$. [You need to graph $1=x^2-y^2+z^2$. Let $x=0$ and draw the resulting curve. Then let $y=0$ and draw the resulting curve. Let either $x$ or $y$ equal some more constants (whichever gave you an ellipse), and then draw the resulting ellipses.]  
  \item Graph the level surface $w=4$. [Divide both sides by $4$ (to get a 1 on the left) and the repeat the previous part.]
  \item Graph the level surface $w=-1$. [Try dividing both sides by a number to get a 1 on the left. If $y=0$ doesn't help, try $y=1$ or $y=2$.]
  \item Graph the level surface that passes through the point $(3,5,4)$. [Hint: what is $f(3,5,4)$?]
 \end{enumerate}
\end{challenge}

%\uday
%\normalsize

\subsection{Vector Fields and Transformations:} % $\vec f\colon \RR^n\to\RR^n$}

We will finish this section by considering vector fields and transformations. 
\begin{itemize}
 \item $\vec T(u,v)=(x,y)$ or $f\colon \mathbb{R}^2\to\mathbb{R}^2$ (2D transformation)
 \item $\vec T(u,v,w)=(x,y,z)$ or $f\colon \mathbb{R}^3\to\mathbb{R}^3$ (3D transformation)
 \item $\vec F(x,y)=(M,N)$ or $f\colon \mathbb{R}^2\to\mathbb{R}^2$ (vector fields in the plane)
 \item $\vec F(x,y,z)=(M,N,P)$ or $f\colon \mathbb{R}^3\to\mathbb{R}^3$ (vector fields in space)
 \end{itemize}
Notice that in all cases, the dimension of the input and output are the same. The difference between vector fields and transformations has to do with the application.

\subsubsection{Transformations}
In this section you will...
\begin{itemize}
\item identify the domain and range of multivariable or vector-valued functions
\end{itemize}

 We've already seen examples of transformations with polar, cylindrical, and spherical coordinates.

\begin{review*}
Consider the coordinate transformation $$\vec T(r,\theta) = (r\cos\theta,r\sin\theta).$$ 
\begin{enumerate}
\item Let $r=3$ and then graph $\vec T(3,\theta)=(3\cos\theta,3\sin\theta)$ for $\theta\in[0,2\pi]$.
\item Let $\theta=\frac{\pi}{4}$ and then, on the same axes as above, add the graph of 
$\vec T\left(r,\frac{\pi}{4}\right)=\left(r\frac{\sqrt 2}{2},r \frac{\sqrt 2}{2}\right)$ for $r\in[0,5]$.
\end{enumerate}
\end{review*}

\begin{problem}\label{graphing spherical coordinates}%
\marginpar{Recall that $\phi$ (``phi'') is the angle down from the $z$ axis, $\theta$ (``theta'') is the angle counterclockwise from the $x$-axis in the $xy$-plane, and $\rho$ (``rho'') is the distance from the origin. Review problem \ref{derive spherical coordinates} if you need a refresher.}%
\larsonfive{\marginpar{See Larson 11.7:89--94, 111--114.}}%
Consider the spherical coordinates transformation as defined in 
	\stewarts{$$\vec T(\rho,\phi,\theta)=(\rho\sin\phi\cos\theta,\rho\sin\phi\sin\theta,\rho\cos\phi),$$}

	\bmw{$$\vec T(\rho,\phi,\theta)=(\rho\sin\phi\cos\theta,\rho\sin\phi\sin\theta,\rho\cos\phi),$$ }

	\larsonfive{$$\vec T(\rho,\theta,\phi)=(\rho\sin\phi\cos\theta,\rho\sin\phi\sin\theta,\rho\cos\phi),$$ }

which could also be written as 
\begin{align*}
	x&=\rho\sin\phi\cos\theta\\
	y&=\rho\sin\phi\sin\theta\\
	z&=\rho\cos\phi 
\end{align*}

  Graphing this transformation requires $3+3=6$ dimensions. In this problem we'll construct parts of this graph by graphing various surfaces. We did something similar for the polar coordinate transformation in problem \ref{polar coordinate transformation graph}. 
\begin{enumerate}
 \item% 
   \marginpar{See \href{http://aleph.sagemath.org/?z=eJxVjsEKhDAMRO9-xeCpKTmId__C-1JEaEBtaPP_bLMirLe8ecOQNdRcGJqFYXm3RFjgWWxyhR5T3EoLt2K8hB__wosuaNAql2FcfWiZCdJGxkODpprO3apsHz2KhUcw7jnGxJijiie_zzp3oi_jZjWn}{Sage} or 
\href{http://www.wolframalpha.com/input/?i=parametric+plot+3d+\%282+sin+phi+cos+theta\%2C+2+sin+phi+sin+theta\%2C+2+cos+phi\%29}{Wolfram Alpha}.}%
Let $\rho=2$ and graph the resulting surface.  What do you get if $\rho = 3$?
 \item % 
\marginpar{See 
\href{http://aleph.sagemath.org/?z=eJxVjrEKwzAMRPd8xZHJMioNTdf8RfZiQsCCNha2_p9WyeJuuveOQ2uouTA0C8PybomwwFlscoQfpriVFi7F-BN-9MKLLmjQKodhXD0uKvcnQdrI6MCgqabPblW2l76Lhc4xrl3GxHhEFSfnn7eZMRN9AfBbOD8}{Sage} or 
\href{http://www.wolframalpha.com/input/?i=parametric+plot+3d+\%28rho+sin+\%28pi\%2F4\%29+cos+theta\%2C+rho+sin+\%28pi\%2F4\%29+sin+theta\%2C+rho+cos+\%28pi\%2F4\%29\%29+}{Wolfram Alpha}.
}%
Let $\phi=\pi/4$ and graph the resulting surface.  What do you get if $\phi=\pi/2$?
 \item Let $\theta=\pi/4$ and graph the resulting surface.  What do you get if $\theta=\pi/2$?
\end{enumerate}

\end{problem}

\subsubsection{Vector Fields}
In this section you will...
\begin{itemize}
\item identify the domain and range of multivariable or vector-valued functions
\item define and sketch two- or three- dimensional vector fields
\end{itemize}

We now explore a vector field example.

\begin{problem}%
\marginpar{See 
\href{http://aleph.sagemath.org/?z=eJxz06jQqdRUsFXQMNKq0K7UqdA20qrU5CrIyS-JL0tNLskvik_LTM1J0XDTUQAq1TU00DE00ASyK2FsTQCKaxIN}{Sage} or
\href{http://wolfr.am/y4gIgX}{Wolfram Alpha}. % http://www.wolframalpha.com/input/?i=plot+a+vector+field&f1={2x%2By%2Cx%2B2y}&x=6&y=7&f=VectorPlot.vectorfunction_{2x%2By%2Cx%2B2y}&f2=x&f=VectorPlot.vectorplotvariable1_x&f3=-10&f=VectorPlot.vectorplotlowerrange1_-10&f4=10&f=VectorPlot.vectorplotupperrange1_10&f5=y&f=VectorPlot.vectorplotvariable2_y&f6=-10&f=VectorPlot.vectorplotlowerrange2_-10&f7=10&f=VectorPlot.vectorplotupperrange2_10
The computer will shrink the largest vector down in size so it does not overlap any of the others, and then reduce the size of all the vectors accordingly. 

\thomasee{See 16.2: 39-44 for more practice.}%
\larsonfive{See Larson 15.1:1--19.}
\stewarts{See 16.1: 1-14 for more practice.}
}%
 Consider the vector field $\vec F(x,y)=(2x+y,x+2y)$.  In this problem, you will construct a graph of this vector field by hand.
\begin{enumerate}
 \item Compute $\vec F(1,0)$. Then draw the vector $F(1,0)$ with its base at $(1,0)$.
 \item Compute $\vec F(1,1)$. Then draw the vector $F(1,1)$ with its base at $(1,1)$.
 \item Repeat the above process for the points $(0,1)$, $(-1,1)$, $(-1,0)$, $(-1,-1)$, $(0,-1),$ and $(1,-1)$. Remember, at each point draw a vector.  
\end{enumerate}
\end{problem}

\begin{problem}[The Spin field]
\marginpar{Use the links above to see the computer plot this.  

\thomasee{See 16.2: 39-44 for more practice.}%
\larsonfive{See Larson 15.1:1--19.}
\stewarts{See 16.1: 1-14 for more practice.}}%
 Consider the vector field $\vec F(x,y)=(-y,x)$. Construct a graph of this vector field. Remember, the key to plotting a vector field is ``at the point $(x,y)$, draw the vector $\vec F(x,y)$ with its base at $(x,y)$.''  Plot at least 8 vectors (a few in each quadrant), so we can see what this field is doing.
\end{problem}

\instructor{Talk about vector field visualization, mention line integral convolutions and streamline plots.  Show Sage or mathematica doing these sorts of plots.}

\instructor{This is a great time to schedule trips to the Viz-Box or other 3-D visualization opportunities}

\href{http://aleph.sagemath.org/?z=eJxz06jQqdSp0lSwVdAA0joVmlwFOfkl8WWpySX5RfFpmak5KcYpGm46CkCFusY6xpo6IIUQlkYVhKEJAOGFExs}{Sage} can also help us visualize 3d vector fields, like $\vec F(x,y,z)=(y,z,x)$. \note{use 3d glasses and Sage/JMol's ability to render for 3d glasses to \emph{really} see this vector field!}

\section{Constructing Functions}
We now know how to draw a vector field provided someone tells us the equation. How do we obtain an equation of a vector field? The following problem will help you develop the gravitational vector field.

\begin{problem}[Radial fields]
\marginpar{Use \href{http://aleph.sagemath.org/?z=eJxz06jQqdSp0lSwVdAA0joVmlwFOfkl8WWpySX5RfFpmak5KcYpGm46CkCFusY6xpo6IIUQlkYVhKEJAOGFExs}{Sage} to plot your vector fields.  

\thomasee{See 16.2: 39-44 for more practice.}%
\larsonfive{See Larson 15.1:1--19.}}%
Do the following:
\begin{enumerate}
	\item Let $P=(x,y,z)$ be a point in space. At that point, what is the $x$, $y$, and $z$ distance to the origin?
	\item At the point $P$, let $\vec F(x,y,z)$ be the vector which points from $P$ to the origin.  Give a formula for $\vec F(x,y,z)$ [Hint: An object following this vector would travel from the point to the origin].
	\item Give an equation of the vector field where at each point $P$ in the plane, the vector $\vec F_2(P)$ is a unit vector that points towards the origin.
 \item Give an equation of the vector field where at each point $P$ in the plane, the vector $\vec F_3(P)$ is a vector of length 7 that points towards the origin.
% \item Give an equation of the vector field where at each point $P$ in the plane, the vector $\vec G(P)$ points towards the origin, and has a magnitude equal to $1/d^2$ where $d$ is the distance to the origin.
\end{enumerate}
\end{problem}

If someone gives us parametric equations for a curve in the plane, we know how to draw the curve.  How do we obtain parametric equations of a given curve? In problem \ref{prob:parametric curve in plane}, we were given the parametric equation for the path of a horse, namely $x=2\cos t, y=3 \sin t$ or $\vec r(t)=(2\cos t,3\sin t)$. From those equations, we drew the path of the horse, and could have written a Cartesian equation for the path. How do we work this in reverse, namely if we had only been given the ellipse $\ds\frac{x^2}{4}+\frac{y^2}{9}=1$, could we have obtained parametric equations $\vec r(t)=(x(t),y(t))$ for the curve?

%\begin{problem}\label{parameterizing plane curves}
%\marginpar{Use \href{http://aleph.sagemath.org/?z=eJwrsS1LLNJQL1HX5CpILErMTS0pykyOL8jJL9GINtJKzi_WKNHUUTDWKs7MA7JidRQ0SnQMdIy0CjI1NQFWdRJT}{Sage} or \href{http://wolfr.am/wAkR8l}{Wolfram Alpha} %http://www.wolframalpha.com/input/?i=parametric+plot+%282+cos+t,+3+sin+t%29
%to visualize your parameterizations.
%}%
 %Give a parametrization of the top half of the ellipse $\ds\frac{x^2}{a^2}+\frac{y^2}{b^2}=1$, so $y\geq 0$.
 %You can write your parametrization in the vector form $\vec r(t)=(?,?)$, or in the parametric form $x=?,\ y=?$. 
 %Include bounds for $t$. 
% [Hint: Review \ref{parametric curve in plane example}.]
%\end{problem}


\begin{problem}
 Give a parametrization of the straight line from $(a,0)$ to $(0,b)$. 
 You can write your parametrization in the vector form $\vec r(t)=(?,?)$, or in the parametric form $x=?,\ y=?$. 
 Remember to include bounds for $t$. \newline [Hint: Review \ref{prob:line equation practice}.]
\end{problem}

%\hrule


\begin{problem}
 Give a parametrization of the parabola $y=x^2$ from $(-1,1)$ to $(2,4)$. 
 Remember the bounds for $t$.
\end{problem}


\begin{problem}
 Give a parametrization of the function $y=f(x)$ for $x\in[a,b]$.
 You can write your parametrization in the vector form $\vec r(t)=(?,?)$, or in the parametric form $x=?,\ y=?$. 
 Include bounds for $t$.
\end{problem}

%\newpage
%
%\uday
%\normalsize
%\begin{itemize}
%\item identify the domain and range of a multivariable or vector-valued function
%\item express curves and surfaces with parametric equations
%\end{itemize}

\vskip0.2in
%%%%%%%%%%%%%%%%%%%%%%%%%%%%%%%%%%%%%%%
%%%%%%%%%%%%%%%%%%%%%%%%%%%%%%%%%%%%%%%
%%%%%%%%%%%%%%%%%%%%%%%%%%%%%%%%%%%%%%%
\note{Idea.  They did the problem below already when drawing spherical coordinates.  They already practiced removing a variable.  I somehow need to make that connect to this part. How to do it, I'm not sure. Think about it, and try something different next time.}
%%%%%%%%%%%%%%%%%%%%%%%%%%%%%%%%%%%%%%%
%%%%%%%%%%%%%%%%%%%%%%%%%%%%%%%%%%%%%%%
%%%%%%%%%%%%%%%%%%%%%%%%%%%%%%%%%%%%%%%
If someone gives us parametric equations for a surface, we can draw the surface. This is what we did in problems \ref{prob:parametric surface example} and \ref{second parametric surface example}. 
How do we work backwards and obtain parametric equations for a given surface?
This requires that we write an equation for $x$, $y$, and $z$ in terms of two input variables (see \ref{prob:parametric surface example} and \ref{second parametric surface example} for examples). 
In vector form, we need a function $\vec r\colon \mathbb{R}^2\to\mathbb{R}^3$. 
We can often use a coordinate transformation $\vec T\colon \mathbb{R}^3\to\mathbb{R}^3$ to obtain a parametrization of a surface.\\
 
The next three problems show how to do this.   

\vskip0.1in
\hrule
\vskip0.1in

\begin{problem}\label{3d parametric plot}
\marginpar{Use \href{http://aleph.sagemath.org/?z=eJwL0ajQqdSp0lSwVYCyuAqKMvNKFJRCNKpsK7QrNRUyi5V0FGA8roLEosTc1JKizOT4gpz8Eg2YhA5Iv66xjjGIVQlhaQIALhka5w}{Sage} or
\href{http://wolfr.am/zk2KTu}{Wolfram Alpha} %http://www.wolframalpha.com/input/?i=parametric+plot+3d+%28x%2Cy%2Cx%2By%29
to plot your parametrization.  

	\thomasee{See 16.5: 1-16 for more practice.}%
	\larsonfive{See Larson 15.5:21--30 and 15.5, Example 3.}
	\stewarts{See section 16.6 for more examples}
	}%
 Consider the surface $z=9-x^2-y^2$ plotted in problem \ref{prob:3dsurface plot}.
\begin{enumerate}
 \item 
Using the rectangular coordinate transformation $\vec T(x,y,z)=(x,y,z)$, give a parametrization $\vec r\colon \mathbb{R}^2\to\mathbb{R}^3$ of the surface. 

This is the same as saying $$x=x, y=y, z=?.$$
[Hint: Use the surface equation to eliminate the input variable $z$ in $T$.]

 \item What bounds must you place on $x$ and $y$ to obtain the portion of the surface above the plane $z=0$?
 \item If $z=f(x,y)$ is any surface, give a parametrization of the surface (i.e., $x=?, y=?, z=?$ or $\vec r (?,?)=(?,?,?)$.)
\end{enumerate}

\end{problem}
\begin{problem}%
\marginpar{Use \href{http://aleph.sagemath.org}{Sage} or \href{http://wolframalpha.com}{Wolfram Alpha} to plot your parametrization with your bounds (see \ref{3d parametric plot} for examples).  

	\thomasee{See 16.5: 1-16 for more practice.}%
	\larsonfive{See Larson 15.5:1--10}
	}%
 Again consider the surface $z=9-x^2-y^2$.
\begin{enumerate}
 \item
Using cylindrical coordinates, $\vec T(r,\theta,z) = (r\cos \theta, r\sin\theta, z)$, obtain a parametrization $\vec r(r,\theta)=(?,?,?)$ of the surface using the input variables $r$ and $\theta$. In other words, if we let $x=r\cos \theta, y=r\sin\theta, z=z$, write $z=9-x^2-y^2$ in terms of $r$ and $\theta$.
 \item What bounds must you place on $r$ and $\theta$ to obtain the portion of the surface above the plane $z=0$?
\end{enumerate}

\end{problem}


\begin{problem}%
\marginpar{
	We did very similar things in problem \ref{graphing spherical coordinates}.

	\thomasee{See 16.5: 1-16 for more practice.}
	\larsonfive{See Larson 15.5:1--10}
	}%

Recall the spherical coordinate transformation as given in 
\stewarts{$$\vec T(\rho,\phi,\theta) = (\rho\sin\phi\cos \theta, \rho\sin\phi\sin \theta,\rho \cos \phi).$$}
\bmw{$$\vec T(\rho,\phi,\theta) = (\rho\sin\phi\cos \theta, \rho\sin\phi\sin \theta,\rho \cos \phi).$$ }%
\larsonfive{$$\vec T(\rho,\theta, \phi) = (\rho\sin\phi\cos \theta, \rho\sin\phi\sin \theta,\rho \cos \phi).$$}%
This is a function of the form $\vec T\colon \mathbb{R}^3\to\mathbb{R}^3$.  If we hold one of the three inputs constant, then we have a function of the form $\vec r\colon \mathbb{R}^2\to\mathbb{R}^3$, which is a parametric surface.
\begin{enumerate}
 \item \marginpar{Use \href{http://aleph.sagemath.org}{Sage} or \href{http://www.wolframalpha.com/}{Wolfram Alpha} to plot each parametrization (see \ref{3d parametric plot} for examples).}%
Give a parametrization of the sphere of radius 2, using $\phi$ and $\theta$ as your input variables. 
 \item What bounds should you place on $\phi$ and $\theta$ if you want to hit each point on the sphere exactly once? [Hint: There are two possible answers here]
 \item What bounds should you place on $\phi$ and $\theta$ if you only want the portion of the sphere above the plane $z=1$? [Hint: There are also two possible answers here]
\end{enumerate}
\end{problem}

\vskip0.1in
\hrule
\vskip0.1in

Sometimes you'll have to invent your own coordinate system when constructing parametric equations for a surface.  If you notice that there are lots of circles parallel to one of the coordinate planes, try using a modified version of cylindrical coordinates. Instead of circles in the $xy$ plane ($x=r\cos\theta,y=r\sin\theta,z=z$), maybe you need circles in the $yz$-plane ($x=x,y=r\sin\theta,z=r\sin\theta$) or the $xz$ plane.  Just look for lots of circles, and then construct your parametrization accordingly.
\begin{challenge}
\marginpar{\larsonfive{See Larson 15.5:21--30.}}%
Find parametric equations for the surface $x^2+z^2=9$. [Hint: Read the paragraph above.]  
\begin{enumerate}
 \item\marginpar{Use \href{http://aleph.sagemath.org}{Sage} or \href{http://www.wolframalpha.com/}{Wolfram Alpha} to plot each parametrization  (see \ref{3d parametric plot} for examples).}%
 What bounds should you use to obtain the portion of the surface between $y=-2$ and $y=3$?
 \item What bounds should you use to obtain the portion of the surface above $z=0$?
 \item What bounds should you use to obtain the portion of the surface with $x\geq 0$ and $y\in[2,5]$?
\end{enumerate}
\end{challenge}

\begin{challenge}
 Construct a graph of the surface $z = x^2-y^2$.  Do so in 2 ways.  (1) Construct a 3D surface plot.  (2) Construct a contour plot, which is a graph with several level curves. Which level curve passes through the point $(3,4)$? 
 Use Wolfram Alpha to know if you're right.  Just type ``plot z=x\^2-y\^2.''
\end{challenge}

\begin{challenge}
Construct a plot of the vector field $$\vec F(x,y) = (x+y, -x+1)$$ by graphing the field at many integer points around the origin (I generally like to get the 8 integer points around the origin, and then a few more).
Then explain how to modify your graph to obtain a plot of the vector field $$\hat F(x,y) = \frac{(x+y, -x+1)}{\sqrt{(x+y)^2+(1-x)^2}}.$$ 
\end{challenge}

\newpage

\section{Summary of Functions}\label{sec:functionlist}

In this unit we've covered a lot of different function types. Be sure you can recognize each one both from its functional form and its name. The following list provides a summary of the unit.
\begin{itemize}
	\item $y=f(x)$ or $f\colon \mathbb{R}\to\mathbb{R}$ (functions of a single variable)
	\item $\vec r(t)=(x,y)$ or $f\colon \mathbb{R}\to\mathbb{R}^2$ (parametric curves)
	\item $\vec r(t)=(x,y,z)$ or $f\colon \mathbb{R}\to\mathbb{R}^3$ (space curves)
	\item $\vec r(u,v)=(x,y,z)$ or $f\colon \mathbb{R}^2\to\mathbb{R}^3$ (parametric surfaces)
	\item $z=f(x,y)$ or $f\colon \mathbb{R}^2\to\mathbb{R}$ (functions of two variables)
	\item $z=f(x,y,z)$ or $f\colon \mathbb{R}^3\to\mathbb{R}$ (functions of three variables)
	\item $\vec T(u,v)=(x,y)$ or $f\colon \mathbb{R}^2\to\mathbb{R}^2$ (2D transformation)
	\item $\vec T(u,v,w)=(x,y,z)$ or $f\colon \mathbb{R}^3\to\mathbb{R}^3$ (3D transformation)
	\item $\vec F(x,y)=(M,N)$ or $f\colon \mathbb{R}^2\to\mathbb{R}^2$ (vector fields in the plane)
	\item $\vec F(x,y,z)=(M,N,P)$ or $f\colon \mathbb{R}^3\to\mathbb{R}^3$ (vector fields in space) 
\end{itemize}

%\clearpage

%%% Local Variables: 
%%% mode: latex
%%% TeX-master: "215-problems"
%%% End: 
