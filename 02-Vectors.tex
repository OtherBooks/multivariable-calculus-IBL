
\noindent 
This unit covers the following ideas. In preparation for the quiz and exam, make sure you have a lesson plan containing examples that explain and illustrate the following concepts.  
\begin{enumerate}
\item Define, draw, and explain what a vector is in 2 and 3 dimensions.
\item Add, subtract, multiply (scalar, dot product, cross product) vectors. Be able to illustrate each operation geometrically.
\item Use vector products to find angles, length, area, projections, and work.
\item Use vectors to give equations of lines and planes, and be able to draw lines and planes in 3D.

\end{enumerate}
You'll have a chance to teach your examples to your peers prior to the exam.

\section{Vectors and Lines}
Learning to work with vectors will be key tool we need for our work in high dimensions.  Let's start with some problems related to finding distance in 3D, drawing in 3D, and then we'll be ready to work with vectors.


\begin{problem}
To find the distance between two points $(x_1,y_1)$ and $(x_2,y_2)$ in the plane, we create a triangle connecting the two points.  The base of the triangle has length $\Delta x=(x_2-x_1)$ and the vertical side has length $\Delta y=(y_2-y_1)$. The Pythagorean theorem gives us the distance between the two points as $\sqrt{\Delta x^2+\Delta y^2}=\sqrt{(x_2-x_1)^2+(y_2-y_1)^2}$.

Show that the distance between two points $(x_1,y_1,z_1)$ and $(x_2,y_2,z_2)$ in 3-dimensions is $\sqrt{\Delta x^2+\Delta y^2+\Delta z^2}=\sqrt{(x_2-x_1)^2+(y_2-y_1)^2+(z_2-z_1)^2}$.
\end{problem}

\begin{problem}\marginpar{\bmw{See 12.1:41-58.}}
Find the distance between the two points $P=(2,3,-4)$ and $Q=(0,-1,1)$. Then find an equation of the sphere passing though point $Q$ whose center is at $P$.
\end{problem}


\begin{problem}\marginpar{\bmw{See 12.1:1-40.}}
For each of the following, construct a rough sketch of the set of points in space (3D) satisfying:
\begin{enumerate}
\item $2\leq z\leq 5$
\item $x=2,y=3$
\item $x^2+y^2+z^2=25$
\end{enumerate}
\end{problem}

\begin{definition}
A vector is a magnitude in a certain direction.  
If $P$ and $Q$ are points, then the vector $\vec{PQ}$ is the directed line segment from $P$ to $Q$. This definition holds in 1D, 2D, 3D, and beyond.  
If $V=(v_1,v_2,v_3)$ is a point in space, then to talk about the vector $\vec v$ from the origin $O$ to $V$ we'll use any of the following notations:
$$\vec v = \vec{OV}=\left<v_1,v_2,v_3\right> 
= v_1{\bf i}+v_2{\bf j}+v_3{\bf k} 
= (v_1,v_2,v_3) 
= \begin{pmatrix}v_1\\ v_2\\ v_3\end{pmatrix}
.$$
The entries of the vector are called the $x$, $y$, and $z$ components of the vector. 
\end{definition}

Note that $(v_1,v_2,v_3)$ could refer to either the point $V$ or the vector $\vec v$. The context of the problem we are working on will help us know if we are dealing with a point or a vector.

\begin{definition}
Let $\mathbb{R}$ represent the set real numbers. Real numbers are actually 1D vectors.  
Let $\mathbb{R}^2$ represent the set of vectors $(x_1,x_2)$ in the plane.
Let $\mathbb{R}^3$ represent the set of vectors $(x_1,x_2,x_3)$ in space. There's no reason to stop at 3, so let $\mathbb{R}^n$ represent the set of vectors $(x_1,x_2,\ldots,x_n)$ in $n$ dimensions.
\end{definition}
In first semester calculus and before, most of our work dealt with problem in $\mathbb{R}$ and $\mathbb{R}^2$. Most of our work now will involve problems in $\mathbb{R}^2$ and $\mathbb{R}^3$. We've got to learn to visualize in $\mathbb{R}^3$.

\begin{definition}
The magnitude, or length, or norm of a vector $\vec v = \left<v_1,v_2,v_3\right>$ is $|\vec v| = \sqrt{v_1^2+v_2^2+v_3^2}$. It is just the distance from the point $(v_1,v_2,v_3)$ to the origin. A unit vector is a vector whose length is one unit. 

The standard unit vectors are ${\bf i}=\left<1,0,0\right>$, ${\bf j}=\left<0,1,0\right>$, ${\bf k}=\left<0,0,1\right>$. 
\end{definition}
Note that in 1D, the length of the vector $\left<-2\right>$ is simply $|-2|=\sqrt{(-2)^2}=2$, the distance to 0. Our use of the absolute value symbols is appropriate, as it generalizes the concept of absolute value (distance to zero) to all dimensions.

\begin{definition}
Suppose $\vec x=\left<x_1,x_2,x_3\right>$ and $\vec y=\left<y_1,y_2,y_3\right>$ are two vectors in 3D, and $c$ is a real number. We define vector addition and scalar multiplication as follows:
\begin{itemize}
\item Vector addition: $\vec x+\vec y = (x_1+y_1,x_2+y_2,x_3+y_3)$ (add component-wise).
\item Scalar multiplication: $c\vec x = (cx_1,cx_2,cx_3)$.
\end{itemize}
\end{definition}


\begin{problem}\marginpar{\bmw{See 12.2:23-24.}}
Consider the vectors $\vec u=(1,2)$ and $\vec v=\left<3,1\right>$.  Draw $\vec u$, $\vec v$, $\vec u+\vec v$, and $\vec u-\vec v$ with their tail placed at the origin.  Then draw $\vec v$ with its tail at the head of $\vec u$. 
\end{problem}

\begin{problem}\marginpar{\bmw{See 11.1: 3,4.}}
Consider the vector $\vec v=(2,-1)$.  Draw $\vec v$, $-\vec v$, and $3\vec v$. 
Suppose a donkey travels along the path given by $(x,y)=\vec v t = (2t,-t)$, where $t$ represents time. Draw the path followed by the donkey. Where is the donkey at time $t=0,1,2$? Put markers on your graph to show the donkey's location. Then determine how fast the donkey is travelling.
\end{problem}

In the previous problem you encountered $(x,y)=(2t,-t)$.  This is an example of a function where the input is $t$ and the output is a vector $(x,y)$.  For each input $t$, you get a single vector output $(x,y)$. Such a function is called a vector-valued function. Often, we'll use the variable $\vec r$ to represent the vector $(x,y)$, or $(x,y,z)$ in 3D.  So we could rewrite the position of the donkey as $\vec r(t)=(2,-1)t$. We use $\vec r$ instead of $r$ to remind us that the output is a vector.

\begin{problem}\label{horse line}\marginpar{\bmw{See 12.2: 1.}}
Suppose a horse races down a path given by the vector valued function $\vec r(t) = (1,2)t+(3,4)$.  
Draw the path followed by the horse. Where is the horse at time $t=0,1,2$? Put markers on your graph to show the horse's location. Give a vector that tells the horse's direction. Then determine how fast the horse is travelling.
\end{problem}

\begin{problem}\marginpar{\bmw{See 12.2: 9,17,25,33 and surrounding.}}
Consider the two points $P=(1,2,3)$ and $Q=(2,-1,0)$. Write the vector $\vec {PQ}$ in component form $(a,b,c)$. Find the length of vector $\vec {PQ}$. Then find a unit vector in the same direction as $\vec{PQ}$. Finally, find a vector of length 7 units that points in the same direction as $\vec{PQ}$. 
\end{problem}


\begin{problem}\marginpar{\bmw{See 12.5: 1-12.}}
A raccoon is sitting at point $P=(0,2,3)$.  It starts to climb in the direction $\vec v=\left<1,-1,2\right>$.  
Write a vector equation $\vec r$ for the line that passes through the point $P$ and is parallel to $\vec v$. [Hint, study problem \ref{horse line}, and base your work off of what you saw there.]

Then generalize your work to give an equation of the line that passes through $P=(x_1,y_1,z_1)$ and is parallel to $\vec v=(v_1,v_2,v_3)$. 
\end{problem}

Make sure you ask me in class to show you how to connect the equation developed above to what you have been doing since middle school. If you can remember $y=mx+b$, then you can quickly remember the equation of a line.  If I don't show you in class, make sure you ask me (or feel free to come by early and ask before class).

\begin{problem}\marginpar{\bmw{See 12.5: 13-20.}}\label{first line between two points}%
Let $P=(3,1)$ and $Q=(-1,4)$.  
\begin{itemize}
\item Write a vector equation $\vec r$ for the line that passes through $P$ and $Q$, with $\vec r(0)=P$ and $\vec r(1)=Q$.
\item Write a vector equation $\vec r$ for the line that passes through $P$ and $Q$, with $\vec r(0)=P$ but whose speed is twice the speed of the first line.
\item Write a vector equation $\vec r$ for the line that passes through $P$ and $Q$, with $\vec r(0)=P$ but whose speed is one unit per second.
\end{itemize}
\end{problem}

\section{The Dot Product}
Now that we've learned how to add and subtract vectors, stretch them by scalars, and use them to find lines, it's time to introduce a way of multiplying vectors called the dot product.  We'll use the dot product to help us find find angles. First, we need to recall the law of cosines.
\begin{theorem*}[The Law of Cosines]
Consider a triangle with side lengths $a$, $b$, and $c$. Let $\theta$ be the angle between the sides of length $a$ and $b$. Then the law of cosines states that 
$$c^2=a^2+b^2-2ab\cos\theta.$$
If $\theta=90^\circ$, then $\cos\theta=0$ and this reduces to the Pythagorean theorem.
\end{theorem*}

\begin{definition}[The Dot Product]
If $\vec u = (u_1,u_2,u_3)$ and $\vec v= (v_1,v_2,v_3)$ are vectors in $\mathbb{R}^3$, then we define the dot product of these two vectors to be 
$$\vec u\cdot \vec v = u_1 v_1+ u_2 v_2+ u_3 v_3.$$
A similar definition holds for vectors in $\mathbb{R}^n$, where
$\vec u\cdot \vec v = u_1 v_1+ u_2 v_2+\cdots+ u_n v_n.$
You just multiply corresponding components together and then add. It is the same process used in matrix multiplication.
\end{definition}

\begin{problem}\label{dot prep} \marginpar{\bmw{Page 693 has the solution if you are struggling.}}
If $\vec u = (u_1,u_2,u_3)$ and $\vec v= (v_1,v_2,v_3)$ are vectors in $\mathbb{R}^3$ (which is often written $\vec u,\vec v\in\mathbb{R}^3$), then show that 
$$|\vec u-\vec v|^2 = |\vec u|^2-2\vec u\cdot \vec v +|\vec v|^2.$$
\end{problem}

\begin{problem} \marginpar{\bmw{See 12.3: 9-12.}}
Sketch in $\mathbb{R}^2$ the vectors $\left<1,2\right>$ and $\left<3,5\right>$.  Use the law of cosines to find the angle between the vectors.
\end{problem}


\begin{problem}  \marginpar{\bmw{See page 693.}}
Let $\vec u,\vec v\in\mathbb{R}^3$. Let $\theta$ be the angle between $\vec u$ and $\vec v$. 
\begin{enumerate}
\item Use the law of cosines to explain why $|\vec u-\vec v|^2=|\vec u|^2+|\vec v|^2-2|\vec u||\vec v|\cos\theta$.
\item Use the above together with problem \ref{dot prep} to explain why $$\vec u\cdot \vec v=|\vec u||\vec v|\cos\theta.$$
\end{enumerate}
\end{problem}

\begin{problem}  \marginpar{\bmw{See 12.3: 9-12.}}
Sketch in $\mathbb{R}^3$ the vectors $\left<1,2,3\right>$ and $\left<-2,1,0\right>$.  Use the law of cosines to find the angle between the vectors. Then use the formula $\vec u\cdot \vec v=|\vec u||\vec v|\cos\theta$ to find the angle between them. Which was easier?
\end{problem}

\begin{definition}
We say that the vectors $\vec u$ and $\vec v$ are orthogonal if $\vec u\cdot \vec v=0$. 
\end{definition}

\begin{problem}
Find two vectors orthogonal to $(1,2)$.  Then find 4 vectors orthogonal to $(3,2,1)$.  
\end{problem}

\begin{problem}
Mark each statement true or false. Explain. You can assume that $\vec u,\vec v,\vec w\in\mathbb{R}^3$ and that $c\in\mathbb{R}$.
\begin{enumerate}
\item $\vec u\cdot \vec v=\vec v\cdot \vec u$. 
\item $\vec u\cdot (\vec v\cdot \vec w)=(\vec u\cdot\vec v)\cdot\vec w$. 
\item $c(\vec u\cdot \vec v)=(c\vec u)\cdot \vec v=\vec u\cdot (c\vec v)$. 
\item $\vec u+(\vec v\cdot \vec w)=(\vec u+\vec v)\cdot(\vec u+\vec w)$. 
\item $\vec u\cdot (\vec v+ \vec w)=(\vec u+\vec v)\cdot(\vec u+\vec w)$. 
\item $\vec u\cdot \vec u= |\vec u|^2$. 
\end{enumerate}
\end{problem}

\begin{problem} \marginpar{\bmw{See page 694.}}
Show that if two nonzero vectors $\vec u$ and $\vec v$ are orthogonal, then the angle between them is 90$^\circ$. Then show that if the angle between them is 90$^\circ$, then the vectors are orthogonal.
\end{problem}
The dot product provides a really easy way to find when two vectors meet at a right angle. The dot product is precisely zero when this happens.

\subsection{Projections and Work}
Suppose a heavy box needs to be lowered down a ramp.  
The box exerts a downward force of 200 Newtons, which we will write in vector notation as $\vec F=\left<0,-200\right>$. 
The ramp was placed so that the box needs to be moved right 6 m, and down 3 m, so we need to get from the origin $(0,0)$ to the point $(6,-3)$.  This displacement can be written as $\vec d=\left<6,-3\right>$. The force $F$ acts straight down, which means the ramp takes some of the force. Our goal is to find out how much of the 200N the ramp takes, and how much force must be applied to prevent the box from sliding down the ramp (neglecting friction). We are going to break the force $\vec F$ into two components, one component in the direction of $\vec d$, and another component orthogonal to $\vec d$. 

\begin{problem}\label{force intro}
Read the preceding paragraph. We want to write $\vec F$ as the sum of two vectors $\vec F = \vec w+\vec n$, where $\vec w$ is parallel to $\vec d$ and $\vec n$ is orthogonal to $\vec d$.
Since $\vec w$ is parallel to $\vec d$, we can write $\vec w=c\vec d$ for some unknown scalar $c$.
This means that $\vec F=c\vec d+\vec n$.  Use the fact that $\vec n$ is orthogonal to $\vec d$ to solve for the unknown scalar $c$.  [Hint:  dot each side of $\vec F=c\vec d+\vec n$ with $\vec d$. This should turn the vectors into numbers, so you can use division.]
\end{problem}
The solution to the previous problem gives us the definition of a projection.
\begin{definition}
The projection of $\vec F$ onto $\vec d$, written $\proj_{\vec d}\vec F$, is defined as $$\proj_{\vec d}\vec F = \left(\frac{\vec F\cdot \vec d}{\vec d\cdot \vec d}\right)\vec d.$$
\end{definition}

\begin{problem} \marginpar{\bmw{See 12.3:1-8 (part d).}}
Let $\vec u=(-1,2)$ and $\vec v=(3,4)$. Draw $\vec u,$ $\vec v$, and $\proj_{\vec v}\vec u$. Then draw a line segment from the head of $\vec u$ to the head of the projection.

Now let $\vec u=(-2,0)$ and keep $\vec v=(3,4)$. Draw $\vec u,$ $\vec v$, and $\proj_{\vec v}\vec u$. Then draw a line segment from the head of $\vec u$ to the head of the projection.

\end{problem}

One final application of projections pertains to the concept of work. Work is the transfer of energy. If a force $F$ acts through a displacement $d$, then the most basic definition of work is $W=Fd$, the product of the force and the displacement.  This basic definition has a few assumptions.
\begin{itemize}
\item The force $F$ must act in the same direction as the displacement.
\item The force $F$ must be constant throughout the entire displacement.
\item The displacement must be in a straight line.
\end{itemize}
Before the semester ends, we will be able to remove all 3 of these assumptions.  The next problem will show you how dot products help us remove the first assumption.

Recall the set up to problem \ref{force intro}.  We want to lower a box down a ramp (which we will assume is frictionless). Gravity exerts a force of $\vec F=\left<0,-200\right>$ N. If we apply no other forces to this system, then gravity will do work on the box through a displacement of $\left<6,-3\right>$ m. The work done by gravity will transfer the potential energy of the box into kinetic energy (remember that work is a transfer of energy).  How much energy is transferred?

\begin{problem} \marginpar{\bmw{See 12.3: 24, 41-44.}}
Find the amount of work done by the force $\vec F=\left<0,-200\right>$ through the displacement $\vec d=\left<6,-3\right>$. Find this by doing the following:
\begin{enumerate}
\item Find the projection of $\vec F$ onto $\vec d$. This tells you how much force acts in the direction of the displacement. Find the magnitude of this projection.
\item Since work equals $W=Fd$, multiply your answer above by $|\vec {d}|$.  
\end{enumerate}
Now compute $\vec F\cdot \vec d$. You have just shown that $W=\vec F\cdot \vec d$ when $\vec F$ and $\vec d$ are not in the same direction.
\end{problem}

\section{The Cross Product and Planes}

The dot product gave us a way of multiplying two vectors together, but the result was a number, not a vectors. We now define the cross product, which will allow us to multiply two vectors together to give us another vector.  We were able to define the dot product in all dimensions.  The cross product is only defined in $\mathbb{R}^3$. 

\begin{definition}[The Cross Product]
\marginpar{This definition is not really a definition.  It is actually a theorem.  If you use the formula given as the definition, then you would need to prove the three facts. We have the tools to give a complete proof of (1) and (3), but we would need a course in linear algebra to prove (2). It shouldn't be too much of a surprise that the cross product is related to area, since it is defined in terms of determinants}
The cross product of two vectors $\vec u = \left<u_1,u_2,u_3\right>$ and $\vec v = \left<v_1,v_2,v_3\right>$ is a new vector $\vec u\times \vec v$. This new vector is (1) orthogonal to both $\vec u$ and $\vec v$, (2) has a length equal to the area of the parallelogram whose sides are these two vectors, and (3) points in the direction your thumb points as you curl the base of your right hand from $\vec u$ to $\vec v$. The formula for the cross product is $$\vec u\times \vec v = \left<u_2v_3-u_3v_2,-(u_1v_3-u_3v_1),u_1v_2-u_2v_1\right> = \det\begin{bmatrix}\vec i & \vec j&\vec k\\ u_1&u_2&u_3\\ v_1&v_2&v_3\\\end{bmatrix}.$$
\end{definition}

\begin{problem}  \marginpar{\bmw{See 12.4: 1-8.}}
Let $\vec u=(1,-2,3)$ and $\vec v=(2,0,-1)$.  
\begin{itemize}
\item Compute $\vec u\times \vec v$.
\item Compute $\vec u \cdot (\vec u\times \vec v)$ and $\vec v \cdot (\vec u\times \vec v)$. Why did you get the answer you got?
\item Compute $\vec v\times \vec u$.  How is this related to $\vec u\times \vec v$?
\item Compute $|\vec u \times \vec v|$.  Compute the area of the parallelogram formed by $\vec u$ and $\vec v$ using trigonometry and $|\vec u|$, $|\vec v|$, and the angle $\theta$ between the two vectors, and compare your answer with $|\vec u \times \vec v|$.
\item Compute $\vec u \times (2\vec u)$.  Why did you get the answer you got?
\end{itemize}
\end{problem}

\begin{problem}  \marginpar{\bmw{See 12.4: 15-18.}}
Let $P=(2,0,0)$, $Q=(0,3,0)$, and $R=(0,0,4)$. Find a vector that orthogonal to both $\vec {PQ}$ and $\vec {PR}$. Then find the area of the triangle $PQR$. Construct a 3D graph of this triangle.
\end{problem}

\begin{problem} \marginpar{\bmw{See 12.3: 9-14.}}
Consider the vectors ${\ii}=(1,0,0)$, ${2\jj}=(0,2,0)$, and ${3\kk}=(0,0,3)$.
\begin{itemize}
\item Compute $\ii\times {2\jj}$ and ${2\jj}\times {\ii}$.
\item Compute ${\ii}\times {3\kk}$ and ${3\kk}\times {\ii}$.
\item Compute ${2\jj}\times {3\kk}$ and ${3\kk}\times {2\jj}$.
\end{itemize}
Give a geometric reason as to why some vectors above have a plus sign, and some have a minus sign.
\end{problem}

We will now combine the dot product with the cross product to develop an equation of a plane in 3D. 
Before doing so, let's look at what information we need to obtain a line in 2D, and a plane in 3D.  
To obtain a line in 2D, one way is to have 2 points. 
The next problem introduces the new idea by showing you how to find an equation of a line in 2D. 

\begin{problem}\label{plane equation normal point}
Suppose the point $P=(1,2)$ lies on line $L$. Suppose that the angle between the line and the vector $\vec n=\left<3,4\right>$ is 90$^\circ$ (whenever this happens we say the vector $\vec n$ is normal to the line). Let $Q=(x,y)$ be another point on the line $L$. Use the fact that $\vec n$ is orthogonal to $\vec {PQ}$ to obtain an equation of the line $L$. 
\end{problem}

\note{Add a problem about the equation of a plane containing two vectors.}

\begin{problem}\label{plane equation three points}\marginpar{\bmw{See page 709.}\larsonfive{See Larson 11.5.}}%
Let $P=(a,b,c)$ be a point on a plane in 3D. Let $\vec n=(A,B,C)$ be a normal vector to the plane (so the angle between the plane and and $\vec n$ is 90$^\circ$).  Let $Q=(x,y,z)$ be another point on the plane.  Show that an equation of the plane through point $P$ with normal vector $\vec n$ is $$A(x-a)+B(y-b)+C(z-c)=0.$$
\end{problem}

\begin{problem}  \marginpar{\bmw{See 12.5: 21-28.}\larsonfive{See Larson 11.5:47--58 for more practice.}}%
Consider the three points $P=(1,0,0), Q=(2,0,-1), R=(0,1,3)$. Find an equation of the plane which passes through these three points.  [Hint: first find a normal vector to the plane.]
\end{problem}

\begin{problem}  \marginpar{\bmw{See 12.5: 57-60.}\larsonfive{See Larson 11.5:91--92 for more practice.}}%
Consider the two planes $x+2y+3z=4$ and $2x-y+z=0$.  These planes meet in a line.  Find a vector that is parallel to this line.  Then find a vector equation of the line.
\end{problem}

\begin{problem}\label{plane equation 2 lines}\marginpar{\larsonfive{See Larson 11.5:47--58 for more practice.}}%
  Find the equation of the plane containing the lines $\vec r_1(t)=(1,3,0)t+(1,0,2)$ and $\vec r_2(t)=(2,0,-1)t+(2,3,2)$.
\end{problem}

\section{Projections, lines, and planes}

Using the dot product, cross product, and projections, derive the
following formulas which find distances between points, lines, and
planes. 

\begin{problem}
  Show that the distance from a point $Q$ to a plane (with normal vector {$\vec n$} and a point $P$) is given by $|\proj_{\vec n}\overrightarrow {PQ}|$. Draw a diagram illustrating your reasoning.
\end{problem}


\begin{problem}
  Show that the distance from a point $Q$ to a line (with direction vector $\vec v$ passing through $P$) is $|\overrightarrow{PQ}-\proj_{\vec v}\overrightarrow {PQ}|$. Draw a diagram illustrating your reasoning.
\end{problem}

\begin{problem}
  Show that the distance from a line (with direction vector $\vec v_1$ passing through $P_1$) to a line (with direction vector $\vec v_2$ passing through $P_2$) is $|\proj_{\vec v_1\times\vec v_2}\overrightarrow {P_1P_2}|$. Draw a diagram illustrating your reasoning.
\end{problem}

\note{Here are two more problems about dot and cross product, but they aren't central to the course.  Maybe these could be good problems for advanced students.  I saw these in Larson:

  \begin{problem} If the statement is true, explain why.  If it is false, give a counterexample.
    \begin{enumerate}
    \item If $\vec u \neq 0$ and $\vec u \times \vec v = \vec u \times \vec w$, then is $\vec v = \vec w$ always?
    \item If $\vec u \neq 0$, $\vec u\cdot \vec v = \vec u \cdot \vec w$,  and $\vec u \times \vec v = \vec u \times \vec w$, then is $\vec v = \vec w$ always?
    \end{enumerate}
  \end{problem}
}
%%% Local Variables: 
%%% mode: latex
%%% TeX-master: "215-problems"
%%% End: 
