\noindent After completing this unit you will be able to...
\begin{enumerate}
\item Model motion in the plane using parametric equations. In particular, describe conic sections using parametric equations. 
\item Find derivatives and tangent lines for parametric equations. Explain how to find velocity, speed, and acceleration from parametric equations.
\item Use integrals to find the lengths of parametric curves.
\end{enumerate}

\note{I split this out as I found many students really could use a quick review of parametric equations, even though `officially' at Valpo they learned it in Calc I or Calc II}

\section{Parametric Equations}
In middle school, you learned to write an equation of a line as $y=mx+b$.  In the vector unit, we learned to write this in vector form as $(x,y)=(1,m)t+(0,b)$. The equation to the left is called a vector equation.  It is equivalent to writing the two equations $$x=1t+0,y=mt+b,$$ which we will call parametric equations of the line. We were able to quickly develop equations of lines in space, by just adding a third equation for $z$.

Parametric equations provide us with a way of specifying the location $(x,y,z)$ of an object by giving an equation for each coordinate.  We will use these equations to model motion in the plane and in space.  In this section we'll focus mostly on planar curves.

\begin{definition}
If each of $f$ and $g$ are continuous functions, then the curve in the plane defined by $x=f(t),y=g(t)$ is called a parametric curve, and the equations $x=f(t),y=g(t)$ are called parametric equations for the curve. You can generalize this definition to 3D and beyond by just adding more variables.
\end{definition}

\begin{problem} 
\marginpar{
	\thomasee{See 11.1: 1-18. This is the same for all the problems below.}
	}%
By plotting points, construct graphs of the three parametric curves given below (just make a $t,x,y$ table, and then plot the $(x,y)$ coordinates).  Place an arrow on your graph to show the direction of motion.
\begin{enumerate}
\item $x=\cos t, y=\sin t$, for $0\leq t\leq 2\pi$.
\item $x=\sin t, y=\cos t$, for $0\leq t\leq 2\pi$.
\item $x=\cos t, y=\sin t, z=t$, for $0\leq t\leq 4\pi$.
\end{enumerate} 
\end{problem}

\begin{problem}
Plot the path traced out by the parametric curve $x=1+2\cos t, y=3+5\sin t$.  Then use the trig identity $\cos^2t+\sin^2t=1$ to give a Cartesian equation of the curve (an equation that only involves $x$ and $y$). What are the foci of the resulting object (it's a conic section).
\end{problem}

\begin{problem}\label{line equation to refer to}  \marginpar{What we did in the previous chapter should help here.}  
Find parametric equations for a line that passes through the points $(0,1,2)$ and $(3,-2,4)$.
\end{problem}

\begin{problem}
Plot the path traced out by the parametric curve $\vec r(t)= (t^2+1, 2t-3).$ Give a Cartesian equation of the curve (eliminate the parameter $t$), and then find the focus of the resulting curve.
\end{problem}

\begin{problem}
Consider the parametric curve given by $x=\tan t, y=\sec t$. Plot the curve for $-\pi/2<t<\pi/2$. Give a Cartesian equation of the curve (a trig identity will help).  Then find the foci of the resulting conic section. [Hint: this problem will probably be easier to draw if you first find the Cartesian equation, and then plot the curve.]
\end{problem}

\subsection{Derivatives and Tangent lines}\label{derivatives and tangent lines}
We're now ready to discuss calculus on parametric curves. The derivative of a vector valued function is defined using the same definition as first semester calculus.

\begin{definition}
If $\vec r(t)$ is a vector equation of a curve (or in parametric form just $x=f(t), y=g(t)$), then we define the derivative to be $$\frac{d\vec r}{dt}=\ds\lim_{h\to 0}\frac{\vec r(t+h)-\vec r(t)}{h}.$$
\end{definition}
The subtraction above requires vector subtraction.  The following problem will provide a simple way to take derivatives which we will use all semester long.

\begin{problem} 
\marginpar{
	\thomasee{See page 728.}
	}%
Show that if $\vec r(t) = (f(t),g(t))$, then the derivative is just $\frac{d\vec r}{dt} = (f'(t),g'(t))$.  

[The definition above says that $\frac{d\vec r}{dt}=\ds\lim_{h\to 0}\frac{\vec r(t+h)-\vec r(t)}{h}$. We were told $\vec r(t) = (f(t),g(t))$, so use this in the derivative definition.  Then try to modify the equation to obtain $\frac{d\vec r}{dt} = (f'(t),g'(t))$.]
\end{problem}
The previous problem shows you can take the derivative of a vector valued function by just differentiating each component separately. The next problem shows you that velocity and acceleration are still connected to the first and second derivatives. 

\begin{problem}  
\marginpar{
	\thomasee{See 13.1:5-8 and 13.1:19-20}
	}%
Consider the parametric curve given by $\vec r(t)=( 3\cos t, 3\sin t )$. 
\begin{enumerate}
\item Graph the curve $\vec r$, and compute $\frac{d\vec r}{dt}$ and $\frac{d^2\vec r}{dt^2}$. 
\item On your graph, draw the vectors $\frac{d\vec r}{dt}\left(\frac{\pi}{4}\right)$ and $\frac{d^2\vec r}{dt^2}\left(\frac{\pi}{4}\right)$ with their tail placed on the curve at $\vec r\left(\frac{\pi}{4}\right)$. These vectors represent the velocity and acceleration vectors.
\item Give a vector equation of the tangent line to this curve at $t=\frac{\pi}{4}$. (You know a point and a direction vector.)
\end{enumerate}
\end{problem}

\begin{definition}\label{definition velocity acceleration}
If an object moves along a path $\vec r(t)$, we can find the velocity and acceleration by just computing the first and second derivatives. The velocity is $\frac{d\vec r}{dt}$, and the acceleration is $\frac{d^2\vec r}{dt^2}$. Speed is a scalar, not a vector. The speed of an object is just the length of the velocity vector.
\end{definition}

\begin{problem}
Consider the curve $\vec r(t) = (2t+3, 4(2t-1)^2)$.
\begin{enumerate}
\item Construct a graph of $\vec r$ for $0\leq t\leq 2$. 
\item If this curve represented the path of a horse running through a pasture, find the velocity of the horse at any time $t$, and then specifically at $t=1$. What is the horse's speed at $t=1$?
\item Find a vector equation of the tangent line to $\vec r$ at $t=1$.  Include this on your graph.
\item Show that the slope of the line is 
$$\ds \frac{dy}{dx}\big|_{x=5} 
= 
\frac{
(dy/dt)\big|_{t=1}
}{
(dx/dt)\big|_{t=1}
}.$$
[How can you turn the direction vector, which involves $(dx/dt)$ and $(dy/dt)$ into a slope $(dy/dx)$?]
\end{enumerate} 
\end{problem}
%
%The previous problem introduced the following key theorem.  Its proof is just the chain rule.
%\begin{theorem}
%If $\vec r(t) = (x(t),y(t))$ is a parametric curve, then the slope $dy/dx$ of the curve can be found using the formula 
%$$\ds\frac{dy}{dx} = \frac{dy}{dt}\frac{dt}{dx} = \frac{dy/dt}{dx/dt}.$$
%The second derivative is then $\ds\frac{d^2y}{dx^2} = \frac{d(y'(x))}{dx} = \frac{d(dy/dx)}{dx}=\frac{d(dy/dx)/dt}{dx/dt}$.
%\end{theorem}
%An easy way to remember this theorem is to find $\frac{dy}{dx}$, just find the derivative of $y$ with respect to $t$, and then divide by $dx/dt$. This will allow you to connect derivatives of vector valued functions to slopes and derivatives back in first semester calculus.
%
%\begin{problem} \marginpar{\bmw{See 11.2:1-14}}
%Consider the parametric curve given by $\vec r(t) = (t^2,t^3)$. 
%\begin{enumerate}
%\item Compute $y'$ and $y''$ at $t=2$ using the theorem above.
%\item Eliminate the parameter $t$ (get a Cartesian equation for the curve). Then find $y'$ and $y''$ at $t=2$ using first semester calculus.
%\end{enumerate}
%\end{problem}

\subsection{Arc Length}\label{arc length}
If an object moves at a constant speed, then the distance travelled is 
$$\text{distance} = \text{speed}\times\text{time}.$$
This requires that the speed be constant.  What if the speed is not constant? Over a really small time interval $dt$, the speed is almost constant, so we can still use the idea above. The following problem will help you develop the key formula for arc length.

\begin{problem}[Derivation of the arc length formula] 
Suppose an object moves along the path given by $\vec r(t)=(x(t),y(t))$ for $a\leq t\leq b$. 
\begin{enumerate}
\item Show that the object's speed at any time $t$ is 
$\ds\sqrt{\left(\frac{dx}{dt}\right)^2+\left(\frac{dy}{dt}\right)^2}$.
\item If you move over a really small time interval, say of length $dt$, then the speed is almost constant. If you move at constant speed $\ds\sqrt{\left(\frac{dx}{dt}\right)^2+\left(\frac{dy}{dt}\right)^2}$ for a time length $dt$, what's the distance $ds$ you have traveled. 
\item  Explain why the length of the path given by $\vec r(t)$ for $a\leq t\leq b$ is  \marginpar{This is the arc length formula. Ask me in class for an alternate way to derive this formula.}
$$s=\int ds=\int_a^b \left|\frac{d\vec r}{dt}\right| dt=\int_a^b \sqrt{\left(\frac{dx}{dt}\right)^2+\left(\frac{dy}{dt}\right)^2}dt.$$
%\item \marginpar{\bmw{See page 639.}} Draw a small curve.  Pick two points close together.  Construct a straight line segment between them (call this $ds$).  Then draw a right triangle that shows the change in $x$ and change in $y$ (written $dx$ and $dy$).  Use the Pythagorean theorem to show that $ds=\sqrt{dx^2+dy^2}=\sqrt{\left(\frac{dx}{dt}\right)^2+\left(\frac{dy}{dt}\right)^2}dt$.  
%\item If the curve is in space (so $\vec r(t)=(x(t),y(t),z(t))$ is the path), then what is the arc length of the curve? 
\end{enumerate}
\end{problem}

\begin{problem} \marginpar{\thomasee{See 11.2: 25-30}}
Find the length of the curve  $\ds \vec r(t) = \left(t^3,\frac{3t^2}{2}\right)$ for $t\in[1,3]$. The notation $t\in[1,3]$ means $1\leq t\leq 3$. Be prepared to show us your integration steps in class (you'll need a $u$-substitution).
\end{problem}

\begin{problem}
For each curve below, set up an integral formula which would give the length, and sketch the curve. Do not worry about integrating them.  \marginpar{The reason I don't want you to actually compute the integrals is that they will get ugly really fast. Try doing one in Wolfram Alpha and see what the computer gives.}
\begin{enumerate}
\item The parabola $\vec p(t) = (t,t^2)$ for $t\in[0,3]$.
\item The ellipse $\vec e(t) = (4\cos t,5\sin t)$ for $t\in[0,2\pi]$.
\item The hyperbola $\vec h(t) = (\tan t,\sec t)$ for $t\in[-\pi/ 4,\pi/4]$.
\end{enumerate}
\end{problem}
To actually compute the integrals above and find the lengths, we would use a numerical technique to approximate the integral (something akin to adding up the areas of lots and lots of rectangles as you did in first semester calculus).