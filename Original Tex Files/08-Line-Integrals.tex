\newcommand{\sageworkurl}{http://bmw.byuimath.com/dokuwiki/doku.php?id=work_calculator}
%{http://aleph.sagemath.org/?z=eJylU0uP2jAQvvtXjMIBpxi0rPaaU6v0tHtpJQ7VFpnYAWuDHTmTkPTXd-w82K6qXoqEMPle84g76fka12Ld03dYpxA_qy-6NFbD4FoPnfRGnirdMM8xhQw4iqct_nycyd9Q4sStpZdXjd78kmicBWAn11rVjKrtXkyi1VfTaZiw0nnAi_4oZjnvxRADHz_1m0Fs-_QeFwSdLpC0pdGVYv09qb8nrb5fdLMkSa-hbXTZVmDKUDHcpEVAB1f5pkHC2TkFdeVwx4a73yAexNN_2jHlyUqZsuRe4DQ6MvzsrnU79aO0Nx313mmmDhm1n_kfD69ioJ_9a7pTDo-1d6otkKuwixRWubGKgiqDWFFdBsGVcHP-bQdAtULhlIbgAmeybWJMaXyDhFxrZ7VFAZI8QsQ7TqMLZ9UO2CEzFvXZ08y5Ooix85Sx2tNzSEJGiANFXnAaIM8SMS5OQAKycvYcDYvW08Y9oaHygLlOj3ufppmI6SQSMPHvYU5J5oMf1xvfCiIEqxmK8-WqF2pG1Xs4B2MBtb82YUI4Mj6OeCY_LzBFBZzK_auAjrPm5Q_N_h-a_aJRB6Lku1j6s-o3L2qIGrV0HgjLAmZqHA2rs-XCFMfwkvE42XlD9SYLD4_jJTnGS8JzMV0TMcy85uJuvE5_A6PQPsA=&lang=sage}

\newcommand{\sagefluxurl}{http://bmw.byuimath.com/dokuwiki/doku.php?id=flux_calculator}
%{http://aleph.sagemath.org/?z=eJytUz1v2zAQ3fkrDvJg2qWNtEGyaWqhoEOytECGIjAYkbKJyqRAUrLUX9_jh-zEWWPAMM139-7d3ePALV36JVuO-J2WK7j6LH7IRmkJk-ktDNwq_tpKRyz1KyiB3q1r4_DM7tZOaTxkgsUvz33O6rjlR-mt-se9MhqAvJpeCxfyPbth39adel938aAGCTmqMRb8QV7TkIqObIoibtcju11PV9qzhJA6yNojS6NkK8h4qT6yzT27v877fZDuXJ1bCb2TTd-CakI_cOLagzdw5H8lcNgbI6Brjd-S6cI8fS4zERZZhWoaapn_sCSk_m6OXZ_bFdKqAYc0SJLkpP4pFfbP1xe2wZ-bl7SpxYP0MUcbe-TtPKnXCdyJd53S-4gai5zAtYD6wPV-vnZqr4O4qu3HEvdRBmI2laHMaiuM33XWiL72FHVEcywqhSQcWuV9i6NQHkwDIX8Lz6jceKhinSCcRF6lvdxbXCWNdVgaX1L_c8Y-UDaBkpDOYjoUOPh4E4uVBUveYVAAr61xLnZT9xZdZxEO7g6gGWTyXl5ZwfKJFaDi36BoLlLMB5u8Fc2JMYFthuIeqRiZmFHxFs7uEWgfMeaANLo5ogr4I3vKYGrkAioNXtqjC436c8i7vZylBO0YUm1T1UcxbZ7EGJPE276e8eq8BAyPPcy9k648P8x6F9xK4_Tylkj3pQyXu2SsXXyCtGL5EbJpjnMHc6Ld6j9AtGK2&lang=sage}

\newcommand{\sageworkfluxurl}{http://bmw.byuimath.com/dokuwiki/doku.php?id=both_flux_and_work}
%{http://aleph.sagemath.org/?z=eJy1VFFv2jAQfvevOFEkHGrYuqnVNMnSNmk8td1UVeKBIpTGDlgNdnR2KPn3O8dpC6x7XB6wuXzf3Xff2dnlyEd70YowythM7nQRHPLFpBX7ZcbwNXA5LpznIRNwOfbG0o5eBx9yDPIjC9oq-WlcG8YSYVUaXanVo2us8iCB872YXIkr4vM27TLG2Blc6zDysM2fNORQNrYIxllAs94EcGUJwYHe5VWTBwJYMDboNeYVuJ1GCBvjoWhwp6dM6RIqY_XqBcL7jVXZVwb0oA4NWri747bZajRFXr2BD9BT3zx6vpe4-LgU0NJ6sSTdqVlaqdeMXpF-hVKZsuQoQsbUXM5OWWOFcAY3TRVMXbXGriE8O0gWeeptrcOG-libnaa_Gw3KBajRqaYIbC6P-1FzKkl2kvfRURUrCJgo7MSoWdXs31FABNa9OkkWY2kE999_XP9km7CtpiF_rDRfsAUOhg8kEyifaLPhQMBsKd7CSBpiEPtgGAJZt9YU4uHEqh6iEjGyFB7mmvHXlKnQaQsvCeYy4eGhiDYdJpz3GIJQh6D69V_4-WF9C8rHYHL2pVhyMzEjghy_AdXCBG5B7buiEfLK6PDDVL3bd9vjBF3oRrWT25Qh4pYsHqQz-H396559i-PBnGYfT_OKk41By8WgrNzzQNDS7AfL_jTXsq5cWB1eNz4TMH7n_mWJcC6hzjHf6kBnfxXZdHD_npegW1U8We29_JyYpoSkRCYlSUB8Sodg6FKC78bPDxPF1LRMUiz7cPEleyN2goAU5YjumdPwpYnfBjyH6SXdmiwFClc5lCPUapSk6OpIDPnx38VEA4_UrFFr2-vpfvyGWLWA3Ndk_gpz-obJi-wPG3Sl-A}

\newcommand{\sagelineintegral}{http://bmw.byuimath.com/dokuwiki/doku.php?id=line_integral_calculator}
%{http://aleph.sagemath.org/?z=eJx1VEtv2zAMvutXEEmB2KmbNelpA3wIdhowYEA7YIc2DRSLdoTaUqBHHv9-lGS7XZHlEoL8-NDHj55Of-8RKi0QdtjqE5xk20ItlQBHAetNzSsEbpCDroGPHrfnDlqJts-rs3NxyYFTIt_pI8b0yhuyTOZykCp6zhc4tFzhAoCllPLr3fl1dXd5XbEALLPVvNKWrAIe5lYqsnLmDFcNlpkr7ovV_CBzxoQtzULIuibAQmnTZTkT65KKlqHO8_2muCRrucnnIlRkbAo_0c0sdPyNHgW1V5WTWoGRzd7RA2twGvDIW88dARSN7bAxvAV6kqEXSJseFeaf_sEZkeVtKKV8h0ZWhBxSCmKm4iFqdYdOdsTVCaGLrZR2FCaqWgwtG3ShG5555cYCCyawJo4VbgdX1htK5N8Y0M-g80bB42M2DnAFXEAikKgYzdUmcEScrKH81ESsI-mM7V3XLlwYMntmz2Zyk1Z2MymSsSmi9-WIVdwyBcJf7xaj_4twITSs6wPEkj_uZsyxsul4WUMKrd9DT73y1kGLJSQglC809vY79AlrAodH0Y-UTduymDYcNm6jBA-tdknLHxS-YGZJ6vKkv0E90Iun8PProspZlMJWH0IHWwpZuazSrTblzKCYFSSX6k2hteVDj7XlgRtOaqBNbcMgWWx7nw9rKebzf4rmt1cTlnnxHzyze0R3pc2QkHm6oWUegPqUpaluY1K4jx91ZGWUPTEodBBAEfwqXUD8RjSyP_JRrhD1kk0-16hI2bvwkekO3qFIKm8vsLvAE2-GytKlunhE5TxvCRAmBH9IXxi6uF8ENCdpKeWifbi9eDew9w2ShG1cKxqjjV1MctYP4DAJeqA4_wsUdqHx&lang=sage}


\newcommand{\sagephysicalpropertiestwod}{http://bmw.byuimath.com/dokuwiki/doku.php?id=physical_properties_in_2d}
%{http://aleph.sagemath.org/?z=eJyFVU1z2jAQvetXaDLMIBORONBeMqNTe-GQi69APMIW4Kllu5KgVhn-e1eybD6atFxs6e2-fX7aFUeuyLillppxhHJR6cJYAuuIte-zR4sUO4rM1Ios55Os1sREFM8nuqjgbR0how1XhsXIiCpnTYFMqni1E4wY2mHUIRFCSKWNKqTADOfFdksUNVBQs2HxVNVKkghpVlRG7BQvSa7pJBBGKBOVUXWRpy2rSHWQQhUZL9MhmLSTXEekZWoZr0FlKI99fdh61lcc9hMO6zkscLx8xpFLRq6NuqpKcchEkgqlasU-qJHLO17Ej7s0MII78lkjOIktLotKXNLCC8S_Igw_JcxBVThJPvyOS_STPmz0vb6_vwxO6K1ltyXBUBmhN3u3bf22s1KoVHKt0xZkk7f2WcIhklvIesj20KKlC0sXMewuP_s-vK0VHpbwhol9n02cb233JL4332eRU7JGiauvfypDFk4DSuywtn4dD-vYrdHeyPLJ8E0pyBIt1cP3YP5olYvScH-soweKr495TV3kt4M6CoiDocAKRgCiVIeMzAj7ToW80LMBWG0Vz055l3M-5ebsuMM4hJhcjyjmcOTGlAJvCoPrLeYqK0W1M3snRYdIPcLTGyQA30Jrg7hSbA3pqr6cT_q8AjNxu6IwTve71u2uVLHb-2_B5DJl9DIsUa9Sss4hkONl3Al2J-60yhAP4a7ITU6ICSH8KBTfid5pPBoEyk4gJD1gsOYyIq6Pemf9h0E9J_2tvd61_a692CNUL_LGpBeJOyLXYTc7Npf33gwtT_FNm_eSFn4Yuuz3WRCxAGkdaDtH2gtkeyjuoKG3Ax4H4gTuvZVr4hOUeJa-h5LAmzjegEHHnx1keygeoNhDSbxGbtzrxmh2Grsp1MaWYvw6zrnei3x8Rg1ruOJSGLhX0qasDdzQfVfDy77IfkCWZl8i1Dyypgbh5N_20Kwu4T4cK-CnPkEXvwX7GlNcih1cQmnJN6Jkdwf1cF8g9Ca-ak5YdNw7JUQ1pvh_9C6vI84KN0iExDSOwBg6mThXPOR8IcvldE69y_6x7gMeL-g0wNNr_IogsXQ6hwh4zj8g6OHpLY70vv5FGmh83cD_Lzhvipq9RH8Ah12GVw%3D%3D}

\newcommand{\sagephysicalpropertiesthreed}{http://bmw.byuimath.com/dokuwiki/doku.php?id=physical_properties_in_3d}
%{http://aleph.sagemath.org/?z=eJx9Vk2PqzYU3ftXWNNIMRNPJslrN5VYTTcjtRvUXZKHPOAk6GGgtpNij-a_99p8BcJrNsH3Hh_fD58LkiuuSYBuTJJlTQ3VywClvFCZNsStbRDWK7OySIY3nuhSkv3uOSkV0QHFu2eVFf5JHwOEfpEh6UAbuqObY_DSrbd0S78dg-BZryYmpJVmUocbpHmRht-eqwzpWLLizEOiaeOlzueOwH9yvVRYsB8cM3y6FonOygLL7HzRuDydsC4xv7H8yjQACpwVmp8ly3F54xLrS6ZwcpU3voYsTzjPCh53ENI-wEG_Iww_yfVVFjiKSHEVXGYJywfwHXqtrh-K1KHcb44UG_jfHqmFv93RlaZJAPsMAAFZyLiSmeA4xGl2OhFJNRRdhf1iXZRSQFdUOA4wVQFKeKFlmaVxPXHWz-B-VQPATABmCrATgG0BqQjHVwDsSExjEQFit3PcIiEXAVuRKAXQx-yjvOrY2LjKWcEfI4XNI2Q9jzQzSDOLtB7pUuMyFkypuHYhzUXzKkY4M8XV8zj7gDM9Dr1Pu0HM993Kft8FPq73aS9IPXJPO-HdpnNHLhX1j9TkvX51a9OvjV_bfm3dGp1kKbBiZ74WmUrWFy1ynImqlBqEoy9xxaQCdYCMKoScd63ZR87JHu1FReTTH21PF4eU55q5WxAsnuAut80-0hb45qQEMJA0ljAIHEj23oVeYK9jt7WV9OBsNi3bXbjVxOBP1YKCxPNM65zjj8zJGzOZ5Lw464uPRg1otcAvY-_gfGsvPCZJWYAaC90lEkDsOT9pcjhJlnxuvz7F1wG6gOsDTRWdWo234qnZOvPBj6AmFzKIlA5yHB5tcJemCJsqQzo-h0nG7u75ZMWwB7a4c0f7OtwAYzDx4A50uUKqk7hh41OAociDjNcFuYvtr_jT2K_msBoi8Od4CWAvAbwwdoG9CPzRc2obsdUdm5llq3_KVs-ymZbNzrOZn7KZCdub13lX7tGl2Arc5J-KtvOdxTxYXBwP96CfRxSPxs54eX8l3v3s8oz9GHEZOsr3-kh7mGnS74dJDzJ3IHsHciOlw9jhwAheJgc3QD7h6Ffx5QHRcFLkTmr9MHG-vNvcuW3vto07skfk3nSoCmHYMME1vD6h6KWGF1w3C6h7Hyc_Cq5U-GuAqlVYlRAq-f-yjatGkzIvZbiUPF1Sv11lloe_baZ8rRzxvB7B3vCcJefFkuIHKjegyR6-azb0xc_ayBU0chWFQjjz1HqkkqXZVYWRpWXFEpBXuFlv79laJtfkyDUQWOATYj9nHdjqezY6insc6X2cDY-P9NE6cJs5bldb6KW6lP-SCsaFquALDloIX17hlrKaq_BveeUB-g_6R3nE&lang=sage}


\note{After 3 semesters of trying this, I need to rewrite the section again.  I put lots of notes in my notebook for this semester. I'll rewrite the section soon. Some problems need to be combined.  Some need to be dropped.  Other sections need more simpler problems (work/flux).  Get rid of many of the ``use a computer to integrate'' problems.  Just have them focus on what I'll ask them to do anyway. }


\noindent 
This unit covers the following ideas. In preparation for the quiz and exam, make sure you have a lesson plan containing examples that explain and illustrate the following concepts.  
\begin{enumerate}
\item Describe how to integrate a function along a curve. Use line integrals to find the area of a sheet of metal with height $z=f(x,y)$ above a curve $\vec r(t)=\left(x,y\right)$ and the average value of a function along a curve.
\item Find the following geometric properties of a curve: centroid, mass, center of mass, inertia, and radii of gyration.
\item Compute the work (flow, circulation) and flux of a vector field along and across piecewise smooth curves.
\item Determine if a field is a gradient field (hence conservative), and use the fundamental theorem of line integrals to simplify work calculations.
\end{enumerate}
You'll have a chance to teach your examples to your peers prior to the exam. Table \ref{line integral summary} contains a summary of the key ideas for this chapter. 
\bmw{Here are some extra homework problems which line up with the text.

{\noindent %\footnotesize 
\begin{tabular}{|l|c|l|l|l|l|}\hline
Topic (11th ed.) &Sec &Basic Practice &Good Problems &Thy/App &Comp \\\hline
Line integrals & 16.1&1-8, 9-22, 23-32  & & &33-36 \\\hline
Work, Flow, Circulation, Flux & 16.2&7-16, 25-28, 37-40 &17-24, 29-30, 41-44 &45-46 &47-52   \\\hline
Gradient Fields & 16.2&1-6 & & &   \\\hline
Gradient Fields & 14.5&1-8 & & &   \\\hline
Potentials & 16.3&1-12,13-24 & 25-33 & 34-38&  \\\hline 
\end{tabular}

}

{\noindent %\footnotesize 
\begin{tabular}{|l|c|l|l|l|l|}\hline
Topic (12th ed.) &Sec &Basic Practice &Good Problems &Thy/App &Comp \\\hline
Line integrals & 16.1&1-8, 9-26, 33-42 &27-32 & &43-46 \\\hline
Work, Flow, Circulation, Flux & 16.2&7-12, 19-24, 31-36, 47-50 &13-18, 25-30, 37-38,  & 51-54 &55-60    \\\hline
Gradient Fields & 16.2&1-6 & & &   \\\hline
Gradient Fields & 14.5&1-10 & & &   \\\hline
Potentials & 16.3&1-12,13-24 & 25-33 & 34-38&  \\\hline 
\end{tabular}
}
}%end \bmw


\begin{table}[h]
 \begin{center}
\begin{tabular}{|c|c|}
 \hline
 Surface Area& 
     $\sigma = \int_C d\sigma=\int_C f ds = \int_a^b f \left|\frac{d\vec r}{dt}\right|dt$\\
 \hline
 Average Value& 
     $\bar f = \frac{\int f ds}{\int ds}$\\
 \hline
 Work, Flow, Circulation &
     $W=\int_C d\text{Work} = \int_C (\vec F\cdot \vec T) ds = \int_C \vec F\cdot d\vec r = \int_C Mdx+Ndy$\\
 \hline
 Flux & 
     $\text{Flux} = \int_C d\text{Flux} = \int_C \vec F\cdot \vec n ds = \oint_C Mdy-Ndx$\\
 \hline
 Mass& 
     $m=\int_C dm = \int_C \delta ds $\\
 \hline
 Centroid& 
     $\left(\bar x,\bar y,\bar z\right) =\left(\frac{\int x ds}{\int_C ds},\frac{\int y ds}{\int_C ds},\frac{\int z ds}{\int_C ds}\right)$\\
 \hline
 Center of Mass & 
     $\left(\bar x,\bar y,\bar z\right) =\left(\frac{\int x dm}{\int_C dm},\frac{\int y dm}{\int_C dm},\frac{\int z dm}{\int_C dm}\right)$\\
 \hline
% First moment of mass & 
%     $M_{yz}=\int x dm$, $M_{xz} = \int x dm$, $M_{xy}=\int x dm$\\
% \hline
% (Second) Moment of Inertia & 
%     $I_x = \int (y^2+z^2) dm$, $I_y = \int (x^2+z^2) dm$, $I_z = \int (x^2+y^2) dm$ \\
% \hline
% Radius of Gyration &
%     $R = \sqrt{I/m}$\\
% \hline
% Fund. Thm. Calculus & 
%     $f(b)-f(a)=\int df = \int \frac{df}{dx}dx = \int_a^b f^\prime(x)dx$\\
 \hline
 Fund. Thm of Line Int. &
    $f(B)-f(A)=\int_C \vec \nabla f \cdot d\vec r$ 
    %\int_C df = \int_C \frac{df}{ds}ds = \int_a^b \frac{df/dt}{ds/dt}ds = 
    %\int_a^b \vec \nabla f \cdot \frac{d\vec r}{dt}dt 
    %= \int_C \vec F \cdot d \vec r$
    \\
\hline
\end{tabular}
\caption{A summary of the ideas in this unit.\label{line integral summary}}
\end{center}
\end{table}


I have created a YouTube playlist to go along with this chapter. Each video is about 4-6 minutes long.
\begin{itemize}
 \item \href{http://www.youtube.com/playlist?list=PL04DF68E73B7ECD54}{YouTube playlist for 08 - Line Integrals}.
 \item \href{http://db.tt/dAFBcMB7}{A PDF copy of the finished product} (so you can follow along on paper).
\end{itemize}
You'll also find the following links to Sage can help you speed up your time spent on homework. Thanks to Dr. Jason Grout at Drake university for contributing many of these (as well as being a constant help with editing, rewriting, and giving me great feedback).  Thanks Jason. 
\begin{itemize}
 \item \href{http://bmw.byuimath.com/dokuwiki/doku.php?id=sage_links}{Sage Links}
 \item \href{https://content.byui.edu/file/3e8d885f-db47-4e74-9e04-c3d72627c835/1/_zips/215-Tech-Introduction.zip}{Mathematica Notebook} (If you have installed Mathematica)
\end{itemize}


\bmw{If you would like homework problems from the text that line up with the ideas we are studying, please use the following tables.

{\noindent \footnotesize 
\begin{tabular}{|l|c|l|l|l|l|}\hline
Topic (11th ed.) &Sec &Basic Practice &Good Problems &Thy/App &Comp \\\hline
Line integrals & 16.1&1-8, 9-22, 23-32  & & &33-36 \\\hline
Work, Flow, Circulation, Flux & 16.2&7-16, 25-28, 37-40 &17-24, 29-30, 41-44 &45-46 &47-52   \\\hline
Gradient Fields & 16.2&1-6 & & &   \\\hline
Gradient Fields & 14.5&1-8 & & &   \\\hline
Potentials & 16.3&1-12,13-24 & 25-33 & 34-38&  \\\hline 
\end{tabular}

}

{\noindent \footnotesize 
\begin{tabular}{|l|c|l|l|l|l|}\hline
Topic (12th ed.) &Sec &Basic Practice &Good Problems &Thy/App &Comp \\\hline
Line integrals & 16.1&1-8, 9-26, 33-42 &27-32 & &43-46 \\\hline
Work, Flow, Circulation, Flux & 16.2&7-12, 19-24, 31-36, 47-50 &13-18, 25-30, 37-38,  & 51-54 &55-60    \\\hline
Gradient Fields & 16.2&1-6 & & &   \\\hline
Gradient Fields & 14.5&1-10 & & &   \\\hline
Potentials & 16.3&1-12,13-24 & 25-33 & 34-38&  \\\hline 
\end{tabular}

}
}




\note{
Do I need to focus on summation notation?  I think it could be valuable to do so.  
Do I need to start with area, or would it be better to start with work.  I'm thinking work might be a better way to start.  Then I can put random work problems in the middle (just to beef up our abilities).
Add flux midway through.

Start with work.  Then talk about area. Then talk about flux. Then talk about average value and center of mass.  Leave out inertia for now (add it in later - this removes 5 problems, an entire day.  They just have to be added later on, and the quizzes and exams have to be updated.)  

Then come back to work, and get to potentials. End with potentials and the fundamental theorem.  

Motivate everything with discussing the airflow along a wing.  

Does this cover it all?

Problems 8.1 and 8.2 have to be shrunk.  They cannot be that long. They really are short, so make them short.  Get rid of all the extra fluff, and put it later. 


}


\section{Work, Flow, Circulation, and Flux}

Now that we can describe motion, let's turn our attention to the work done by a vector field as we move through the field. Work is a transfer of energy. 
\begin{itemize}
 \item A tornado picks might pick up a couch, and applies forces to the couch as the couch swirls around the center. Work transfers the energy from the tornado to then couch, giving the cough it's kinetic energy. 
 \item When an object falls, gravity does work on the object. The work done by gravity converts potential energy to kinetic energy. 
 \item If we consider the flow of water down a river, it's gravity that gives the water its kinetic energy. We can place a hydro electric dam next to river to capture a lot of this kinetic energy.  Work transfers the kinetic energy of the river to rotational energy of the turbine, which eventually ends up as electrical energy available in our homes.  
\end{itemize}
When we study work, we are really studying how energy is transferred. This is one of the key components of modern life.

Let's start with simple review. Recall that the work done by a vector field $\vec F$ through a displacement $\vec d$ is the dot product $\vec F\cdot \vec d$. 
\begin{review*}
 An object moves from $A=(6,0)$ to $B=(0,3)$. Along the way, it encounters the constant force $\vec F = (2,5)$.  How much work is done by $\vec F$ as the object moves from $A$ to $B$? See \footnote{The displacement is $B-A=(-6,3)$. The work is $\vec F\cdot \vec d = (2,5)\cdot(-6,3) = -12+15=3$.}.
\end{review*}

\begin{problem}
 An object moves from $A=(6,0)$ to $B=(0,3)$. A parametrization of the objects path is $\vec r(t) = (-6,3)t+(6,0)$ for $0\leq t\leq 1$.  
\begin{enumerate}
 \item  For $0\leq t\leq .5$, the force encountered is $\vec F = (2,5)$.  For $.5\leq t\leq 1$, the force encountered is $(2,6)$.  How much work is done in the first half second? How much work is done in the last half second?  How much total work is done?
 \item If we encounter a constant force $\vec F$ over a small displacement $d\vec r$, explain why the work done is \
$\ds dW = \vec F\cdot d\vec r =F\cdot \frac{d\vec r}{dt}dt $. 
 \item\marginpar{You can visualize what's happening in this problem as follows.  Attach a clothesline between the points (maybe representing two trees in your backyard).  Put a cub scout space derby ship on the clothesline.  Then the wind starts to blow.  As the ship moves along the clothesline, the wind changes direction.}%
Suppose that the force constantly changes as we move along the curve. At $t$, we'll assume we encountered the force $F(t) = (2,5+2t)$, which we could think of as the wind blowing stronger and stronger to the north.  
 Explain why the total work done by this force along the path is $$\ds W=\int \vec F\cdot d\vec r = \int_0^1 (2,5+2t)\cdot (-6,3)dt.$$ Then compute this integral. It should be slightly larger than the first part. 
 \item (Optional) If you are familiar with the units of energy, complete the following. What are the units of $\vec F$,  $d\vec r$, and $dW$.
\end{enumerate}
\end{problem}

We know how to compute work when we move along a straight line. Prior to problem \ref{first work problem} on page \pageref{first work problem}, we made the following statements. 
\begin{quote}If a force $F$ acts through a displacement $d$, then the most basic definition of work is $W=Fd$, the product of the force and the displacement.  This basic definition has a few assumptions.
\begin{itemize}
\item The force $F$ must act in the same direction as the displacement.
\item The force $F$ must be constant throughout the  displacement.
\item The displacement must be in a straight line.
\end{itemize}
\end{quote}
We used the dot product to remove the first assumption, and we showed in problem \ref{first work problem} that the work is simply the dot product $$W=\vec F\cdot \vec r,$$
where $\vec F$ is a force acting through a displacement $\vec r$. The previous problem showed that we can remove the assumption that $\vec F$ is constant, by integrating to obtain $$W=\int \vec F \cdot d\vec r = \int_a^b F\cdot \frac{d\vec r}{dt}dt, $$ provided we have a parametrization of $\vec r$ with $a\leq t\leq b$. The next problem gets rid of the assumption that $\vec r$ is a straight line.  

\begin{problem}
\marginpar{\href{http://www.youtube.com/watch?v=9TGZIIpEaHw&list=PL04DF68E73B7ECD54&index=2&feature=plpp_video}{Watch a YouTube video} about work.}%
 Suppose that we move along the circle $\vec r(t) = (3\cos t,3\sin t)$. As we move along this circle, we encounter a rotation force $\vec F(x,y) = (-2y,2x)$.
\begin{enumerate}
 \item Draw the curve $\vec r(t)$. Then at several points on the curve, draw the vector field $\vec F(x,y)$.  For example, at the point $(3,0)$ you should have the vector $\vec F(3,0)=(-2(0),2(3))=(0,6)$, a vector sticking straight up 6 units. Are we moving with the vector field, or against the vector field?
 \item Explain why we can state that a little bit of work done over a small displacement is $dW = \vec F\cdot d\vec r$. Why does it not matter that $\vec r$ moves in a straight line? 
 \item Explain why the work done by $\vec F$ along the circle $C$ \marginpar{We put the $C$ under the integral $\int_C$ to remind us that we are integrating along the curve $C$.  This means we need to get a parametrization of the curve $C$, and give bounds before we can integrate with respect to $t$.}is 
$$W = \int_C\left(-2y,2x\right)\cdot d\vec r
= \int_0^{2\pi}\left(-2(3\sin t),2(3\cos t)\right)\cdot(-3\sin t, 3\cos t)dt.$$\
 Then integrate to show that the work done by $\vec F$ along this circle is $36\pi$.  
\end{enumerate}
 
\end{problem}

It's time for a definition. 
\begin{definition}
The work done by a vector field $\vec F$, along a curve $C$ with parametrization $\vec r(t)$ for $a\leq t\leq b$ is
$$W = \int_C \vec F\cdot d\vec r= \int_a^b \vec F\cdot \frac{d\vec r}{dt}dt.$$
If we let $\vec F = (M,N)$ and we let $\vec r(t)=(x,y)$, so that $d\vec r = (dx,dy)$, 
then we can write work in the differential form 
$$W = \int_C \vec F\cdot d\vec r= \int_C (M,N)\cdot (dx,dy) = \int_C Mdx+Ndy.$$
\end{definition}

\begin{review*}
 Consider the curve $y=3x^2-5x$ for $-2\leq x\leq 1$.  Give a parametrization of this curve. See \footnote{Whenever you have a function of the form $y=f(x)$, you can always use $x=t$ and $y=f(t)$ to parametrize the curve.  So we can use $\vec r(t) = (t, 3t^2-5t)$ for $-2\leq t\leq 1$ as a parametrization.}. 
\end{review*}


\begin{problem}
\marginpar{Please use this \href{\sageworkurl}{Sage link} to check your work.}%
 Consider the parabolic curve $y=4-x^2$ for $-1\leq x\leq 2$, and the vector field $\vec F(x,y) = (2x+y,-x)$. 
\begin{enumerate}
 \item Give a parametrization $\vec r(t)$ of the parabolic curve that starts at $(-1,3)$ and ends at $(2,0)$.  See the review problem above if you need a hint.
 \item Compute $d\vec r$ and state $dx$ and $dy$. What are $M$ and $N$ in terms of $t$?
 \item Compute the work done by $\vec F$ to move an object along the parabola from $(-1,3)$ to $(2,0)$ (i.e. compute $\int _C Mdx+Ndy$). Check your answer with \href{\sageworkurl}{Sage}. 
 \item How much work is done by $\vec F$ to move an object along the parabola from $(2,0)$ to $(-1,3)$.  In general, if you traverse along a path backwards, how much work is done?  
\end{enumerate}
\end{problem}

\begin{problem}
 Again consider the vector field $\vec F(x,y) = (2x+y,-x)$. In the previous problem we considered how much work was done by $\vec F$ as an object moved along the the parabolic curve $y=4-x^2$ for $-1\leq x\leq 2$. We now want to know how much work is done to move an object along a straight line from $(-1,3)$ to $(2,0)$.    
\begin{enumerate}
 \item Give a parametrization $\vec r(t)$ of the straight line curve that starts at $(-1,3)$ and ends at $(2,0)$. The opening problem of this chapter shows you the key to parameterizing a straight line segment.  Make sure you give bounds for $t$. 
 \item Compute $d\vec r$ and state $dx$ and $dy$. What are $M$ and $N$ in terms of $t$?
 \item Compute the work done by $\vec F$ to move an object along the straight line path from $(-1,3)$ to $(2,0)$. Check your answer with \href{\sageworkurl}{Sage}. \marginpar{When you enter your curve in Sage, remember to type the times symbol in ``(3*t-1, ...)''.  Otherwise, you'll get an error.}
 \item Optional (we'll discuss this in class if you don't have it).  How much work does it take to go along the closed path that starts at $(2,0)$, follows the parabola $y=4-x^2$ to $(-3,1)$, and then returns to $(2,0)$ along a straight line. Show that this total work is $W=-9$.   
\end{enumerate}
\end{problem}


The examples above showed us that we can compute work along any closed curve.  All we have to do is parametrize the curve, take a derivative, and then compute $dW = \vec F \cdot d\vec r$. This gives us a little bit of work along a curve, and we sum up the little bits of work (integrate) to find the total work. 

In the examples above, the vector fields represented forces. However, vector fields can represent much more than just forces. The vector field might represent the flow of water down a river, or the flow of air across an airplane wing.  When we think of the vector field as a velocity field, then we mights ask the question, how much of the fluid flows along our curve. Alternately, we might ask how much of the fluid flows across our curve.  These two problems lead to flow along a curve, and flux across a curve. Flow along a curve is directly related to the lift of an airplane wing (which occurs when the flow along the top of the wing is different than the flow below the wing).  The flux across a curve will quickly take us to powering a wind mill as wind flows across the surface of a blade (once we hit 3D integrals).


\begin{review*}
 If the unit tangent vector is $\vec T = \dfrac{(3,4)}{5}$, give two unit vectors that are orthogonal to $\vec T$. See \footnote{We just reverse the order and change a sign to get $\vec N_1 = \dfrac{(-4,3)}{5}$ and $\vec N_1 = \dfrac{(4,-3)}{5}$ as the orthogonal vectors.}.  
\end{review*}


\begin{problem}
We used the formulas $$W = \int_C \vec F\cdot d\vec r  = \int_C (M,N)\cdot(dx,dy)$$ to compute the work done by $\vec F$ along a curve $C$ parametrized by $\vec r = (x,y)$. 
\begin{enumerate}
 \item Explain why $W = \int_C \vec F\cdot \vec T ds$. [Why does $\vec T ds = d\vec r$?  Look up $\vec T$ in the last chapter.]
\marginpar{
\href{http://www.youtube.com/watch?v=6WcN36FbeWc&list=PL04DF68E73B7ECD54}{Watch a YouTube video about flow and circulation.}
}%
 \item Show that $\vec F\cdot \vec T$ is the projection of $\vec F$ onto the vector $T$ (the amount of work in the tangential direction). [Just write down the formula for a projection.  How long is $\vec T$?  This should be really fast.]
 \item 
\marginpar{\href{http://www.youtube.com/watch?v=5DNdI72XEYY&list=PL04DF68E73B7ECD54&index=4&feature=plpp_video}{Watch a YouTube video about flux}.
}%
We know that $\vec T = \dfrac{(dx, dy)}{\sqrt{(dx)^2+(dy)^2}}$.  Suppose $\vec n$ is a unit normal vector to the curve. Give two options for $\vec n$. [Hint: Look at the review problem.]
 \item We know  $\vec T ds = (dx,dy)$. Why does $\vec n ds$ equal $(dy,-dx)$ or $(-dy,dx)$?
 \item The integral $W = \int_C \vec F\cdot \vec T ds$ measures how much of the vector field flows along the curve.  What does the integral $\Phi = \int_C \vec F\cdot \vec n ds$ measure?
\end{enumerate}

\end{problem}

\begin{problem}[Intro to Flux]
 Consider the curve $\vec r(t) = (5\cos t, 5\sin t)$, and the vector field $\vec F(x,y) = (3x, 3y)$. This is a radial field that pushes things straight outwards (away from the origin).  
\begin{enumerate}
 \item Compute the work $\ds W= \int_C (M,N)\cdot (dx,dy)$ and show that is equals zero. (Can you give a reason why it should be zero?) \marginpar{See \href{\sageworkurl}{Sage} for the work calculation.}
 \item To get a normal vector, we could change $(dx,dy)$ to $(dy,-dx)$ or to $(-dy,dx)$. Compute both 
$\ds \int_C (M,N)\cdot (dy,-dx)$ and $\ds \int_C (M,N)\cdot (-dy,dx)$. (They should differ by a sign.) Both integrals measure the flow of the field across the curve, instead of along the curve. 
 \item If we want flux to measure the flow of a vector field outwards across a curve, then the flux of this vector field should be positive.  Which vector, $(dy,-dx)$ or $(-dy,dx)$,  should we choose above for $n$. \marginpar{See \href{\sagefluxurl}{Sage} for the flux computation}
 \item (Challenge, we'll discuss in class.) Suppose $\vec r$ is a counterclockwise parametrization of a closed curve.  The outward normal vector would always point to the right as you move along the curve.  Prove that $(dy,-dx)$ always points to the right of the curve. [Hint: If you want a right pointing vector, what should $\vec B=\vec T\times \vec n$ always equal (either $(0,0,1)$ or $(0,0,-1)$). Use the fact that $\vec B\times \vec T = \vec n$ to get $\vec n$.]
\end{enumerate}

\end{problem}

 

\begin{definition}[Flow, Circulation, and Flux]
\marginpar{
\begin{itemize}
 \item 
\href{http://www.youtube.com/watch?v=6WcN36FbeWc&list=PL04DF68E73B7ECD54}{Watch a YouTube video about flow and circulation.}
 \item
\href{http://www.youtube.com/watch?v=5DNdI72XEYY&list=PL04DF68E73B7ECD54&index=4&feature=plpp_video}{Watch a YouTube video about flux}.
\end{itemize}
}%
 Suppose $C$ is a smooth curve with parametrization $\vec r(t)=(x,y)$.  Suppose that $\vec F(x,y)$ is a vector field that represents the velocity of some fluid (like water or air).  
\begin{itemize}
 \item We say that $C$ is closed curve if $C$ begins and ends at the same point.
 \item We say that $C$ is a simple curve if $C$ does not cross itself. 
 \item The flow of $\vec F$ along $C$ is the integral $$\text{Flow} = \int_C (M,N)\cdot (dx,dy) = \int_C Mdx+Ndy.$$
 \item If $C$ is a simple closed curve parametrized counter clockwise, then the flow of $\vec F$ along $C$ is called circulation, and we write 
$\text{Circulation} = \oint_C Mdx+Ndy$
 \item The flux of $\vec F$ across $C$ is the flow of the fluid across the curve (an area/second). If $C$ is a simple closed curve parametrized counter clockwise, then the outward flux is the integral   
$$\text{Flux} = \Phi = \int_C(M,N)\cdot (dy,-dx) =\int_C Mdy-Ndx .$$
\end{itemize}

\end{definition}

\begin{problem}
\marginpar{If you haven't yet, please watch the YouTube videos for 
\begin{itemize}
 \item \href{http://www.youtube.com/watch?v=9TGZIIpEaHw&list=PL04DF68E73B7ECD54&index=2&feature=plpp_video}{work},
 \item \href{http://www.youtube.com/watch?v=6WcN36FbeWc&list=PL04DF68E73B7ECD54}{flow and circulation}, and 
 \item \href{http://www.youtube.com/watch?v=5DNdI72XEYY&list=PL04DF68E73B7ECD54&index=4&feature=plpp_video}{flux}.
\end{itemize}
}%
 Consider the vector field $\vec F(x,y) = (2x+y,-x+2y)$. When you construct a plot of this vector field, you'll notice that it causes objects to spin outwards in the clockwise direction. Suppose an object moves counterclockwise around a circle $C$ of radius 3 that is centered at the origin. (You'll need to parameterize the curve.)
\begin{enumerate}
 \item Should the circulation of $\vec F$ along $C$ be positive or negative?  Make a guess, and then compute the circulation $\oint_C Mdx+Ndy$. Whether your guess was right or wrong, explain why you made the guess. 
 \item Should the flux of $\vec F$ across $C$ be positive or negative? Make a guess, and then actually compute the flux $\oint_C Mdy-Ndx$. Whether your guess was right or wrong, explain why you made the guess. 
 \item Please use this \href{\sageworkfluxurl}{Sage link} to check both computations. 
\end{enumerate}
\end{problem}

We'll tackle more work, flow, circulation, and flux problems, as we proceed through this chapter.














\section{Area and Average Value}

\note{From Jason. 
I use a physical model.  I cut out the 
area under a curve on a piece of paper.  We then draw the representative 
calc 2 rectangle and label it with f(x) and dx.  Then I take the piece 
of paper and lay the "x-axis" of the paper on a curve on the whiteboard, 
so the area of paper is projecting out at them, and they see pretty 
clearly, in Real Life 3D, how to think about the surface area, and why 
the ds gives the width of the rectangle instead of dx.  Then the Sage 3d 
graph caps it off.}

In first semester calculus, we learned that the area under a function $f(x)$ above the $x$-axis is given by $A = \int_a^b f(x) dx$.  The quantity $dA= f(x) dx$ represents a small bit of area whose length is $dx$ and whose height is $f(x)$.  To get the total area, we just added up the little bits of area, which is why 
$$A=\int dA = \int_a^b f(x) dx.$$

\begin{problem}
\marginpar{\href{http://www.youtube.com/watch?v=sYsMcqtXBrc&list=PL04DF68E73B7ECD54&index=1&feature=plpp_video}{Watch a YouTube video.}}%
\marginpar{See \href{http://aleph.sagemath.org/?z=eJx1j8EOgyAQRO98hTeBroniqQe-pFFjECupFQJro39ftPXQxN5mknk7sz1dYGXymi21yNZaEE-RSSq4siEqSEoezBQVI-jb6a4lRchBcGcY8UU0c4xv0C2vINlFUcHMe8o3Ezk1-5durENjpyA7o5AqO1ovU6-7FHAw6jHpEGT5zQbpWt8-NXqjGjdapHtRHtd8NgDnP0fZ5RQoGPzJkzBojSc1B0Dn-GRxjA-XPf8GnQ5jSA}{Sage} for a picture of this sheet.}%
 Consider the surface in space that is below the function 
$f(x,y)=9-x^2-y^2$ and above the curve $C$ parametrized by 
$\vec r(t)=(2\cos t, 3\sin t)$ for $t\in[0,2\pi]$.  Think of this region as a metal plate that has been stood up with its base on $C$ where the height above each spot is given by $z=f(x,y)$.
\begin{enumerate}
 \item \marginpar{See Problem \ref{horse track chain rule introduction}}%
  Draw the curve $C$ in the $xy$-plane. If we cut the curve up into lots of tiny bits, explain why the length of each bit is approximately $$ds=\sqrt{(-2\sin t)^2+(3\cos t)^2}dt.$$
% \item The height of the surface above each little arc is given by $f(x,y)$.  Explain why the surface area of a little part of the surface is $$d\sigma = (9-4\cos^2t-9\sin^2t)\sqrt{(-2\sin t)^2+(3\cos t)^2}dt.$$
 \item Explain why the area of the metal sheet that lies above $C$ and under $f$ is given by the integral
$$\sigma = \int_C f ds = \int_0^{2\pi}(9-(2\cos t)^2-(3\sin t)^2)\sqrt{(-2\sin t)^2+(3\cos t)^2}dt.$$
 You'll need to explain why a little bit of surface area equals $d\sigma =fds$. 
 \item
\marginpar{\href{\sagelineintegral}{This Sage worksheet will compute the integral.}}%
Find the surface area of the metal sheet. [Use technology to do this integral.] 
\end{enumerate}
\end{problem}






Our results from the problem above suggest the following definition.

\begin{definition}[Line Integral]\marginpar{The line integral is also called the path integral, contour integral, or curve integral.}%
 Let $f$ be a function and let $C$ be a piecewise smooth curve whose parametrization is $\vec r(t)$ for $t\in[a,b]$. We'll require that the composition $f(\vec r(t))$ be continuous for all $t\in [a,b]$. Then we define the line integral
of $f$ over $C$ to be the integral 
$$\int_C f ds 
= \int_a^b f(\vec r(t))\frac{ds}{dt}dt
= \int_a^b f(\vec r(t))\left|\frac{d\vec r}{dt}\right|dt.$$
\end{definition}

Notice that this definition suggests the following four steps.  These four steps are the key to computing any line integral. \marginpar{When we ask you to set up a line integral, it means that you should do steps 1--3, so that you get an integral with a single variable and with bounds that you could plug into a computer or complete by hand.}%
\begin{enumerate}
 \item Start by getting a parametrization $\vec r(t)$ for $a\leq t\leq b$ of the curve $C$. 
 \item Find the speed by computing the velocity $\dfrac{d\vec r}{dt}$ and then the speed $\left|\dfrac{d\vec r}{dt}\right|$.
 \item Multiply $f$ by the speed, and replace each $x$, $y$, and/or $z$ with what it equals in terms of $t$.
 \item 
\marginpar{You should use the \href{\sagelineintegral}{Sage line integral calculator} to check all your answers.}%
Integrate the product from the previous step. Practice doing this by hand on every problem, unless it specifically says to use technology. Some of the integrals are impossible to do by hand.
\end{enumerate}

\note{We commented out a few problems here to cut down on the problems in this chapter.  They were originally put here to practice (a) parametrizing curves, (b) basic integration techniques, and (c) computing line integrals.  We'll try to practice these things in the problems in the rest of the chapter, though.}
%%%%%%%%%%%%%%%%%%%%%%%%%%%%%%%%%%%%%%%%%%%%%%%%%%IMPORTANT. SEE THE NOTE BELOW.
\note{There are 3 problems below that I used the first semester, but not the second.  I think it went way better when I included these problems.  They are really fast (just plug in numbers and compute), but they help the students get the hand of line integrals.  I suggest including them next time. I'll probably include these in the exact same problem, just have parts a,b,c.}

\begin{problem}\marginparbmw{See 16.1: 9-32.  Some problems give you a parametrization, some expect you to come up with one on your own.}%
\marginpar{\href{\sagelineintegral}{Check your answer with Sage.}}%
  Let $f(x,y,z)=x^2+y^2-2z$ and let $C$ be two coils of the helix $\vec r(t)=(3\cos t, 3\sin t, 4t)$, starting at $t=0$. Remember that the parameterization means $x=3\cos t$, $y=3\sin t$, and $z=4t$.  Compute $\int_Cf ds$. [You will have to find the end bound yourself. How much time passes to go around two coils?]
\end{problem}

\begin{problem}\marginparbmw{To practice matching parameterizations to curves, try 16.1:1-8.}%
\marginpar{\href{\sagelineintegral}{Check your answer with Sage.}}%
 Consider the function $f(x,y)=3xy+2$. Let $C$ be a circle of radius 4 centered at the origin.  Compute $\int_C fds$.  [You'll have to come up with your own parameterization.]
\end{problem}


\begin{problem}\marginpar{If you've forgotten how to parametrize line segments, see \ref{first line between two points}.}%
\marginpar{\href{\sagelineintegral}{Check your answer with Sage.}}%
 Let $f(x,y,z)=x^2+3yz$. Let $C$ be the straight line segment from $(1,0,0)$ to $(0,4,5)$. Compute $\int_C f ds$. 
\end{problem}

\begin{problem}\marginpar{See \ref{parameterizing plane curves} if you forgot how to parametrize plane curves.}%
\larsonfive{\marginpar{See Larson 15.2: 1--20, 63--70.  Some problems give you a parameterization, some expect you to come up with one on your own.}}%
\marginpar{\href{\sagelineintegral}{Check your answer with Sage.}}%
 Let $f(x,y)=x^2+y^2-25$. Let $C$ be the portion of the parabola $y^2=x$ between $(1,-1)$ and $(4,2)$. We want to compute $\int_C fds$.  
\begin{enumerate}
\item Draw the curve $C$ and the function $f(x,y)$ on the same 3D $xyz$ axes.
\item Without computing the line integral $\int_C fds$, determine if the integral should be positive or negative. Explain why this is so by looking at the values of $f(x,y)$ at points along the curve $C$.  Is $f(x,y)$ positive, negative, or zero, at points along $C$?
 \item Parametrize the curve and set up the line integral $\int_C f ds$. [Hint: if you let $y=t$, then $x=$? What bounds do you put on $t$?]
 \item Use technology to compute $\int_C fds$ to get a numeric answer.  Was your answer the sign that you determined above?
\end{enumerate}
\end{problem}






Work, flow, circulation, and flux are all examples of line integrals.  Remember that work, flow, and circulation are 
$$W=\int_C (\vec F\cdot \vec T)ds =\int_C (M,N)\cdot(dx,dy) =  \int_C Mdx+Ndy,$$
while the formula for flux is
$$\Phi=\int_C (\vec F\cdot \vec n)ds =\int_C (M,N)\cdot(dy,-dx) =  \int_C Mdy+Ndx.$$
Do you see how these are both the line integral of a function $f = \vec F\cdot \vec T$ or $f=\vec F\cdot \vec n$ along a curve $C$.  The function $f$ inside the integrand does not have to represent the height of a sheet. We'll use it to represent lots of things.  Let's practice two more work/flux problems, to sharpen our skills with these concepts. 






\begin{problem}
\marginpar{\href{http://www.youtube.com/watch?v=6WcN36FbeWc&list=PL04DF68E73B7ECD54&index=3&feature=plpp_video}{Watch a YouTube video}.  Also, see \href{http://aleph.sagemath.org/?z=eJw9yjEKhDAQRuF-TzKDvxLTp80lgohodMMORjZBnNtvqm0-XvE8PVB21CueTvl1Sa7zHdeav_Oeomzk0ZZ-GGGHkUH6b-6uLHrkk0IwsBOChWma5oQ9iTi_SImo77R-zliKs_wDRwwhgQ}{Sage} for a picture.}%
\instructor{Answers---\href{http://aleph.sagemath.org/?q=38b590f1-dc52-4690-975f-ccde49161dcf}{Sage}.
%http://aleph.sagemath.org/?z=eJy9Vd9r2zAQfs9fcaSByK2c1S2DURBsK8tT241SyIMbghvJiakjh7Ps2P_9TlKa2GnL3paHSLof3313J53rBNm44S0342AwFbVamgJZHLa8uWjnwaA6iC75JZ3rw_nKnXcd_RWdB2dg1qgUyCxNFSptIM-0glKtNnQoBxiJmNVhFZybi4oDM_ySR8F8gFck34W1lddd-TXJq3Bn5buOfODjLtJM5XLxUlRaliCAsYaHEb8OyLT1OyKFeVYaEWPE8YrjNXl_z7RRmCzNQKoUFuxWOJvgZgD0Q05xTGkSNByM0jIQt05xBnfKjEvYJK8KEkgrvTRZoQGz1dpAkaZgClB1kleJIQMNNswKkxyKWiGVJithWWGtJg7OxrblWbyZsf2GInomjo0yFWp4fGS62ijMlkl-dOh4TMrqpWSNwPhyzqGlNZpTHXp5xLZpFtTHR2H7xJAbL5UzMT1FOJdIed9Xucm2eZvpFZhdAb78JeW7UmZNua2yWtFxTa0vDGyxkBVV14LORD9HOetQoLYxE9jOSRuNQyjxQFJO86r5gBE5Ob1Tn4BbWQd_37anHz_vfrn92mzyiUlecsViJ4hxOHqmdIDi8DYYDTlM57yvQuJoFdhRmBFQ2Vdq6G5lv8wdM-kBrLfEU9wpO8D7wKepdoFmwvvA89KWuAs869iRGVXDttKtn_nMTrlokKVV-I50A_sueARrRd26B9lCCA8gG0fAmvS8nM_IM3F7t-2DONG9bMMHj2Lt5g7AThIaJX_ufj-9e6pUZ6NEPNSFptoP07zYuaVqhvP9q9mKbV6YRXdEsCmH8w9mhr8qaYH2MdNzBTcHOo_PdXffXD8KyPCgfveC_nmnHb8LAdsEk40y9JwXlizDD64Rp4GxfNWqLMX10TtLwddA-OSPZN9SyWwipbudrAtoQ9ASelnwJfoW9J0dOSB2CWKxY3QvRWZHKV7A5CsNgsALlkVeoBijkuMjLZX3iFE7_hsx28cesxV9gvSeW7PJNG82SSM-aD-1JY5u_KVrrWH7mWF0NCzXRGHLISm3ZLjAhL4BIuI2kDhGs3_cYoojsP0L_gLPyDNp
}%
Let $\vec F=(-y,x+y)$ and $C$ be the triangle with vertices $(2,0)$, $(0,2)$, and $(0,0)$.
\begin{enumerate}
\item Look at a drawing of $C$ and the vector field (see margin for the Sage link).  We'll move along the triangle in a counter clockwise manner. Without doing any computations, for each side of the triangle make a guess to determine if the flow along that edge is positive, negative, or zero. Similarly, guess the sign of the flux along each edge.  Explain. 
\item Obtain three parameterizations for the edges of the triangle.  One of the parameterizations is $\vec r(t) = (0,-2)t+(0,2)$. 
\item Now find the counterclockwise circulation (work) done by $\vec F$ along $C$.  You'll have three separate calculations, one for each side. We'll do the flux computation in class. Check your work on each piece with the \href{\sageworkfluxurl}{Sage calculator}. 
\end{enumerate}
\end{problem}

\begin{problem}\marginpar{See \href{http://aleph.sagemath.org/?z=eJxNjEEOQDAUBfdO8j-vSZWtrUs0ZUFFo0FopL09VmwnM9NSROKGVB5FQuRs91voLzuE7egnZ_1ILR5HKJQMSpComQvvVktaP1Qa6BKVMQizG5bVnmejuHg3VIvYqa_-CzduCCO3}{Sage}. Think of an airplane wing as you solve this problem.}%
\instructor{Answers---\href{http://aleph.sagemath.org/?q=ec0be342-7afb-4cb7-8af1-c5da03352d28}{Sage}.
%http://aleph.sagemath.org/?z=eJy9VUuP2zYQvutXDJwFTO5Sm5XdAEUAAm2D-rSbFkEAHxTX4JqULUQWjRHllf59h6RjS84GucUHkZzHN988SB8VsmkneuGmPFnIo9k4iyyf3XZpL7oVT9qzLJ2JBxIcz4JMzP05nNZFaSq9frZtrRuQwFgn0rmYcQGsF2km3nGeJJjJnB3Tlt-6u5Y0TjyIjK-SN9BJ93YOjQWHqt4aKBtgXpfgTOZnBnPyy8RvKYs7_t9sxQcwCVZl42SOmcAZHf8oa2dQbVyiTQFr9kEGA_4-AfqhIEfXOIVOgDO15vJDULyBR-OmDezVVwMKirbeuNLWgOV258AWBTgL5qiqVjkyqMGH2aKqwB4NgtsR-U2LR3Mf4HzsqqzN-psZO20oYmQS2BjXYg2fPrG63RssN6q6OAw87pv2uWGdxPxhJaCnNfM1GOWR-z550BgfpS6LgqFwUaqXcnGNcKuR8n5qK1ceqr6st-BeLMTCN5Tv1rgd5bYtj4aOOwPaOjig1S1V14Mu5ThHvRxQoJlgjvux0D6agFTjmaReVG33CiNyCvqgvgL3sgH-qW2f__zr8e-w37l9de_Uc2VYHgQ5Tm6-UDpAcUTPbyYCFisxViFx9AocKNwNhIGchDEbl3lgpiOA99Z4jbtgZ_gY-DrVIdBSRh_4svElHgIvB3ZkRtXwrQzrj3yW11xq0I1XxI4MA8cuRARvRd16At1DCh9Bd4GANxl5BZ-byCTsw3YMEkRPuk8_RhRvtwoA1P-Erv6_j_98_u6qUp2dkfmktjXVflJU9iUsbTdZnW7NQR4q69bD94ctBNy-8iDFUSks-stM1xXCOzC4fKG7p-bGp4AMz-rvbtBPZzrwu5NwUKj2xtF1XnuyDF8ZI0EPxuZrbZpGzi_eZQGxBjImfyH7LZXSJ9KE6WRDQB-CljTK-Nvsdz52DuSA2ClE-8JoLmXpn1G8g_t39BDwKNjYyqKcotHTCy1TjYhRO34ZMd_HEbMtGlOfuHX7shbdXnXylfZTW_LsfRy63hv2PzLMLobNjigcBKjmQIZrVPQfIDPhA8lLNP8RHlNegP2H_w9xpCjx
Again, we'll discuss why some are positive, and some are negative. I want to emphasize flow in, and flow out. One is clearly positive, the other clearly negative.  The overall sum is positive. It should be obvious with a picture.}%
 Consider the vector field $\vec F=(2x-y,x)$. Let $C$ be the curve that starts at $(-2,0)$, follows a straight line to $(1,3)$, and then back to $(-2,0)$ along the parabola $y=4-x^2$.  

\begin{enumerate}
\item Look at a drawing of $C$ and the vector field (see margin for the Sage link).  If we go counterclockwise around $C$, for each part of $C$, guess the signs of the counterclockwise circulation and the flux (positive, negative, zero).
\item Find the flux of $\vec F$ across $C$.  There are two curves to parametrize. Make sure you traverse along the curves in the correct direction. [Hint: You should get integer values along both parts. Check your work with \href{\sageworkfluxurl}{Sage}, but make sure you show us how to do the integrals by hand.]
\end{enumerate}
\end{problem}

Ask me in class to change the vector fields above, and examine what happens with different vectors fields.  In particular, it's possible to have any combination of values for circulation and flux. We'll be able to use technology to rapidly compute many values. 

\note{In both of the preceding problems, both the circulation and flux are positive.  Everything is always positive. This would give a student a false impression that work and flux are always positive.  We need to change some computations.  Make sure to look into this. It needs to change.}















\section{Average Value}

The concept of averaging values together has many applications.  In first-semester calculus, we saw how to generalize the concept of averaging numbers together to get an average value of a function.  We'll review both of these concepts. Later, we'll generalize average value to calculate centroids and center of mass.

\begin{problem}\label{average value methods}
\instructor{Throughout the section, I point out how each formula is a variation on one of the patterns below.}
  Suppose a class takes a test and there are three scores of 70, five scores of 85, one score of 90, and two scores of 95.  We will calculate the average class score, $\bar s$, four different ways to emphasize four ways of thinking about the averages.  We are emphasizing the pattern of the calculations in this problem, rather than the final answer, so it is important to write out each calculation completely in the form $\bar s = \blank{1cm}$ before calculating the number $\bar s$.
  \begin{enumerate}
  \item \marginpar{$\bar s=\frac{\sum \text{values}}{\text{number of values}}$}%
 Compute the average by adding 11 numbers together and dividing by the number of scores.  Write down the whole computation before doing any arithmetic.
  \item \marginpar{$\bar s=\frac{\sum (\text{value}\cdot\text{weight})}{\sum \text{weight}}$}%
Compute the numerator of the fraction in the previous part by multiplying each score by how many times it occurs, rather than adding it in the sum that many times.  Again, write down the calculation for $\bar s$ before doing any arithmetic.
  \item \marginpar{$\bar s=\sum (\text{value}\cdot\text{(\% of stuff)})$}%
Compute $\bar s$ by splitting up the fraction in the previous part into the sum of four numbers.  This is called a ``weighted average'' because we are multiplying each score value by a weight.
  \item \marginpar{$\text{(number of values)}\bar s = \sum \text{values}$\\ $(\sum \text{weight})\bar s = \sum (\text{value}\cdot\text{weight})$}%
Another way of thinking about the average $\bar s$ is that $\bar s$ is the number so that if all 11 scores were the same value $\bar s$, you'd have the same sum of scores.  Write this way of thinking about these computations by taking the formulas for $\bar s$ in the first two parts and multiplying both sides by the denominator.
  \end{enumerate}
\end{problem}

In the next problem, we generalize the above ways of thinking about averages from a discrete situation to a continuous situation.  You did this in first-semester calculus when you did average value using integrals.

\begin{problem}
 Suppose the price of a stock is \$10 for one day.  Then the price of the stock jumps to \$20 for two days.  Our goal is to determine the average price of the stock over the three days.
\begin{enumerate}
 \item Why is the average stock price not \$15?
%\item Find the average price of the stock using all of the methods from Problem~\ref{average value methods}.
 \item Let $f(t) = \begin{cases}10 &0<t<1\\20&1<t<3\end{cases}$, the price of the stock for the three-day period. Draw the function $f$, and find the area under $f$ where $t\in[0,3]$.
 \item Now let $y=\bar f$.  The area under $\bar f$ over $[0,3]$ is simply width times height, or $(b-a)\bar f$. What should $\bar f$ equal so that the area under $\bar f$ over $[0,3]$ matches the area under $f$ over $[0,3]$.
 \item We found a constant $\bar f$ so that the area under $\bar f$ matched the area under $f$. In other words, we solved the equation below for $\bar f$: 
$$\int_a^b \bar f dx = \int_a^b f dx$$
  Solve for $\bar f$ symbolically (without doing any of the integrals). This quantity is called the average value of $f$ over $[a,b]$.%, and was crucial to proving the Fundamental Theorem of Calculus from first-semester calculus.
\item \instructor{I also write $\bar f=\int_a^b f \frac{dx}{\int_a^b dx}$ to emphasize the weighted average approach}%
 The formula for $\bar f$ in the previous part resembles at least one of the ways of calculating averages from Problem~\ref{average value methods}.  Which ones and why?
\end{enumerate}
\end{problem}

\instructor{I talk about ants building a mound. Then after removing the ants, so none get hurt, you shake their tank.  The average value is the height of the dirt. The mountains filled in the valleys.}%
Ask me in class about the ``ant farm'' approach to average value. 

\begin{problem}\label{Average Value intro}%
\marginpar{\href{http://www.youtube.com/watch?v=t7T0MzfgV0Q&list=PL04DF68E73B7ECD54&index=5&feature=plpp_video}{Watch a YouTube video.}}%
 Consider the elliptical curve $C$ given by the parametrization $\vec r(t) = (2\cos t, 3\sin t)$.  Let $f$ be the function $f(x,y)=9-x^2-y^2$.    \begin{enumerate}
  \item Draw the surface $f$ in 3D.  Add to your drawing the curve $C$ in the $xy$ plane. Then draw the sheet whose area is given by the integral $\int_C f ds$. 
  \item What's the maximum height and minimum height of the sheet? \marginpar{See problem \ref{horse track chain rule introduction}.}
  \item We'd like to find a constant height $\bar f$ so that the area under $f$, above $C$, is the same as area under $\bar f$, above $C$. This height $\bar f$ is called the average value of $f$ along $C$. 
%  \item 
\marginparbmw{Please read 
\href{https://www.lds.org/scriptures/ot/isa/40.4?lang=eng\#3}{Isaiah 40:4} and
\href{https://www.lds.org/scriptures/nt/luke/3.5?lang=eng\#4}{Luke 3:5}. These scriptures should help you remember how to find average value.
}%
\instructor{Again, I emphasize how this relates to the ways of computing averages from Problem~\ref{average value methods}.}
Explain why the average value of $f$ along $C$ is 
$$\bar f = \frac{\int_C f ds}{\int_C ds}.$$
Connect this formula with the ways of thinking about averages from Problem~\ref{average value methods}.
[Hint: The area under $\bar f$ above $C$ is $\int_C \bar f ds$. The area under $f$ above $C$ is $\int_C f ds$. Set them equal and solve for $\bar f$. ]  
  \item 
\marginpar{You can use the \href{\sagelineintegral}{Sage line integral calculator}.}%
Use a computer to evaluate the integrals $\int_C f ds$ and $\int_C ds$, and then give an approximation to the average value of $f$ along $C$. Is your average value between the maximum and minimum of $f$ along $C$? Why should it be?
 \end{enumerate}
\end{problem}

\begin{problem}\instructor{After this problem, I like to emphasize that they should have noticed a linear growth rate, and then I show them how I would have guessed the answer.}%
 The temperature $T(x,y,z)$ at points on a wire helix $C$ given by $\vec r(t) = (\sin t, 2t, \cos t)$ is known to be $T(x,y,z)=x^2+y+z^2$. What are the temperatures at $t=0$, $t=\pi/2$, $t=\pi$, $t=3\pi/2$ and $t=2\pi$?  You should notice the temperature is constantly changing.  Make a guess as to what the average temperature is (share with the class why you made the guess you made---it's OK if you're wrong). Then compute the average temperature of the wire using the integral formula from the previous problem. You can do all these computations by hand.
\end{problem}




























\section{Physical Properties}

A number of physical properties of real-world objects can be calculated using the concepts of averages and line integrals.  We explore some of these in this section.  Additionally, many of these concepts and calculations are used in statistics.

\subsection{Centroids}%

\begin{definition}[Centroid]
  Let $C$ be a curve. If we look at all of the $x$-coordinates of the points on $C$, the ``center'' $x$-coordinate, $\bar x$, is the average of all these $x$-coordinates.  Likewise, we can talk about the averages of all of the $y$ coordinates or $z$ coordinates of points on the function ($\bar y$ or $\bar z$, respectively).  The \emph{centroid} of an object is the geometric center $(\bar x, \bar y, \bar z)$, the point with coordinates that are the average $x$, $y$, and $z$ coordinates.
\end{definition}

\begin{problem}[Centroid]\label{centroid of curve}\marginpar{\href{http://www.youtube.com/watch?v=t7T0MzfgV0Q&list=PL04DF68E73B7ECD54&index=5&feature=plpp_video}{Watch a YouTube video.}}%
  Notice the word ``average'' in the definition of the centroid. Use the concept of average value to explain why the coordinates of the centroid are %the formulas below. [Hint:  If we have a curve $C$ with parametrization $\vec r(t)$ and function $f$ so that $f(\vec r(t))$ is continuous, then we've developed in Problem~\ref{Average Value intro} a formula for the average value of $f$ along $C$.  What function $f(x,y,z)$ gives the $x$-coordinate of a point?]
\marginpar{These are the formulas for the centroid.}
$$
\bar x = \frac{\int_C x ds}{\int_C  ds},\quad
\bar y = \frac{\int_C y ds}{\int_C  ds},\quad 
\text{and}\quad
\bar z = \frac{\int_C z ds}{\int_C  ds}.
$$
Notice that the denominator in each case is just the arc length $s=\int_C ds$. 
\end{problem}


\begin{problem}\label{semicircle centroid}
 Let $C$ be the semicircular arc $\vec r(t)=(a\cos t, a\sin t)$ for $t\in[0,\pi]$. Without doing any computations, make an educated guess for the centroid $(\bar x, \bar y)$ of this arc.  Then compute the integrals given in problem \ref{centroid of curve} to find the actual centroid. Share with the class your guess, even if it was incorrect. 
\end{problem}

\subsection{Mass and Center of Mass}
\note{Jason made a comment that aluminum and copper won't combine.  So this should be changed eventually.}
Density is generally a mass per unit volume.  However, when talking about a curve or wire, as in this chapter, it's simpler to let density be the mass per unit length.  Sometimes an object is made out of a composite material, and the density of the object is different at different places in the object. For example, we might have a straight wire where one end is aluminum and the other end is copper. In the middle, the wire slowly transitions from being all aluminum to all copper.  The centroid is the midpoint of the wire.  However, since copper has a higher density than aluminum, the balance point (the center of mass) would not be at the midpoint of the wire, but would be closer to the denser and heavier copper end.  In this section, we'll develop formulas for the mass and center of mass of such a wire. Such composite materials are engineered all the time (though probably not our example wire).  \bmw{In future mechanical engineering courses, you would learn how to determine the density $\delta$ (mass per unit length) at each point on such a composite wire.}

\begin{problem}[Mass]\label{mass of curve}%
\marginpar{\href{http://www.youtube.com/watch?v=mz-Udq5TeS4&list=PL04DF68E73B7ECD54&index=6&feature=plpp_video}{Watch a YouTube video.}}%
 Suppose a wire $C$ has the parameterization $\vec r(t)$ for $t\in[a,b]$.  Suppose the wire's density at a point $(x,y,z)$ on the wire is given by the function $\delta(x,y,z)$. \bmw{You'll learn to calculate this function in a future class. For the purposes of our class, we'll just assume we know what $\delta(x,y,z)$ is.}
 \begin{enumerate}
  \item Consider a small portion of the curve at $t=t_0$ of length $ds$.  Explain why the mass of the small portion of the curve is $dm=\delta(\vec r(t_0)) ds$.
  \item Explain why the mass $m$ of an object is given by the formulas below (explain why each equals sign is true):
$$m=\int_C dm = \int_C \delta ds = \int_a^b \delta(\vec r(t)) \left|\frac{d\vec r}{dt}\right|dt.$$
 \end{enumerate}
\end{problem}

\begin{problem}\instructor{I found many students struggled with setting up a really simple sum.  In class, after they present this one, I would suggest actually taking time to show them how to write the problem in summation notation with 2 points.  It will prepare them for the proof of center of mass coming up.}%
\larsonfive{\marginpar{See Larson 15.2:23--26 for more practice.}}%
 A wire lies along the straight segment from $(0,2,0)$ to $(1,1,3)$.  The wire's density (mass per unit length) at a point $(x,y,z)$ is $\delta(x,y,z)=x+y+z$. 
 \begin{enumerate}
 \item Is the wire heavier at $(0,2,0)$ or at $(1,1,3)$?
 \item  What is the total mass of the wire?  [You'll need to parameterize the line as your first step---see Problem~\ref{first line between two points} if you need a refresher.]
 \end{enumerate}
\end{problem}

\instructor{Here I introduce the center of mass of an object, talk about moments, and talk about how the center of mass formula is also an averaging of the coordinates, where the weight is the amount of mass at a particular coordinate value (i.e., $xdm$).}

\marginpar{\href{http://en.wikipedia.org/wiki/Center_of_mass}{Wikipedia} has some interesting applications of center of mass.}%
The center of mass of an object is the point where the object balances.  In order to calculate the $x$-coordinate of the center of mass, we average the $x$-coordinates, but we weight each $x$-coordinate with its mass.  Similarly, we can calculate the $y$ and $z$ coordinates of the center of mass.

The next problem helps us reason about the center of mass of a collection of objects.  Calculating the center of mass of a collection of objects is important, for example, in astronomy when you want to calculate how two bodies orbit each other.

\begin{problem}\label{center of mass with two points}\instructor{After a student presents, this is a great time to connect the averages back to Problem~\ref{average value methods}.  I point out that we can think about the object as 2 points of the same mass at $P_1$ and 3 points of the same mass at $P_2$.  This suggests averaging 5 things with method 1.  Alternately, I suggest the approach $\frac{2}{5}P_1+\frac{3}{5}P_2$, suggesting a weighted average.}%
 Suppose two objects are positioned at the points $P_1=(x_1,y_1,z_1)$ and $P_2=(x_2,y_2,z_2)$.
 Our goal in this problem is to understand the difference between the centroid and the center of mass.
\begin{enumerate}
\item Find the centroid of two objects.
 \item Suppose both objects have the same mass of 2 kg.  Find the center of mass.% by averaging the $x$, $y$, and $z$ coordinates, weighted by how much mass is at each coordinate.
 \item If the mass of the object at point $P_1$ is 2 kg, and the mass of the object at point $P_2$ is 5 kg, will the center of mass be closer to $P_1$ or $P_2$? Give a physical reason for your answer before doing any computations.  Then find the center of mass $(\bar x, \bar y, \bar z)$ of the two points. [Hint: You should get $\bar x= \frac{2x_1+5x_2}{2+5}$.] 
\end{enumerate}
\end{problem}


\begin{problem}
 This problem reinforces what you just did with two points in the previous problem. However, it now involves two people on a seesaw. \marginpar{See \href{http://en.wikipedia.org/wiki/Seesaw}{Wikipedia} for a seesaw picture.}%
Ignore the mass of the seesaw in your work below (pretend it's an extremely light seesaw, so its mass is negligible compared to the masses of the people).
\begin{enumerate}
 \item 
 My daughter and her friend are sitting on a seesaw.  Both girls have the same mass of 30 kg. My wife stands about 1 m behind my daughter. We'll measure distance in this problem from my wife's perspective.  We can think of my daughter as a point mass located at $(1\text{m},0)$ whose mass is $30$ kg. Suppose her friend is located at $(5\text{m},0)$. Suppose the kids are sitting just right so that the seesaw is perfectly balanced.  That means the the center of mass of the girls is precisely at the pivot point of the seesaw. Find the distance from my wife to the pivot point by finding the center of mass of the two girls. 
 \item My daughter's friend has to leave, so I plan to take her place on the seesaw. My mass is 100 kg. Her friend was sitting at the point $(5,0)$. I would like to sit at the point $(a,0)$ so that the seesaw is perfectly balanced. Without doing any computations, is $a>5$ or $a<5$? Explain.
 \item Suppose I sit at the spot $(x,0)$ (perhaps causing my daughter or I to have a highly unbalanced ride). Find the center of mass of the two points $(1,0)$ and $(x,0)$ whose masses are $30$ and $100$, respectively (units are meters and kilograms). 
 \item Where should I sit so that the seesaw is perfectly balanced (what is $a$)?
\end{enumerate}
\end{problem}

\begin{problem}[Center of mass]\label{center of mass of curve}%
\marginpar{\href{http://www.youtube.com/watch?v=mz-Udq5TeS4&list=PL04DF68E73B7ECD54&index=6&feature=plpp_video}{Watch a YouTube video.}}%
In problem \ref{center of mass with two points}, we focused on a system with two points $(x_1,y_1)$ and $(x_2,y_2)$ with masses $m_1$ and $m_2$. The center of mass in the $x$ direction is given by  
$$
\bar x = \frac{x_1m_1+x_2m_2}{m_1+m_2} = \frac{\sum_{i=1}^2x_i m_i}{\sum_{i=1}^2m_i}$$
\begin{enumerate}
\item If we consider a system with 3 points, what formula gives the center of mass in the $x$ direction? What if there are 4 points, 5 points, or $n$ points?  
%\item Consider a system with $n$ points labeled $(x_1, y_1,z_1)$, $(x_2, y_2,z_2)$, \ldots, $(x_n, y_n,z_n)$, having masses $m_1$, $m_2$, \ldots, $m_n$ respectively. Give a formula for the center of mass in the $y$ direction (the $x$ and $z$ directions are similar).
 \item Suppose now that we have a wire located along a curve $C$. The density of the wire is known to be $\delta(x,y,z)$ (which could be different at different points on the curve).  Imagine cutting the wire into a thousand or more tiny chunks.  Each chunk would be centered at some point $(x_i,y_i,z_i)$ and have length $ds_i$. Explain why the mass of each little chunk is $dm_i\approx\delta ds_i$. 
 \item Give a formula for the center of mass in the $y$ direction of the thousands of points $(x_i,y_i,z_i)$, each with mass $dm_i$. [This should almost be an exact copy of the first part.] 
 Then explain why $$\bar y = \frac{\int_C y dm}{\int_C dm}=\frac{\int_C y \delta ds}{\int_C \delta ds}.$$
\end{enumerate}
\end{problem}

Ask me in class to show you another way to obtain the formula for center of mass. It involves looking at masses weighted by their distance (called a moment of mass).  Many of you will have already seen an idea similar to this in statics, but in that class you are talking about moments of force, not moments of mass. 


For quick reference, the formulas for the centroid of a wire along $C$ are
$$
\bar x = \frac{\int_C x ds}{\int_C  ds},\quad
\bar y = \frac{\int_C y ds}{\int_C  ds},\quad 
\text{and}\quad
\bar z = \frac{\int_C z ds}{\int_C  ds}.  \quad\text{(Centroid)}
$$
If the wire has density $\delta$, then the formulas for the center of mass are 
\marginpar{The quantity $\int_C x dm$ is sometimes called the first moment of mass about the $yz$-plane (so $x=0$). Notationally, some people write $M_{yz} =\int_C x ds$. Similarly, we could write $M_{xz}=\int_C y dm$ and $M_{xy}=\int_C zdm$.  With this notation, we could write the center of mass formulas as 
$$(\bar x,\bar y,\bar z) 
= 
\left(
\frac{M_{yz}}{m},
\frac{M_{xz}}{m},
\frac{M_{xy}}{m}
\right)
.
$$ }%
$$
\bar x = \frac{\int_C x dm}{\int_C  dm},\quad
\bar y = \frac{\int_C y dm}{\int_C  dm},\quad 
\text{and}\quad
\bar z = \frac{\int_C z dm}{\int_C  dm},  \quad\text{(Center of mass)}
$$
where $dm=\delta ds$. Notice that the denominator in each case is just the mass $m=\int_C dm$.

We'll often use the notation $(\bar x, \bar y,\bar z)$ to talk about both the centroid and the center of mass. If no density is given in a problem, then $(\bar x, \bar y,\bar z)$ is the centroid. If a density is provided, then $(\bar x, \bar y,\bar z)$ refers to the center of mass. If the density is constant, it doesn't matter (the centroid and center of mass are the same, which is what the seesaw problem showed).

\begin{problem}
\instructor{I purposefully put this problem in to show students how to generalize a formula from prime numbers to any number.  I mention how I use primes (5, 7, 11, 13) when I'm looking for a pattern.  I want to help them develop this skill a little. However, if you are pressed for time, then skip 5.}%
Suppose a wire with density $\delta(x,y)=x^2+y$ lies along the curve $C$ which is the upper half of a circle around the origin with radius $7$.
\begin{enumerate}
\item Parametrize $C$ (find $\vec r(t)$ and the domain for $t$).
 \item Where is the wire heavier, at $(7,0)$ or $(0,7)$? [Compute $\delta$ at both.]
 \item In problem \ref{semicircle centroid}, we showed that the centroid of the wire is $(\bar x, \bar y) = \left(0,\frac{2(7)}{\pi})\right)$.  We now seek the center of mass. Before computing, will $\bar x$ change?  Will $\bar y$ change?  How will each change? Explain.
 \item Set up the integrals needed to find the center of mass. Then use technology to compute the integrals. Give an exact answer (involving fractions), rather than a numerical approximation.
 \item Change the radius from 7 to $a$, and guess what the center of mass will be.  (This is why you need the exact answer above, not a numerical answer).
\end{enumerate}
\end{problem}









\section{The Fundamental Theorem of Line Integrals}

In this final section we'll return to the concept of work. Many vector fields are actually the derivative of a function.  When this occurs, computing work along a curve is extremely easy.  All you have to know is the endpoints of the curve, and the function $f$ whose derivative gives you the vector field. This function is called a potential for a vector field.  Once we are comfortable finding potentials, we'll show that the work done by such a vector field is the difference in the potential at the end points.  This makes finding work extremely fast.

\begin{definition}[Gradients and Potentials]
\marginpar{\href{http://www.youtube.com/watch?v=8Tk2pEIOnwg&list=PL04DF68E73B7ECD54&index=9&feature=plpp_video}{Watch a YouTube Video}.}%
 Let $\vec F$ be a vector field.  A potential for the vector field is a function $f$ whose derivative equals $\vec F$. So if $Df=\vec F$, then we say that $f$ is a potential for $\vec F$. When we want to emphasize that the derivative of $f$ is a vector field, we call $Df$ the gradient of $f$ and write $Df = \vec \nabla f$.
\marginpar{The symbol $\vec \nabla f$ is read ``the gradient of $f$'' or ``del f.''}
 If $\vec F$ has a potential, then we say that $\vec F$ is a gradient field. 
\end{definition}

We'll quickly see that if a vector field has a potential, then the work done by the vector field is the difference in the potential.  If you've ever dealt with kinetic and potential energy, then you hopefully recall that the change in kinetic energy is precisely the difference in potential energy.  This is the reason we use the word ``potential.''

\begin{problem}
\marginpar{\href{http://www.youtube.com/watch?v=8Tk2pEIOnwg&list=PL04DF68E73B7ECD54&index=9&feature=plpp_video}{Watch a YouTube Video}.}%
Let's practice finding gradients and potentials.
\begin{enumerate}
 \item  Let $f(x,y) = x^2+3xy+2y^2$. Find the gradient of $f$, i.e. find $Df(x,y)$. Then compute $D^2f(x,y)$ (you should get a square matrix). What are $f_{xy}$ and $f_{yx}$?
 \item Consider the vector field $\vec F(x,y)=(2x+y,x+4y)$. Find the derivative of $\vec F(x,y)$ (it should be a square matrix). Then find a function $f(x,y)$ whose gradient is $\vec F$ (i.e. $Df=\vec F$). What are $f_{xy}$ and $f_{yx}$?
 \item \marginpar{See problem \ref{second partials agree}.}%
Consider the vector field $\vec F(x,y)=(2x+y,3x+4y)$.  Find the derivative of $\vec F$.  Why is there no function $f(x,y)$ so that $Df(x,y)=\vec F(x,y)$? [Hint: what would $f_{xy}$ and $f_{yx}$ have to equal?] 
\end{enumerate}
\end{problem}

Based on your observations in the previous problem, we have the following key theorem.

\begin{theorem}
 Let $\vec F$ be a vector field that is everywhere continuously differentiable. Then $\vec F$ has a potential if and only if the derivative $D\vec F$ is a symmetric matrix. We say that a matrix is symmetric if interchanging the rows and columns results in the same matrix (so if you replace row 1 with column 1, and row 2 with column 2, etc., then you obtain the same matrix).  
\end{theorem}

\begin{problem}
\marginpar{If you haven't yet, please watch this \href{http://www.youtube.com/watch?v=8Tk2pEIOnwg&list=PL04DF68E73B7ECD54&index=9&feature=plpp_video}{YouTube video}.}%
For each of the following vector fields, find a potential, or explain why none exists.
\begin{enumerate}
 \item $\vec F(x,y)=(2x-y, 3x+2y)$
 \item $\vec F(x,y)=(2x+4y, 4x+3y)$
 \item $\vec F(x,y)=(2x+4xy, 2x^2+y)$
 \item $\vec F(x,y,z)=(x+2y+3z,2x+3y+4z,2x+3y+4z)$
 \item $\vec F(x,y,z)=(x+2y+3z,2x+3y+4z,3x+4y+5z)$
 \item $\vec F(x,y,z)=(x+yz,xz+z,xy+y)$
 \item $\ds \vec F(x,y) = \left(\frac{x}{1+x^2}+\arctan (y),\frac{x}{1+y^2}\right)$
\end{enumerate}
\end{problem}


If a vector field has a potential, then there is an extremely simple way to compute work. To see this, we must first review the fundamental theorem of calculus. The second half of the fundamental theorem of calculus states,
\begin{quote}
 If $f$ is continuous on $[a,b]$ and $F$ is an anti-derivative of $f$, then $F(b)-F(a) = \int_a^b f(x) dx$.
\end{quote}
If we replace $f$ with $f'$, then an anti-derivative of $f'$ is $f$, and we can write,
\begin{quote}
 If $f$ is continuously differentiable on $[a,b]$, then $$f(b)-f(a)=\int_a^b f'(x) dx.$$
\end{quote}
This last version is the version we now generalize.

\begin{theorem}[The Fundamental Theorem of Line Integrals]
\marginpar{\href{http://www.youtube.com/watch?v=5ZsCN6NN3yg&list=PL04DF68E73B7ECD54&index=11&feature=plpp_video}{Watch a YouTube video}.}%
 Suppose $f$ is a continuously differentiable function, defined along some open region containing the smooth curve $C$. Let $\vec r(t)$ be a parametrization of the curve $C$ for $t\in[a,b]$. Then we have
$$f(\vec r(b))-f(\vec r(a))=\int_a^b Df(\vec r(t))D\vec r(t)\ dt.$$
\end{theorem}
Notice that if $\vec F$ is a vector field, and has a potential $f$, which means $\vec F = Df$, then we could rephrase this theorem as follows. 
\begin{quote}
 Suppose $\vec F$ is a a vector field that is continuous along some open region containing the curve $C$. Suppose $\vec F$ has a potential $f$. Let $A$ and $B$ be the start and end points of the smooth curve $C$.  Then the work done by $\vec F$ along $C$ depends only on the start and end points, and is precisely
$$f(B)-f(A)=\int_C \vec F\cdot d\vec r = \int_C Mdx+Ndy.$$
 The work done by $\vec F$ is the difference in a potential.
\end{quote}
If you are familiar with kinetic energy, then you should notice a key idea here.  Work is a transfer of energy. As an object falls, energy is transferred from potential energy to kinetic energy.  The total kinetic energy at the end of a fall is precisely equal to the difference between the potential energy at the top of the fall and the potential energy at the bottom of the fall (neglecting air resistance). So work (the transfer of energy) is exactly the difference in potential energy.  

\begin{problem}[Proof of Fundamental Theorem]\marginpar{The proof of the fundamental theorem of line integrals is quite short. All you need is the fundamental theorem of calculus, together with the chain rule (\ref{chain rule def}).}
 Suppose $f(x,y)$ is continuously differentiable, and suppose that $\vec r(t)$ for $t\in[a,b]$ is a parametrization of a smooth curve $C$. Prove that $f(\vec r(b))-f(\vec r(a)) = \int_a^b Df(\vec r(t))D\vec r(t)\ dt$. [Let $g(t) = f(\vec r(t))$. Why does $g(b)-g(a) = \int_a^b g'(t)dt$? Use the chain rule (matrix form) to compute $g'(t)$. Then just substitute things back in.]  
\end{problem}

\begin{problem}
\marginpar{\href{http://www.youtube.com/watch?v=5ZsCN6NN3yg&list=PL04DF68E73B7ECD54&index=11&feature=plpp_video}{Watch a YouTube video}.}%
For each vector field and curve below, find the work done by $\vec F$ along $C$. In other words, compute the integral $\int_C Mdx+Ndy$ or $\int_C Mdx+Ndy+Pdz$. 
\begin{enumerate}
 \item\marginpar{See \href{http://aleph.sagemath.org/?z=eJxz06jQqdS01TDSqtCu1KnQNtGq1OQqyMkviS9LTS7JL4pPy0zNSdFw01EAKtQ11jHSBLIqdQx0LDU1tUHqNCx1K-KMdGCymgDJPhZ0}{Sage} for a picture.}%
 Let $\vec F(x,y) = (2x+y,x+4y)$ and $C$ be the parabolic path $y=9-x^2$ for $x$ from $-3$ to $2$.
 \item\marginpar{See \href{http://aleph.sagemath.org/?z=eJwVi7EKgDAMRPd-hWPSRtCIo6s_IdJBKwhFRYo0_XrT6R5372bIJFRwArbZiS3Etris2bAVBUHzxDv5L2zpfv1xhrgPO8zU6LMnRgWhdqwEhToaEF08rwALcO27aqiow4rmBwN1HWM}{Sage} for a picture.}%
 Let $\vec F(x,y,z) = (2x+yz,2z+xz,2y+xy)$ and $C$ be the straight segment from $(2,-5,0)$ to $(1,2,3)$. 
\end{enumerate}
[Hint: If you parametrize the curve, then you've done the problem the HARD way. You don't need any parameterizations at all. Did you find a potential, and then plug in the end points?]
\end{problem}

\begin{problem}\marginpar{See \href{http://aleph.sagemath.org/?z=eJytkM1uhCAUhfc-hTsueMmMYNOV23mJydQYpR1SK0TItPr0hVFbm5kumnQBl597zvngUg9APKHJAT5wxImWoU440sRY78pWNx78WTevvXKulDSxpe2Mry6q8WaonrXqWtnCAdOg4zkKGlYj8gILijAhl_hIgyorbT3Ub8oPuqmiA0BjHPjQ7nR_rXGAxz0KZjVFxiLBVdrpXsER8q-7FAQWKOkJ07UtzkfSmM4M5FSSQbVkI53bgy567Le6e2SCr0wFmykZ5Dzu5bJfUa3eiR-ov3oxwfzO6tVSsm-7B_YPhuFgLllnXiDPIrPns92SEzPnhzyJP-RlMe-qvX3CbWa-4G0-a3N4P9adzTtYmnwCd8u7lQ}{Sage}---$C_1$ and $C_2$ are in blue, and several possible $C_3$ are shown in red.}%
  Let $\vec F = (x,z,y)$. Let $C_1$ be the curve which starts at $(1,0,0)$ and follows a helical path $(\cos t, \sin t, t)$ to $(1,0,2\pi)$. Let $C_2$ be the curve which starts at $(1,0, 2\pi)$ and follows a straight line path to $(2,4,3)$. Let $C_3$ be any smooth curve that starts at $(2,4,3)$ and ends at $(0,1,2)$.
 \begin{itemize}
  \item Find the work done by $\vec F$ along each path $C_1$, $C_2$, $C_3$. \marginpar{If you are parameterizing the curves, you're doing this the really hard way. Are you using the potential of the vector field?}
  \item Find the work done by $\vec F$ along the path $C$ which follows $C_1$, then $C_2$, then $C_3$.  
  \item If $C$ is any path that can be broken up into finitely many smooth sub-paths, and $C$ starts at $(1,0,0)$ and ends at $(0,1,2)$, what is the work done by $\vec F$ along $C$?
 \end{itemize}
\end{problem}

In the problem above, the path we took to get from one point to another did not matter. The vector field had a potential, which meant that the work done did not depend on the path traveled. 
\begin{definition}[Conservative Vector Field]
 We say that a vector field is conservative if the integral $\int_C \vec F\cdot d\vec r$ does not depend on the path $C$. We say that a curve $C$ is piecewise smooth if it can be broken up into finitely many smooth curves.
\end{definition}
 
\begin{review*}
 Compute $\ds \int \frac{x}{\sqrt{x^2+4}}dx$. See \footnote{
Let $u=x^2+4$, which means $du=2xdx$ or $dx=\frac{du}{2x}$.  This means
$$\ds \int \frac{x}{\sqrt{x^2+4}}dx 
= \int \frac{x}{\sqrt{u}}\frac{du}{2x} 
= \frac{1}{2}\int u^{-1/2}du
= \frac{1}{2}\frac{u^{1/2}}{1/2}
= \sqrt{u} = \sqrt{x^2+4}.
$$
}.
\end{review*}


\begin{problem}
 The gravitational vector field is directly related to the radial field $\ds\vec F = \frac{\left(-x,-y,-z\right)}{(x^2+y^2+z^2)^{3/2}}$. Show that this vector field is conservative, by finding a potential for $\vec F$.  Then compute the work done by an object that moves from $(1,2,-2)$ to $(0,-3,4)$ along ANY path that avoids the origin. 

[See the review problem just before this if you're struggling with the integral.]
 %[While $\vec F$ is not continuously differentiable everywhere (it's not at the origin), we can still use the fundamental theorem of line integrals if we stay away from the origin. This shows that $\vec F$ is a conservative vector field, provided we dodge the origin.]
\end{problem}

%If a vector field is everywhere continuously differentiable, then the last problem shows that conservative vector fields and gradient fields are exactly the same.

\begin{problem}
 Suppose $\vec F$ is a gradient field.  Let $C$ be a piecewise smooth closed curve. Compute $\int_C \vec F\cdot d\vec r$ (you should get a number). Explain how you know your answer is correct.
\end{problem}




